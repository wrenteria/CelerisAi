%% Generated by Sphinx.
\def\sphinxdocclass{report}
\documentclass[letterpaper,10pt,english]{sphinxmanual}
\ifdefined\pdfpxdimen
   \let\sphinxpxdimen\pdfpxdimen\else\newdimen\sphinxpxdimen
\fi \sphinxpxdimen=.75bp\relax
\ifdefined\pdfimageresolution
    \pdfimageresolution= \numexpr \dimexpr1in\relax/\sphinxpxdimen\relax
\fi
%% let collapsible pdf bookmarks panel have high depth per default
\PassOptionsToPackage{bookmarksdepth=5}{hyperref}

\PassOptionsToPackage{booktabs}{sphinx}
\PassOptionsToPackage{colorrows}{sphinx}

\PassOptionsToPackage{warn}{textcomp}
\usepackage[utf8]{inputenc}
\ifdefined\DeclareUnicodeCharacter
% support both utf8 and utf8x syntaxes
  \ifdefined\DeclareUnicodeCharacterAsOptional
    \def\sphinxDUC#1{\DeclareUnicodeCharacter{"#1}}
  \else
    \let\sphinxDUC\DeclareUnicodeCharacter
  \fi
  \sphinxDUC{00A0}{\nobreakspace}
  \sphinxDUC{2500}{\sphinxunichar{2500}}
  \sphinxDUC{2502}{\sphinxunichar{2502}}
  \sphinxDUC{2514}{\sphinxunichar{2514}}
  \sphinxDUC{251C}{\sphinxunichar{251C}}
  \sphinxDUC{2572}{\textbackslash}
\fi
\usepackage{cmap}
\usepackage[T1]{fontenc}
\usepackage{amsmath,amssymb,amstext}
\usepackage{babel}



\usepackage{tgtermes}
\usepackage{tgheros}
\renewcommand{\ttdefault}{txtt}



\usepackage[Bjarne]{fncychap}
\usepackage{sphinx}

\fvset{fontsize=auto}
\usepackage{geometry}


% Include hyperref last.
\usepackage{hyperref}
% Fix anchor placement for figures with captions.
\usepackage{hypcap}% it must be loaded after hyperref.
% Set up styles of URL: it should be placed after hyperref.
\urlstyle{same}

\addto\captionsenglish{\renewcommand{\contentsname}{Contents:}}

\usepackage{sphinxmessages}
\setcounter{tocdepth}{1}

\DeclareUnicodeCharacter{200B}{}

\title{CelerisAi}
\date{Apr 05, 2025}
\release{0.0.1}
\author{Willington Rentería}
\newcommand{\sphinxlogo}{\vbox{}}
\renewcommand{\releasename}{Release}
\makeindex
\begin{document}

\ifdefined\shorthandoff
  \ifnum\catcode`\=\string=\active\shorthandoff{=}\fi
  \ifnum\catcode`\"=\active\shorthandoff{"}\fi
\fi

\pagestyle{empty}
\sphinxmaketitle
\pagestyle{plain}
\sphinxtableofcontents
\pagestyle{normal}
\phantomsection\label{\detokenize{index::doc}}


\sphinxAtStartPar
\sphinxstylestrong{CelerisAi} is Python\sphinxhyphen{}Taichi\sphinxhyphen{}based software for nearshore wave modeling. This solver offers high\sphinxhyphen{}performance simulations on various hardware platforms and seamlessly integrates with machine learning and artificial intelligence environments. The solver makes use of the flexibility of the Python language for customization and interoperability, while the Taichi implementation provides high\sphinxhyphen{}performance capabilities and facilitates integration into artificial intelligence environments such as PyTorch.

\sphinxAtStartPar
Check out the {\hyperref[\detokenize{usage::doc}]{\sphinxcrossref{\DUrole{doc}{Usage}}}} section for further information,, including how to {\hyperref[\detokenize{usage:installation}]{\sphinxcrossref{\DUrole{std}{\DUrole{std-ref}{install}}}}} the project.

\begin{sphinxadmonition}{note}{Note:}
\sphinxAtStartPar
This project is under active development.
\end{sphinxadmonition}


\chapter{Documentation for the Code}
\label{\detokenize{index:documentation-for-the-code}}
\sphinxstepscope


\section{Introduction}
\label{\detokenize{introduction:introduction}}\label{\detokenize{introduction::doc}}

\subsection{CelerisAi: A Nearshore Wave Modeling Framework for Integrated AI Applications}
\label{\detokenize{introduction:celerisai-a-nearshore-wave-modeling-framework-for-integrated-ai-applications}}
\sphinxAtStartPar
CelerisAi is a Python and Taichi\sphinxhyphen{}based framework for nearshore wave modeling that seamlessly integrates with AI environments. It provides GPU and multi\sphinxhyphen{}CPU acceleration, flexible setup, and interactive visualization—enhancing coastal engineering applications such as depth inversion and complex nearshore simulations.


\subsection{Methods}
\label{\detokenize{introduction:methods}}
\sphinxAtStartPar
CelerisAi is built upon Celeris Base \sphinxcite{introduction:tavakkollynett2020}, which solves the extended Boussinesq equations derived by \sphinxcite{introduction:madsesorensen1992} to account for nonlinear and dispersive effects in nearshore wave transformation, and follows the architecture presented in the CelerisWebGPU version (\sphinxhref{https://plynett.github.io/}{CelerisWebGPU}).


\subsection{Characteristics}
\label{\detokenize{introduction:characteristics}}
\sphinxAtStartPar
The main characteristics of CelerisAi include:
\begin{itemize}
\item {} 
\sphinxAtStartPar
\sphinxstylestrong{Optimized Kernels:} Leverages custom Taichi\sphinxhyphen{}kernels for efficient computation.

\item {} 
\sphinxAtStartPar
\sphinxstylestrong{OS Agnosticism:} Compatible across various operating systems.

\item {} 
\sphinxAtStartPar
\sphinxstylestrong{High Performance:} Utilizes multiple CPUs, GPUs, or cloud services for accelerated simulations.

\item {} 
\sphinxAtStartPar
\sphinxstylestrong{Modularity:} Structured around four main classes, facilitating easy expansion with additional physics or models.

\item {} 
\sphinxAtStartPar
\sphinxstylestrong{AI Integration:} Incorporates PyTorch to send data directly from numerical models to neural networks without offloading from the GPU.

\item {} 
\sphinxAtStartPar
\sphinxstylestrong{Visualization Options:} Supports both headless and interactive visualization.

\end{itemize}


\subsection{Class Architecture and Workflow}
\label{\detokenize{introduction:class-architecture-and-workflow}}
\sphinxAtStartPar
The simulation setup can be performed directly within a Python script or through JSON files by specifying the domain geometry, initial conditions, and boundary conditions.

\begin{figure}[htbp]
\centering
\capstart

\noindent\sphinxincludegraphics[width=0.900\linewidth]{{Celeris}.png}
\caption{CelerisAi classes and setup.}\label{\detokenize{introduction:id3}}\end{figure}

\sphinxAtStartPar
CelerisAi can solve the shallow water equations or the extended Boussinesq equations. In both cases, the scalars and vectors are the same, but the methods used to solve the numerical scheme are different.


\subsubsection{Workflow to solve the Shallow water equations}
\label{\detokenize{introduction:workflow-to-solve-the-shallow-water-equations}}
\begin{figure}[htbp]
\centering
\capstart

\noindent\sphinxincludegraphics[width=0.800\linewidth]{{Celeris_SWE}.png}
\caption{SWE workflow}\label{\detokenize{introduction:id4}}\end{figure}


\subsubsection{Workflow to solve the extended Boussinesq equations}
\label{\detokenize{introduction:workflow-to-solve-the-extended-boussinesq-equations}}
\begin{figure}[htbp]
\centering
\capstart

\noindent\sphinxincludegraphics[width=0.800\linewidth]{{Celeris_Bouss}.png}
\caption{Boussinesq workflow}\label{\detokenize{introduction:id5}}\end{figure}


\subsection{Performance}
\label{\detokenize{introduction:performance}}
\sphinxAtStartPar
Ongoing efforts focus on replicating benchmark cases to verify accuracy—such as the solitary wave run\sphinxhyphen{}up on a sloping beach.

\sphinxAtStartPar
At run\sphinxhyphen{}time, simulation states can be transferred to PyTorch tensors, enabling data\sphinxhyphen{}driven updates or neural network\textendash{}based parameter updating in real\sphinxhyphen{}time. This approach reduces the necessity of offline data generation, facilitating a new paradigm of concurrent modeling learning, and allows for advanced applications like nearshore bathymetry inversion.​


\subsection{References}
\label{\detokenize{introduction:references}}
\sphinxstepscope


\section{Usage}
\label{\detokenize{usage:usage}}\label{\detokenize{usage::doc}}

\subsection{Installation}
\label{\detokenize{usage:installation}}\label{\detokenize{usage:id1}}
\sphinxAtStartPar
The prerequisites for CelerisAi are standard Python libraries, with Taichi being the most important. You can install these libraries manually:

\begin{sphinxVerbatim}[commandchars=\\\{\}]
\PYG{g+gp}{\PYGZdl{} }pip\PYG{+w}{ }install\PYG{+w}{ }imageio\PYGZgt{}\PYG{o}{=}\PYG{l+m}{2}.36.0\PYG{+w}{ }matplotlib\PYGZgt{}\PYG{o}{=}\PYG{l+m}{3}.7.2\PYG{+w}{ }numpy\PYGZgt{}\PYG{o}{=}\PYG{l+m}{1}.24.3\PYG{+w}{ }scipy\PYGZgt{}\PYG{o}{=}\PYG{l+m}{1}.9.0\PYG{+w}{ }taichi\PYGZgt{}\PYG{o}{=}\PYG{l+m}{1}.7.0
\end{sphinxVerbatim}

\sphinxAtStartPar
Alternatively, you can install all required dependencies automatically using the provided requirements file:

\begin{sphinxVerbatim}[commandchars=\\\{\}]
\PYG{g+gp}{\PYGZdl{} }pip\PYG{+w}{ }install\PYG{+w}{ }\PYGZhy{}r\PYG{+w}{ }requirements.txt
\end{sphinxVerbatim}


\subsection{Downloading the Source Code}
\label{\detokenize{usage:downloading-the-source-code}}
\sphinxAtStartPar
Clone the repository from GitHub to download the source code:

\begin{sphinxVerbatim}[commandchars=\\\{\}]
\PYG{g+gp}{\PYGZdl{} }git\PYG{+w}{ }clone\PYG{+w}{ }https://github.com/wrenteria/CelerisAi.git
\end{sphinxVerbatim}


\subsection{Running the Examples}
\label{\detokenize{usage:running-the-examples}}
\begin{sphinxadmonition}{note}{Note:}
\sphinxAtStartPar
Make sure to execute these examples from within the CelerisAi directory.
\end{sphinxadmonition}

\sphinxAtStartPar
After installing the dependencies and downloading the source, you can verify the installation by running the provided examples.

\sphinxAtStartPar
For a 1D example, execute:

\begin{sphinxVerbatim}[commandchars=\\\{\}]
\PYG{g+gp}{\PYGZdl{} }python\PYG{+w}{ }setrun\PYGZus{}1D.py
\end{sphinxVerbatim}

\sphinxAtStartPar
For a 2D example based on the configuration files created by CelerisWebGPU, execute:

\begin{sphinxVerbatim}[commandchars=\\\{\}]
\PYG{g+gp}{\PYGZdl{} }python\PYG{+w}{ }setrun\PYGZus{}web.py
\end{sphinxVerbatim}

\sphinxAtStartPar
For more details on configuring CelerisWebGPU, please refer to its application at
\sphinxhref{https://plynett.github.io/}{CelerisWebGPU}.

\sphinxstepscope


\section{Celeris Modules}
\label{\detokenize{modules:celeris-modules}}\label{\detokenize{modules::doc}}
\sphinxstepscope


\subsection{Modules}
\label{\detokenize{celeris:modules}}\label{\detokenize{celeris::doc}}

\subsubsection{Domain module}
\label{\detokenize{celeris:module-celeris.domain}}\label{\detokenize{celeris:domain-module}}\index{module@\spxentry{module}!celeris.domain@\spxentry{celeris.domain}}\index{celeris.domain@\spxentry{celeris.domain}!module@\spxentry{module}}\index{Topodata (class in celeris.domain)@\spxentry{Topodata}\spxextra{class in celeris.domain}}

\begin{fulllineitems}
\phantomsection\label{\detokenize{celeris:celeris.domain.Topodata}}
\pysigstartsignatures
\pysiglinewithargsret
{\sphinxbfcode{\sphinxupquote{class\DUrole{w}{ }}}\sphinxcode{\sphinxupquote{celeris.domain.}}\sphinxbfcode{\sphinxupquote{Topodata}}}
{\sphinxparam{\DUrole{n}{filename}\DUrole{o}{=}\DUrole{default_value}{None}}\sphinxparamcomma \sphinxparam{\DUrole{n}{datatype}\DUrole{o}{=}\DUrole{default_value}{None}}\sphinxparamcomma \sphinxparam{\DUrole{n}{path}\DUrole{o}{=}\DUrole{default_value}{None}}}
{}
\pysigstopsignatures
\sphinxAtStartPar
Manages all topography/bathymetry data formats for CelerisAi.

\sphinxAtStartPar
This class handles different input formats for bathymetric or topographic data,
such as 2D (XYZ), 1D (XZ), and Celeris\sphinxhyphen{}native formats. It reads the corresponding
files (or folders) and loads them into a NumPy array for further processing or
analysis.
\index{filename (celeris.domain.Topodata attribute)@\spxentry{filename}\spxextra{celeris.domain.Topodata attribute}}

\begin{fulllineitems}
\phantomsection\label{\detokenize{celeris:celeris.domain.Topodata.filename}}
\pysigstartsignatures
\pysigline
{\sphinxbfcode{\sphinxupquote{filename}}}
\pysigstopsignatures
\sphinxAtStartPar
Name of the file containing topographic/bathymetric
data. If \sphinxtitleref{path} is provided, the file is assumed to be located there.
\begin{quote}\begin{description}
\sphinxlineitem{Type}
\sphinxAtStartPar
str, optional

\end{description}\end{quote}

\end{fulllineitems}

\index{datatype (celeris.domain.Topodata attribute)@\spxentry{datatype}\spxextra{celeris.domain.Topodata attribute}}

\begin{fulllineitems}
\phantomsection\label{\detokenize{celeris:celeris.domain.Topodata.datatype}}
\pysigstartsignatures
\pysigline
{\sphinxbfcode{\sphinxupquote{datatype}}}
\pysigstopsignatures
\sphinxAtStartPar
Format of the data. Accepted values are:
\sphinxhyphen{} “xyz”: 2D data in three columns (x, y, z)
\sphinxhyphen{} “xz”: 1D data in two columns (x, z)
\sphinxhyphen{} “celeris”: 2D data located in a folder with a file named “bathy.txt”
\begin{quote}\begin{description}
\sphinxlineitem{Type}
\sphinxAtStartPar
str, optional

\end{description}\end{quote}

\end{fulllineitems}

\index{path (celeris.domain.Topodata attribute)@\spxentry{path}\spxextra{celeris.domain.Topodata attribute}}

\begin{fulllineitems}
\phantomsection\label{\detokenize{celeris:celeris.domain.Topodata.path}}
\pysigstartsignatures
\pysigline
{\sphinxbfcode{\sphinxupquote{path}}}
\pysigstopsignatures
\sphinxAtStartPar
Directory path where the file is located. If not provided,
\sphinxtitleref{filename} should be a complete path or in the current working directory.
\begin{quote}\begin{description}
\sphinxlineitem{Type}
\sphinxAtStartPar
str, optional

\end{description}\end{quote}

\end{fulllineitems}

\subsubsection*{Example}

\begin{sphinxVerbatim}[commandchars=\\\{\}]
\PYG{g+gp}{\PYGZgt{}\PYGZgt{}\PYGZgt{} }\PYG{n}{Bfrom} \PYG{n}{celeris}\PYG{o}{.}\PYG{n}{domain} \PYG{k+kn}{import}\PYG{+w}{ }\PYG{n+nn}{Topodata}
\PYG{g+gp}{\PYGZgt{}\PYGZgt{}\PYGZgt{} }\PYG{n}{baty} \PYG{o}{=} \PYG{n}{Topodata}\PYG{p}{(}\PYG{n}{datatype}\PYG{o}{=}\PYG{l+s+s1}{\PYGZsq{}}\PYG{l+s+s1}{celeris}\PYG{l+s+s1}{\PYGZsq{}}\PYG{p}{,}\PYG{n}{path}\PYG{o}{=}\PYG{l+s+s1}{\PYGZsq{}}\PYG{l+s+s1}{./examples/DuckFRF\PYGZus{}NC}\PYG{l+s+s1}{\PYGZsq{}}\PYG{p}{)}
\end{sphinxVerbatim}
\index{z() (celeris.domain.Topodata method)@\spxentry{z()}\spxextra{celeris.domain.Topodata method}}

\begin{fulllineitems}
\phantomsection\label{\detokenize{celeris:celeris.domain.Topodata.z}}
\pysigstartsignatures
\pysiglinewithargsret
{\sphinxbfcode{\sphinxupquote{z}}}
{}
{}
\pysigstopsignatures
\sphinxAtStartPar
Loads the bathymetry/topography data based on the specified datatype.

\sphinxAtStartPar
Depending on the \sphinxtitleref{datatype}, this method reads the corresponding file(s) and
returns the data as a NumPy array.
\begin{itemize}
\item {} 
\sphinxAtStartPar
For “xyz”: Expects a file with three columns (x, y, z).

\item {} 
\sphinxAtStartPar
For “xz”: Expects a file with two columns (x, z).

\item {} 
\sphinxAtStartPar
For “xyz” and “xz”, bathymetry must be positive.

\item {} 
\sphinxAtStartPar
For “celeris”: Expects a folder containing a file named “bathy.txt”.
The returned array values are multiplied by \sphinxhyphen{}1.

\end{itemize}
\begin{quote}\begin{description}
\sphinxlineitem{Returns}
\sphinxAtStartPar
\begin{itemize}
\item {} 
\sphinxAtStartPar
A NumPy array if \sphinxtitleref{datatype} is recognized and the file is successfully
read.

\item {} 
\sphinxAtStartPar
The string “No supported format” if \sphinxtitleref{datatype} is not recognized.

\end{itemize}


\sphinxlineitem{Return type}
\sphinxAtStartPar
numpy.ndarray or str

\sphinxlineitem{Raises}
\sphinxAtStartPar
\sphinxstyleliteralstrong{\sphinxupquote{OSError}} \textendash{} If the file (or directory for “celeris”) cannot be found or read.

\end{description}\end{quote}

\end{fulllineitems}


\end{fulllineitems}

\index{BoundaryConditions (class in celeris.domain)@\spxentry{BoundaryConditions}\spxextra{class in celeris.domain}}

\begin{fulllineitems}
\phantomsection\label{\detokenize{celeris:celeris.domain.BoundaryConditions}}
\pysigstartsignatures
\pysiglinewithargsret
{\sphinxbfcode{\sphinxupquote{class\DUrole{w}{ }}}\sphinxcode{\sphinxupquote{celeris.domain.}}\sphinxbfcode{\sphinxupquote{BoundaryConditions}}}
{\sphinxparam{\DUrole{n}{celeris}\DUrole{o}{=}\DUrole{default_value}{True}}\sphinxparamcomma \sphinxparam{\DUrole{n}{precision}\DUrole{o}{=}\DUrole{default_value}{ti.f32}}\sphinxparamcomma \sphinxparam{\DUrole{n}{North}\DUrole{o}{=}\DUrole{default_value}{10}}\sphinxparamcomma \sphinxparam{\DUrole{n}{South}\DUrole{o}{=}\DUrole{default_value}{10}}\sphinxparamcomma \sphinxparam{\DUrole{n}{East}\DUrole{o}{=}\DUrole{default_value}{10}}\sphinxparamcomma \sphinxparam{\DUrole{n}{West}\DUrole{o}{=}\DUrole{default_value}{10}}\sphinxparamcomma \sphinxparam{\DUrole{n}{WaveType}\DUrole{o}{=}\DUrole{default_value}{\sphinxhyphen{}1}}\sphinxparamcomma \sphinxparam{\DUrole{n}{Amplitude}\DUrole{o}{=}\DUrole{default_value}{0.5}}\sphinxparamcomma \sphinxparam{\DUrole{n}{path}\DUrole{o}{=}\DUrole{default_value}{\textquotesingle{}./scratch\textquotesingle{}}}\sphinxparamcomma \sphinxparam{\DUrole{n}{filename}\DUrole{o}{=}\DUrole{default_value}{\textquotesingle{}waves.txt\textquotesingle{}}}\sphinxparamcomma \sphinxparam{\DUrole{n}{BoundaryWidth}\DUrole{o}{=}\DUrole{default_value}{20}}\sphinxparamcomma \sphinxparam{\DUrole{n}{init\_eta}\DUrole{o}{=}\DUrole{default_value}{5}}\sphinxparamcomma \sphinxparam{\DUrole{n}{sine\_wave}\DUrole{o}{=}\DUrole{default_value}{None}}}
{}
\pysigstopsignatures
\sphinxAtStartPar
Manages boundary conditions for a rectangular domain in the CelerisAi model.

\sphinxAtStartPar
The domain has four faces: north, south, east, and west. Each face can be configured
with different boundary types (e.g., sponge layer, solid wall, incoming wave).
Boundary conditions can be set in one of two ways:
\begin{enumerate}
\sphinxsetlistlabels{\arabic}{enumi}{enumii}{}{)}%
\item {} 
\sphinxAtStartPar
\sphinxstylestrong{Celeris format} (\sphinxtitleref{celeris=True}): Reads from a \sphinxtitleref{config.json} file.

\item {} 
\sphinxAtStartPar
\sphinxstylestrong{Manual} (\sphinxtitleref{celeris=False}): Uses manually supplied values.

\end{enumerate}

\sphinxAtStartPar
If incoming waves (boundary type = 2) are defined, the wave type can be specified via
\sphinxtitleref{WaveType}. If \sphinxtitleref{WaveType} is \sphinxhyphen{}1, the wave parameters are read from a file (e.g.,
\sphinxtitleref{“waves.txt”}). Otherwise, a sine wave is assumed and set by \sphinxtitleref{sine\_wave}.

\sphinxAtStartPar
This class also includes utility methods for handling wave data input and conversion
from NumPy arrays to Taichi tensors.
\index{precision (celeris.domain.BoundaryConditions attribute)@\spxentry{precision}\spxextra{celeris.domain.BoundaryConditions attribute}}

\begin{fulllineitems}
\phantomsection\label{\detokenize{celeris:celeris.domain.BoundaryConditions.precision}}
\pysigstartsignatures
\pysigline
{\sphinxbfcode{\sphinxupquote{precision}}}
\pysigstopsignatures
\sphinxAtStartPar
The precision used by Taichi (e.g. \sphinxtitleref{ti.f32}, \sphinxtitleref{ti.f64}).
\begin{quote}\begin{description}
\sphinxlineitem{Type}
\sphinxAtStartPar
taichi.types.primitive\_types

\end{description}\end{quote}

\end{fulllineitems}

\index{North (celeris.domain.BoundaryConditions attribute)@\spxentry{North}\spxextra{celeris.domain.BoundaryConditions attribute}}

\begin{fulllineitems}
\phantomsection\label{\detokenize{celeris:celeris.domain.BoundaryConditions.North}}
\pysigstartsignatures
\pysigline
{\sphinxbfcode{\sphinxupquote{North}}}
\pysigstopsignatures
\sphinxAtStartPar
Boundary type for the north face. Valid types are:
\begin{itemize}
\item {} 
\sphinxAtStartPar
0: Solid wall

\item {} 
\sphinxAtStartPar
1: Sponge layer

\item {} 
\sphinxAtStartPar
2: Incoming wave

\end{itemize}
\begin{quote}\begin{description}
\sphinxlineitem{Type}
\sphinxAtStartPar
int

\end{description}\end{quote}

\end{fulllineitems}

\index{South (celeris.domain.BoundaryConditions attribute)@\spxentry{South}\spxextra{celeris.domain.BoundaryConditions attribute}}

\begin{fulllineitems}
\phantomsection\label{\detokenize{celeris:celeris.domain.BoundaryConditions.South}}
\pysigstartsignatures
\pysigline
{\sphinxbfcode{\sphinxupquote{South}}}
\pysigstopsignatures
\sphinxAtStartPar
Boundary type for the south face.
\begin{quote}\begin{description}
\sphinxlineitem{Type}
\sphinxAtStartPar
int

\end{description}\end{quote}

\end{fulllineitems}

\index{East (celeris.domain.BoundaryConditions attribute)@\spxentry{East}\spxextra{celeris.domain.BoundaryConditions attribute}}

\begin{fulllineitems}
\phantomsection\label{\detokenize{celeris:celeris.domain.BoundaryConditions.East}}
\pysigstartsignatures
\pysigline
{\sphinxbfcode{\sphinxupquote{East}}}
\pysigstopsignatures
\sphinxAtStartPar
Boundary type for the east face.
\begin{quote}\begin{description}
\sphinxlineitem{Type}
\sphinxAtStartPar
int

\end{description}\end{quote}

\end{fulllineitems}

\index{West (celeris.domain.BoundaryConditions attribute)@\spxentry{West}\spxextra{celeris.domain.BoundaryConditions attribute}}

\begin{fulllineitems}
\phantomsection\label{\detokenize{celeris:celeris.domain.BoundaryConditions.West}}
\pysigstartsignatures
\pysigline
{\sphinxbfcode{\sphinxupquote{West}}}
\pysigstopsignatures
\sphinxAtStartPar
Boundary type for the west face.
\begin{quote}\begin{description}
\sphinxlineitem{Type}
\sphinxAtStartPar
int

\end{description}\end{quote}

\end{fulllineitems}

\index{WaveType (celeris.domain.BoundaryConditions attribute)@\spxentry{WaveType}\spxextra{celeris.domain.BoundaryConditions attribute}}

\begin{fulllineitems}
\phantomsection\label{\detokenize{celeris:celeris.domain.BoundaryConditions.WaveType}}
\pysigstartsignatures
\pysigline
{\sphinxbfcode{\sphinxupquote{WaveType}}}
\pysigstopsignatures
\sphinxAtStartPar
Wave type indicator:
\begin{itemize}
\item {} 
\sphinxAtStartPar
\sphinxhyphen{}1: Wave parameters read from a file (\sphinxtitleref{waves.txt})

\item {} 
\sphinxAtStartPar
Any other integer: Use \sphinxtitleref{sine\_wave} array for wave parameters

\end{itemize}
\begin{quote}\begin{description}
\sphinxlineitem{Type}
\sphinxAtStartPar
int

\end{description}\end{quote}

\end{fulllineitems}

\index{Amplitude (celeris.domain.BoundaryConditions attribute)@\spxentry{Amplitude}\spxextra{celeris.domain.BoundaryConditions attribute}}

\begin{fulllineitems}
\phantomsection\label{\detokenize{celeris:celeris.domain.BoundaryConditions.Amplitude}}
\pysigstartsignatures
\pysigline
{\sphinxbfcode{\sphinxupquote{Amplitude}}}
\pysigstopsignatures
\sphinxAtStartPar
Wave amplitude, used if not reading from a file.
\begin{quote}\begin{description}
\sphinxlineitem{Type}
\sphinxAtStartPar
float

\end{description}\end{quote}

\end{fulllineitems}

\index{path (celeris.domain.BoundaryConditions attribute)@\spxentry{path}\spxextra{celeris.domain.BoundaryConditions attribute}}

\begin{fulllineitems}
\phantomsection\label{\detokenize{celeris:celeris.domain.BoundaryConditions.path}}
\pysigstartsignatures
\pysigline
{\sphinxbfcode{\sphinxupquote{path}}}
\pysigstopsignatures
\sphinxAtStartPar
Path to the directory containing \sphinxtitleref{waves.txt} and/or \sphinxtitleref{config.json}.
\begin{quote}\begin{description}
\sphinxlineitem{Type}
\sphinxAtStartPar
str

\end{description}\end{quote}

\end{fulllineitems}

\index{filename (celeris.domain.BoundaryConditions attribute)@\spxentry{filename}\spxextra{celeris.domain.BoundaryConditions attribute}}

\begin{fulllineitems}
\phantomsection\label{\detokenize{celeris:celeris.domain.BoundaryConditions.filename}}
\pysigstartsignatures
\pysigline
{\sphinxbfcode{\sphinxupquote{filename}}}
\pysigstopsignatures
\sphinxAtStartPar
Name of the file that stores wave parameters (e.g. \sphinxtitleref{waves.txt}).
\begin{quote}\begin{description}
\sphinxlineitem{Type}
\sphinxAtStartPar
str

\end{description}\end{quote}

\end{fulllineitems}

\index{BoundaryWidth (celeris.domain.BoundaryConditions attribute)@\spxentry{BoundaryWidth}\spxextra{celeris.domain.BoundaryConditions attribute}}

\begin{fulllineitems}
\phantomsection\label{\detokenize{celeris:celeris.domain.BoundaryConditions.BoundaryWidth}}
\pysigstartsignatures
\pysigline
{\sphinxbfcode{\sphinxupquote{BoundaryWidth}}}
\pysigstopsignatures
\sphinxAtStartPar
Width of the sponge or boundary zone.
\begin{quote}\begin{description}
\sphinxlineitem{Type}
\sphinxAtStartPar
int

\end{description}\end{quote}

\end{fulllineitems}

\index{sine\_wave (celeris.domain.BoundaryConditions attribute)@\spxentry{sine\_wave}\spxextra{celeris.domain.BoundaryConditions attribute}}

\begin{fulllineitems}
\phantomsection\label{\detokenize{celeris:celeris.domain.BoundaryConditions.sine_wave}}
\pysigstartsignatures
\pysigline
{\sphinxbfcode{\sphinxupquote{sine\_wave}}}
\pysigstopsignatures
\sphinxAtStartPar
Parameters defining a sine wave if \sphinxtitleref{WaveType} is not \sphinxhyphen{}1.
\begin{quote}\begin{description}
\sphinxlineitem{Type}
\sphinxAtStartPar
list of float

\end{description}\end{quote}

\end{fulllineitems}

\index{celeris (celeris.domain.BoundaryConditions attribute)@\spxentry{celeris}\spxextra{celeris.domain.BoundaryConditions attribute}}

\begin{fulllineitems}
\phantomsection\label{\detokenize{celeris:celeris.domain.BoundaryConditions.celeris}}
\pysigstartsignatures
\pysigline
{\sphinxbfcode{\sphinxupquote{celeris}}}
\pysigstopsignatures
\sphinxAtStartPar
If True, boundary conditions are read from \sphinxtitleref{config.json}.
\begin{quote}\begin{description}
\sphinxlineitem{Type}
\sphinxAtStartPar
bool

\end{description}\end{quote}

\end{fulllineitems}

\index{configfile (celeris.domain.BoundaryConditions attribute)@\spxentry{configfile}\spxextra{celeris.domain.BoundaryConditions attribute}}

\begin{fulllineitems}
\phantomsection\label{\detokenize{celeris:celeris.domain.BoundaryConditions.configfile}}
\pysigstartsignatures
\pysigline
{\sphinxbfcode{\sphinxupquote{configfile}}}
\pysigstopsignatures
\sphinxAtStartPar
Loaded JSON configuration when \sphinxtitleref{celeris=True}.
\begin{quote}\begin{description}
\sphinxlineitem{Type}
\sphinxAtStartPar
dict, optional

\end{description}\end{quote}

\end{fulllineitems}

\index{data (celeris.domain.BoundaryConditions attribute)@\spxentry{data}\spxextra{celeris.domain.BoundaryConditions attribute}}

\begin{fulllineitems}
\phantomsection\label{\detokenize{celeris:celeris.domain.BoundaryConditions.data}}
\pysigstartsignatures
\pysigline
{\sphinxbfcode{\sphinxupquote{data}}}
\pysigstopsignatures
\sphinxAtStartPar
Placeholder array for wave data (size depends on \sphinxtitleref{N\_data}).
\begin{quote}\begin{description}
\sphinxlineitem{Type}
\sphinxAtStartPar
numpy.ndarray

\end{description}\end{quote}

\end{fulllineitems}

\index{N\_data (celeris.domain.BoundaryConditions attribute)@\spxentry{N\_data}\spxextra{celeris.domain.BoundaryConditions attribute}}

\begin{fulllineitems}
\phantomsection\label{\detokenize{celeris:celeris.domain.BoundaryConditions.N_data}}
\pysigstartsignatures
\pysigline
{\sphinxbfcode{\sphinxupquote{N\_data}}}
\pysigstopsignatures
\sphinxAtStartPar
Number of wave entries to read from file.
\begin{quote}\begin{description}
\sphinxlineitem{Type}
\sphinxAtStartPar
int

\end{description}\end{quote}

\end{fulllineitems}

\index{W\_data (celeris.domain.BoundaryConditions attribute)@\spxentry{W\_data}\spxextra{celeris.domain.BoundaryConditions attribute}}

\begin{fulllineitems}
\phantomsection\label{\detokenize{celeris:celeris.domain.BoundaryConditions.W_data}}
\pysigstartsignatures
\pysigline
{\sphinxbfcode{\sphinxupquote{W\_data}}}
\pysigstopsignatures
\sphinxAtStartPar
Unused placeholder (could store wave data in some contexts).
\begin{quote}\begin{description}
\sphinxlineitem{Type}
\sphinxAtStartPar
None

\end{description}\end{quote}

\end{fulllineitems}

\subsubsection*{Example}

\begin{sphinxVerbatim}[commandchars=\\\{\}]
\PYG{g+gp}{\PYGZgt{}\PYGZgt{}\PYGZgt{} }\PYG{k+kn}{from}\PYG{+w}{ }\PYG{n+nn}{celeris}\PYG{n+nn}{.}\PYG{n+nn}{domain}\PYG{+w}{ }\PYG{k+kn}{import} \PYG{n}{BoundaryConditions}
\PYG{g+gp}{\PYGZgt{}\PYGZgt{}\PYGZgt{} }\PYG{n}{bc} \PYG{o}{=} \PYG{n}{BoundaryConditions}\PYG{p}{(}\PYG{n}{celeris}\PYG{o}{=}\PYG{k+kc}{True}\PYG{p}{,}\PYG{n}{path}\PYG{o}{=}\PYG{l+s+s1}{\PYGZsq{}}\PYG{l+s+s1}{./examples/DuckFRF\PYGZus{}NC}\PYG{l+s+s1}{\PYGZsq{}}\PYG{p}{,}\PYG{n}{precision}\PYG{o}{=}\PYG{n}{precision}\PYG{p}{)}
\end{sphinxVerbatim}
\index{SineWave() (celeris.domain.BoundaryConditions method)@\spxentry{SineWave()}\spxextra{celeris.domain.BoundaryConditions method}}

\begin{fulllineitems}
\phantomsection\label{\detokenize{celeris:celeris.domain.BoundaryConditions.SineWave}}
\pysigstartsignatures
\pysiglinewithargsret
{\sphinxbfcode{\sphinxupquote{SineWave}}}
{}
{}
\pysigstopsignatures
\sphinxAtStartPar
Provides the sine wave parameters as a NumPy array of the specified precision.
\begin{quote}\begin{description}
\sphinxlineitem{Returns}
\sphinxAtStartPar
An array of length 4 containing the sine wave parameters.

\sphinxlineitem{Return type}
\sphinxAtStartPar
numpy.ndarray

\end{description}\end{quote}

\end{fulllineitems}

\index{Sponge() (celeris.domain.BoundaryConditions method)@\spxentry{Sponge()}\spxextra{celeris.domain.BoundaryConditions method}}

\begin{fulllineitems}
\phantomsection\label{\detokenize{celeris:celeris.domain.BoundaryConditions.Sponge}}
\pysigstartsignatures
\pysiglinewithargsret
{\sphinxbfcode{\sphinxupquote{Sponge}}}
{\sphinxparam{\DUrole{n}{width}\DUrole{o}{=}\DUrole{default_value}{None}}}
{}
\pysigstopsignatures
\sphinxAtStartPar
Returns the sponge (boundary) width to be used in the model.
\begin{quote}\begin{description}
\sphinxlineitem{Parameters}
\sphinxAtStartPar
\sphinxstyleliteralstrong{\sphinxupquote{width}} (\sphinxstyleliteralemphasis{\sphinxupquote{int}}\sphinxstyleliteralemphasis{\sphinxupquote{, }}\sphinxstyleliteralemphasis{\sphinxupquote{optional}}) \textendash{} Overrides the class\sphinxhyphen{}level sponge width. Defaults to None.

\sphinxlineitem{Returns}
\sphinxAtStartPar
The sponge width (either the class attribute or the passed\sphinxhyphen{}in argument).

\sphinxlineitem{Return type}
\sphinxAtStartPar
int

\end{description}\end{quote}

\end{fulllineitems}

\index{get\_data() (celeris.domain.BoundaryConditions method)@\spxentry{get\_data()}\spxextra{celeris.domain.BoundaryConditions method}}

\begin{fulllineitems}
\phantomsection\label{\detokenize{celeris:celeris.domain.BoundaryConditions.get_data}}
\pysigstartsignatures
\pysiglinewithargsret
{\sphinxbfcode{\sphinxupquote{get\_data}}}
{}
{}
\pysigstopsignatures
\sphinxAtStartPar
Ensures wave data is loaded if reading from a file, then returns a Taichi field.

\sphinxAtStartPar
If \sphinxtitleref{WaveType} is \sphinxhyphen{}1, calls \sphinxtitleref{load\_data()} to populate \sphinxtitleref{self.data} from the file.
Converts the resulting NumPy array into a Taichi field and returns it.
\begin{quote}\begin{description}
\sphinxlineitem{Returns}
\sphinxAtStartPar
A Taichi field of shape \sphinxtitleref{(N\_data, 4)} containing wave parameters.

\sphinxlineitem{Return type}
\sphinxAtStartPar
ti.field

\end{description}\end{quote}

\end{fulllineitems}

\index{load\_data() (celeris.domain.BoundaryConditions method)@\spxentry{load\_data()}\spxextra{celeris.domain.BoundaryConditions method}}

\begin{fulllineitems}
\phantomsection\label{\detokenize{celeris:celeris.domain.BoundaryConditions.load_data}}
\pysigstartsignatures
\pysiglinewithargsret
{\sphinxbfcode{\sphinxupquote{load\_data}}}
{}
{}
\pysigstopsignatures
\sphinxAtStartPar
Loads wave parameters from \sphinxtitleref{waves.txt} if \sphinxtitleref{WaveType} is \sphinxhyphen{}1.

\sphinxAtStartPar
If a valid file is found, it reads the wave parameters:
\begin{itemize}
\item {} 
\sphinxAtStartPar
The first three lines are header\sphinxhyphen{}like info (only the ‘NumberOfWaves’ line is used).

\item {} 
\sphinxAtStartPar
After skipping three lines, wave parameters are loaded. The shape of \sphinxtitleref{self.data}
is \sphinxtitleref{(N\_data, 4)}, where \sphinxtitleref{N\_data} is determined by reading the ‘NumberOfWaves’
line in the file.

\end{itemize}

\end{fulllineitems}

\index{tseries() (celeris.domain.BoundaryConditions method)@\spxentry{tseries()}\spxextra{celeris.domain.BoundaryConditions method}}

\begin{fulllineitems}
\phantomsection\label{\detokenize{celeris:celeris.domain.BoundaryConditions.tseries}}
\pysigstartsignatures
\pysiglinewithargsret
{\sphinxbfcode{\sphinxupquote{tseries}}}
{}
{}
\pysigstopsignatures
\sphinxAtStartPar
Checks if a wave parameter file is specified.
\begin{quote}\begin{description}
\sphinxlineitem{Returns}
\sphinxAtStartPar
True if \sphinxtitleref{self.filename} is not None, indicating that time series
wave data can be read from a file. Otherwise, False.

\sphinxlineitem{Return type}
\sphinxAtStartPar
bool

\end{description}\end{quote}

\end{fulllineitems}


\end{fulllineitems}

\index{Domain (class in celeris.domain)@\spxentry{Domain}\spxextra{class in celeris.domain}}

\begin{fulllineitems}
\phantomsection\label{\detokenize{celeris:celeris.domain.Domain}}
\pysigstartsignatures
\pysiglinewithargsret
{\sphinxbfcode{\sphinxupquote{class\DUrole{w}{ }}}\sphinxcode{\sphinxupquote{celeris.domain.}}\sphinxbfcode{\sphinxupquote{Domain}}}
{\sphinxparam{\DUrole{n}{precision}\DUrole{o}{=}\DUrole{default_value}{ti.f32}}\sphinxparamcomma \sphinxparam{\DUrole{n}{x1}\DUrole{o}{=}\DUrole{default_value}{0.0}}\sphinxparamcomma \sphinxparam{\DUrole{n}{x2}\DUrole{o}{=}\DUrole{default_value}{0.0}}\sphinxparamcomma \sphinxparam{\DUrole{n}{y1}\DUrole{o}{=}\DUrole{default_value}{0.0}}\sphinxparamcomma \sphinxparam{\DUrole{n}{y2}\DUrole{o}{=}\DUrole{default_value}{0.0}}\sphinxparamcomma \sphinxparam{\DUrole{n}{Nx}\DUrole{o}{=}\DUrole{default_value}{1}}\sphinxparamcomma \sphinxparam{\DUrole{n}{Ny}\DUrole{o}{=}\DUrole{default_value}{1}}\sphinxparamcomma \sphinxparam{\DUrole{n}{topodata}\DUrole{o}{=}\DUrole{default_value}{None}}\sphinxparamcomma \sphinxparam{\DUrole{n}{north\_sl}\DUrole{o}{=}\DUrole{default_value}{0.0}}\sphinxparamcomma \sphinxparam{\DUrole{n}{south\_sl}\DUrole{o}{=}\DUrole{default_value}{0.0}}\sphinxparamcomma \sphinxparam{\DUrole{n}{east\_sl}\DUrole{o}{=}\DUrole{default_value}{0.0}}\sphinxparamcomma \sphinxparam{\DUrole{n}{west\_sl}\DUrole{o}{=}\DUrole{default_value}{0.0}}\sphinxparamcomma \sphinxparam{\DUrole{n}{Courant}\DUrole{o}{=}\DUrole{default_value}{0.2}}\sphinxparamcomma \sphinxparam{\DUrole{n}{isManning}\DUrole{o}{=}\DUrole{default_value}{0}}\sphinxparamcomma \sphinxparam{\DUrole{n}{friction}\DUrole{o}{=}\DUrole{default_value}{0.001}}\sphinxparamcomma \sphinxparam{\DUrole{n}{base\_depth}\DUrole{o}{=}\DUrole{default_value}{None}}\sphinxparamcomma \sphinxparam{\DUrole{n}{BoundaryShift}\DUrole{o}{=}\DUrole{default_value}{4}}}
{}
\pysigstopsignatures
\sphinxAtStartPar
Defines the numerical domain for wave propagation solver in CelerisAi.

\sphinxAtStartPar
This class sets up the domain geometry (x1, x2, y1, y2) and resolution (Nx, Ny),
handles bathymetric/topographic data (via an instance of a \sphinxtitleref{Topodata} class),
configures boundary sea levels on each face (north, south, east, west),
and stores critical parameters such as Courant number (\sphinxtitleref{Courant}), friction,
and base depth. Two main branches are supported for configuration:
\begin{enumerate}
\sphinxsetlistlabels{\arabic}{enumi}{enumii}{}{.}%
\item {} 
\sphinxAtStartPar
\sphinxstylestrong{Celeris format}: Reads from a \sphinxtitleref{config.json} file (if \sphinxtitleref{topodata.datatype == “celeris”}).

\item {} 
\sphinxAtStartPar
\sphinxstylestrong{Manual}: Uses provided arguments directly.

\end{enumerate}

\sphinxAtStartPar
The class also defines utility methods to:
\begin{itemize}
\item {} 
\sphinxAtStartPar
Create a meshgrid for the domain (\sphinxtitleref{grid}).

\item {} 
\sphinxAtStartPar
Load and interpolate topographic/bathymetric data (\sphinxtitleref{topofield}, \sphinxtitleref{bottom}).

\item {} 
\sphinxAtStartPar
Compute maximum depth (\sphinxtitleref{maxdepth}) and highest topography (\sphinxtitleref{maxtopo}).

\item {} 
\sphinxAtStartPar
Compute the time step (\sphinxtitleref{dt}) based on Courant criteria.

\item {} 
\sphinxAtStartPar
Provide reflection indices for solid boundary conditions (\sphinxtitleref{reflect\_x}, \sphinxtitleref{reflect\_y}).

\item {} 
\sphinxAtStartPar
Create Taichi field templates for solver states (\sphinxtitleref{states}, \sphinxtitleref{states\_one}).

\end{itemize}
\index{precision (celeris.domain.Domain attribute)@\spxentry{precision}\spxextra{celeris.domain.Domain attribute}}

\begin{fulllineitems}
\phantomsection\label{\detokenize{celeris:celeris.domain.Domain.precision}}
\pysigstartsignatures
\pysigline
{\sphinxbfcode{\sphinxupquote{precision}}}
\pysigstopsignatures
\sphinxAtStartPar
Taichi precision (e.g., \sphinxtitleref{ti.f32}, \sphinxtitleref{ti.f64}).
\begin{quote}\begin{description}
\sphinxlineitem{Type}
\sphinxAtStartPar
ti.types.primitive\_types

\end{description}\end{quote}

\end{fulllineitems}

\index{x1 (celeris.domain.Domain attribute)@\spxentry{x1}\spxextra{celeris.domain.Domain attribute}}

\begin{fulllineitems}
\phantomsection\label{\detokenize{celeris:celeris.domain.Domain.x1}}
\pysigstartsignatures
\pysigline
{\sphinxbfcode{\sphinxupquote{x1}}}
\pysigstopsignatures
\sphinxAtStartPar
Minimum x\sphinxhyphen{}coordinate of the domain.
\begin{quote}\begin{description}
\sphinxlineitem{Type}
\sphinxAtStartPar
float

\end{description}\end{quote}

\end{fulllineitems}

\index{x2 (celeris.domain.Domain attribute)@\spxentry{x2}\spxextra{celeris.domain.Domain attribute}}

\begin{fulllineitems}
\phantomsection\label{\detokenize{celeris:celeris.domain.Domain.x2}}
\pysigstartsignatures
\pysigline
{\sphinxbfcode{\sphinxupquote{x2}}}
\pysigstopsignatures
\sphinxAtStartPar
Maximum x\sphinxhyphen{}coordinate of the domain.
\begin{quote}\begin{description}
\sphinxlineitem{Type}
\sphinxAtStartPar
float

\end{description}\end{quote}

\end{fulllineitems}

\index{y1 (celeris.domain.Domain attribute)@\spxentry{y1}\spxextra{celeris.domain.Domain attribute}}

\begin{fulllineitems}
\phantomsection\label{\detokenize{celeris:celeris.domain.Domain.y1}}
\pysigstartsignatures
\pysigline
{\sphinxbfcode{\sphinxupquote{y1}}}
\pysigstopsignatures
\sphinxAtStartPar
Minimum y\sphinxhyphen{}coordinate of the domain.
\begin{quote}\begin{description}
\sphinxlineitem{Type}
\sphinxAtStartPar
float

\end{description}\end{quote}

\end{fulllineitems}

\index{y2 (celeris.domain.Domain attribute)@\spxentry{y2}\spxextra{celeris.domain.Domain attribute}}

\begin{fulllineitems}
\phantomsection\label{\detokenize{celeris:celeris.domain.Domain.y2}}
\pysigstartsignatures
\pysigline
{\sphinxbfcode{\sphinxupquote{y2}}}
\pysigstopsignatures
\sphinxAtStartPar
Maximum y\sphinxhyphen{}coordinate of the domain.
\begin{quote}\begin{description}
\sphinxlineitem{Type}
\sphinxAtStartPar
float

\end{description}\end{quote}

\end{fulllineitems}

\index{Nx (celeris.domain.Domain attribute)@\spxentry{Nx}\spxextra{celeris.domain.Domain attribute}}

\begin{fulllineitems}
\phantomsection\label{\detokenize{celeris:celeris.domain.Domain.Nx}}
\pysigstartsignatures
\pysigline
{\sphinxbfcode{\sphinxupquote{Nx}}}
\pysigstopsignatures
\sphinxAtStartPar
Number of grid cells in the x\sphinxhyphen{}direction.
\begin{quote}\begin{description}
\sphinxlineitem{Type}
\sphinxAtStartPar
int

\end{description}\end{quote}

\end{fulllineitems}

\index{Ny (celeris.domain.Domain attribute)@\spxentry{Ny}\spxextra{celeris.domain.Domain attribute}}

\begin{fulllineitems}
\phantomsection\label{\detokenize{celeris:celeris.domain.Domain.Ny}}
\pysigstartsignatures
\pysigline
{\sphinxbfcode{\sphinxupquote{Ny}}}
\pysigstopsignatures
\sphinxAtStartPar
Number of grid cells in the y\sphinxhyphen{}direction.
\begin{quote}\begin{description}
\sphinxlineitem{Type}
\sphinxAtStartPar
int

\end{description}\end{quote}

\end{fulllineitems}

\index{topodata (celeris.domain.Domain attribute)@\spxentry{topodata}\spxextra{celeris.domain.Domain attribute}}

\begin{fulllineitems}
\phantomsection\label{\detokenize{celeris:celeris.domain.Domain.topodata}}
\pysigstartsignatures
\pysigline
{\sphinxbfcode{\sphinxupquote{topodata}}}
\pysigstopsignatures
\sphinxAtStartPar
Instance that handles bathymetry/topography data.
\begin{quote}\begin{description}
\sphinxlineitem{Type}
\sphinxAtStartPar
{\hyperref[\detokenize{celeris:celeris.domain.Topodata}]{\sphinxcrossref{Topodata}}}

\end{description}\end{quote}

\end{fulllineitems}

\index{north\_sl (celeris.domain.Domain attribute)@\spxentry{north\_sl}\spxextra{celeris.domain.Domain attribute}}

\begin{fulllineitems}
\phantomsection\label{\detokenize{celeris:celeris.domain.Domain.north_sl}}
\pysigstartsignatures
\pysigline
{\sphinxbfcode{\sphinxupquote{north\_sl}}}
\pysigstopsignatures
\sphinxAtStartPar
Sea level at the north boundary.
\begin{quote}\begin{description}
\sphinxlineitem{Type}
\sphinxAtStartPar
float

\end{description}\end{quote}

\end{fulllineitems}

\index{south\_sl (celeris.domain.Domain attribute)@\spxentry{south\_sl}\spxextra{celeris.domain.Domain attribute}}

\begin{fulllineitems}
\phantomsection\label{\detokenize{celeris:celeris.domain.Domain.south_sl}}
\pysigstartsignatures
\pysigline
{\sphinxbfcode{\sphinxupquote{south\_sl}}}
\pysigstopsignatures
\sphinxAtStartPar
Sea level at the south boundary.
\begin{quote}\begin{description}
\sphinxlineitem{Type}
\sphinxAtStartPar
float

\end{description}\end{quote}

\end{fulllineitems}

\index{east\_sl (celeris.domain.Domain attribute)@\spxentry{east\_sl}\spxextra{celeris.domain.Domain attribute}}

\begin{fulllineitems}
\phantomsection\label{\detokenize{celeris:celeris.domain.Domain.east_sl}}
\pysigstartsignatures
\pysigline
{\sphinxbfcode{\sphinxupquote{east\_sl}}}
\pysigstopsignatures
\sphinxAtStartPar
Sea level at the east boundary.
\begin{quote}\begin{description}
\sphinxlineitem{Type}
\sphinxAtStartPar
float

\end{description}\end{quote}

\end{fulllineitems}

\index{west\_sl (celeris.domain.Domain attribute)@\spxentry{west\_sl}\spxextra{celeris.domain.Domain attribute}}

\begin{fulllineitems}
\phantomsection\label{\detokenize{celeris:celeris.domain.Domain.west_sl}}
\pysigstartsignatures
\pysigline
{\sphinxbfcode{\sphinxupquote{west\_sl}}}
\pysigstopsignatures
\sphinxAtStartPar
Sea level at the west boundary.
\begin{quote}\begin{description}
\sphinxlineitem{Type}
\sphinxAtStartPar
float

\end{description}\end{quote}

\end{fulllineitems}

\index{Courant (celeris.domain.Domain attribute)@\spxentry{Courant}\spxextra{celeris.domain.Domain attribute}}

\begin{fulllineitems}
\phantomsection\label{\detokenize{celeris:celeris.domain.Domain.Courant}}
\pysigstartsignatures
\pysigline
{\sphinxbfcode{\sphinxupquote{Courant}}}
\pysigstopsignatures
\sphinxAtStartPar
Courant number for numerical stability.
\begin{quote}\begin{description}
\sphinxlineitem{Type}
\sphinxAtStartPar
float

\end{description}\end{quote}

\end{fulllineitems}

\index{isManning (celeris.domain.Domain attribute)@\spxentry{isManning}\spxextra{celeris.domain.Domain attribute}}

\begin{fulllineitems}
\phantomsection\label{\detokenize{celeris:celeris.domain.Domain.isManning}}
\pysigstartsignatures
\pysigline
{\sphinxbfcode{\sphinxupquote{isManning}}}
\pysigstopsignatures
\sphinxAtStartPar
Switch indicating whether Manning friction is used.
\begin{quote}\begin{description}
\sphinxlineitem{Type}
\sphinxAtStartPar
int

\end{description}\end{quote}

\end{fulllineitems}

\index{friction (celeris.domain.Domain attribute)@\spxentry{friction}\spxextra{celeris.domain.Domain attribute}}

\begin{fulllineitems}
\phantomsection\label{\detokenize{celeris:celeris.domain.Domain.friction}}
\pysigstartsignatures
\pysigline
{\sphinxbfcode{\sphinxupquote{friction}}}
\pysigstopsignatures
\sphinxAtStartPar
Friction (e.g., Manning n) value.
\begin{quote}\begin{description}
\sphinxlineitem{Type}
\sphinxAtStartPar
float

\end{description}\end{quote}

\end{fulllineitems}

\index{base\_depth\_ (celeris.domain.Domain attribute)@\spxentry{base\_depth\_}\spxextra{celeris.domain.Domain attribute}}

\begin{fulllineitems}
\phantomsection\label{\detokenize{celeris:celeris.domain.Domain.base_depth_}}
\pysigstartsignatures
\pysigline
{\sphinxbfcode{\sphinxupquote{base\_depth\_}}}
\pysigstopsignatures
\sphinxAtStartPar
Reference base depth of the domain.
If None, it is inferred from the topography.
\begin{quote}\begin{description}
\sphinxlineitem{Type}
\sphinxAtStartPar
float or None

\end{description}\end{quote}

\end{fulllineitems}

\index{Boundary\_shift (celeris.domain.Domain attribute)@\spxentry{Boundary\_shift}\spxextra{celeris.domain.Domain attribute}}

\begin{fulllineitems}
\phantomsection\label{\detokenize{celeris:celeris.domain.Domain.Boundary_shift}}
\pysigstartsignatures
\pysigline
{\sphinxbfcode{\sphinxupquote{Boundary\_shift}}}
\pysigstopsignatures
\sphinxAtStartPar
Shift parameter used for boundary indexing or conditions.
\begin{quote}\begin{description}
\sphinxlineitem{Type}
\sphinxAtStartPar
int

\end{description}\end{quote}

\end{fulllineitems}

\index{pixels (celeris.domain.Domain attribute)@\spxentry{pixels}\spxextra{celeris.domain.Domain attribute}}

\begin{fulllineitems}
\phantomsection\label{\detokenize{celeris:celeris.domain.Domain.pixels}}
\pysigstartsignatures
\pysigline
{\sphinxbfcode{\sphinxupquote{pixels}}}
\pysigstopsignatures
\sphinxAtStartPar
Taichi field for potential 2D visualization or debugging (shape = {[}Nx, Ny{]}).
\begin{quote}\begin{description}
\sphinxlineitem{Type}
\sphinxAtStartPar
ti.field

\end{description}\end{quote}

\end{fulllineitems}

\index{g (celeris.domain.Domain attribute)@\spxentry{g}\spxextra{celeris.domain.Domain attribute}}

\begin{fulllineitems}
\phantomsection\label{\detokenize{celeris:celeris.domain.Domain.g}}
\pysigstartsignatures
\pysigline
{\sphinxbfcode{\sphinxupquote{g}}}
\pysigstopsignatures
\sphinxAtStartPar
Gravitational constant (9.80665).
\begin{quote}\begin{description}
\sphinxlineitem{Type}
\sphinxAtStartPar
float

\end{description}\end{quote}

\end{fulllineitems}

\index{configfile (celeris.domain.Domain attribute)@\spxentry{configfile}\spxextra{celeris.domain.Domain attribute}}

\begin{fulllineitems}
\phantomsection\label{\detokenize{celeris:celeris.domain.Domain.configfile}}
\pysigstartsignatures
\pysigline
{\sphinxbfcode{\sphinxupquote{configfile}}}
\pysigstopsignatures
\sphinxAtStartPar
Loaded JSON dictionary if using Celeris format.
\begin{quote}\begin{description}
\sphinxlineitem{Type}
\sphinxAtStartPar
dict or None

\end{description}\end{quote}

\end{fulllineitems}

\index{seaLevel (celeris.domain.Domain attribute)@\spxentry{seaLevel}\spxextra{celeris.domain.Domain attribute}}

\begin{fulllineitems}
\phantomsection\label{\detokenize{celeris:celeris.domain.Domain.seaLevel}}
\pysigstartsignatures
\pysigline
{\sphinxbfcode{\sphinxupquote{seaLevel}}}
\pysigstopsignatures
\sphinxAtStartPar
Reference sea level (set to 0.0 here).
\begin{quote}\begin{description}
\sphinxlineitem{Type}
\sphinxAtStartPar
float

\end{description}\end{quote}

\end{fulllineitems}

\subsubsection*{Example}

\begin{sphinxVerbatim}[commandchars=\\\{\}]
\PYG{g+gp}{\PYGZgt{}\PYGZgt{}\PYGZgt{} }\PYG{k+kn}{from}\PYG{+w}{ }\PYG{n+nn}{celeris}\PYG{n+nn}{.}\PYG{n+nn}{domain}\PYG{+w}{ }\PYG{k+kn}{import} \PYG{n}{Domain}\PYG{p}{,}\PYG{n}{Topodata}
\PYG{g+gp}{\PYGZgt{}\PYGZgt{}\PYGZgt{} }\PYG{n}{topo} \PYG{o}{=} \PYG{n}{Topodata}\PYG{p}{(}\PYG{n}{filename}\PYG{o}{=}\PYG{l+s+s1}{\PYGZsq{}}\PYG{l+s+s1}{bathy.txt}\PYG{l+s+s1}{\PYGZsq{}}\PYG{p}{,} \PYG{n}{datatype}\PYG{o}{=}\PYG{l+s+s1}{\PYGZsq{}}\PYG{l+s+s1}{xyz}\PYG{l+s+s1}{\PYGZsq{}}\PYG{p}{)}
\PYG{g+gp}{\PYGZgt{}\PYGZgt{}\PYGZgt{} }\PYG{n}{domain} \PYG{o}{=} \PYG{n}{Domain}\PYG{p}{(}\PYG{n}{x1}\PYG{o}{=}\PYG{l+m+mi}{0}\PYG{p}{,} \PYG{n}{x2}\PYG{o}{=}\PYG{l+m+mi}{100}\PYG{p}{,} \PYG{n}{y1}\PYG{o}{=}\PYG{l+m+mi}{0}\PYG{p}{,} \PYG{n}{y2}\PYG{o}{=}\PYG{l+m+mi}{100}\PYG{p}{,} \PYG{n}{Nx}\PYG{o}{=}\PYG{l+m+mi}{100}\PYG{p}{,} \PYG{n}{Ny}\PYG{o}{=}\PYG{l+m+mi}{100}\PYG{p}{,} \PYG{n}{topodata}\PYG{o}{=}\PYG{n}{topo}\PYG{p}{)}
\PYG{g+gp}{\PYGZgt{}\PYGZgt{}\PYGZgt{} }\PYG{n}{xgrid}\PYG{p}{,} \PYG{n}{ygrid}\PYG{p}{,} \PYG{n}{bathy} \PYG{o}{=} \PYG{n}{domain}\PYG{o}{.}\PYG{n}{grid}\PYG{p}{(}\PYG{p}{)}
\PYG{g+gp}{\PYGZgt{}\PYGZgt{}\PYGZgt{} }\PYG{n}{dt} \PYG{o}{=} \PYG{n}{domain}\PYG{o}{.}\PYG{n}{dt}\PYG{p}{(}\PYG{p}{)}
\PYG{g+gp}{\PYGZgt{}\PYGZgt{}\PYGZgt{} }\PYG{n+nb}{print}\PYG{p}{(}\PYG{l+s+sa}{f}\PYG{l+s+s2}{\PYGZdq{}}\PYG{l+s+s2}{Time step: }\PYG{l+s+si}{\PYGZob{}}\PYG{n}{dt}\PYG{l+s+si}{\PYGZcb{}}\PYG{l+s+s2}{\PYGZdq{}}\PYG{p}{)}
\end{sphinxVerbatim}
\index{bottom() (celeris.domain.Domain method)@\spxentry{bottom()}\spxextra{celeris.domain.Domain method}}

\begin{fulllineitems}
\phantomsection\label{\detokenize{celeris:celeris.domain.Domain.bottom}}
\pysigstartsignatures
\pysiglinewithargsret
{\sphinxbfcode{\sphinxupquote{bottom}}}
{}
{}
\pysigstopsignatures
\sphinxAtStartPar
Creates a 3D NumPy array of shape (4, Nx, Ny) to store bottom elevation
(inverted sign), plus any other auxiliary fields (e.g., near\sphinxhyphen{}dry flags).
\begin{description}
\sphinxlineitem{Index mapping:}\begin{itemize}
\item {} 
\sphinxAtStartPar
{[}2, :, :{]} =\textgreater{} Stores the bathymetry/topography (with a \sphinxhyphen{}1 factor).

\item {} 
\sphinxAtStartPar
{[}3, :, :{]} =\textgreater{} A placeholder used for near\sphinxhyphen{}dry or similar state flags.

\end{itemize}

\end{description}
\begin{quote}\begin{description}
\sphinxlineitem{Returns}
\sphinxAtStartPar
A Taichi field of shape {[}4, Nx, Ny{]} with the bottom information.

\sphinxlineitem{Return type}
\sphinxAtStartPar
ti.field

\end{description}\end{quote}

\end{fulllineitems}

\index{dt() (celeris.domain.Domain method)@\spxentry{dt()}\spxextra{celeris.domain.Domain method}}

\begin{fulllineitems}
\phantomsection\label{\detokenize{celeris:celeris.domain.Domain.dt}}
\pysigstartsignatures
\pysiglinewithargsret
{\sphinxbfcode{\sphinxupquote{dt}}}
{}
{}
\pysigstopsignatures\begin{description}
\sphinxlineitem{Computes the time step based on the Courant criterion:}
\sphinxAtStartPar
dt = Courant * dx / sqrt(g * maxdepth)

\end{description}
\begin{quote}\begin{description}
\sphinxlineitem{Returns}
\sphinxAtStartPar
The computed time step.

\sphinxlineitem{Return type}
\sphinxAtStartPar
float

\end{description}\end{quote}

\end{fulllineitems}

\index{grid() (celeris.domain.Domain method)@\spxentry{grid()}\spxextra{celeris.domain.Domain method}}

\begin{fulllineitems}
\phantomsection\label{\detokenize{celeris:celeris.domain.Domain.grid}}
\pysigstartsignatures
\pysiglinewithargsret
{\sphinxbfcode{\sphinxupquote{grid}}}
{}
{}
\pysigstopsignatures
\sphinxAtStartPar
Returns the meshgrid of domain coordinates and corresponding bathymetry/topography.
\begin{quote}\begin{description}
\sphinxlineitem{Returns}
\sphinxAtStartPar
\begin{itemize}
\item {} 
\sphinxAtStartPar
\sphinxstylestrong{xx} (numpy.ndarray): x\sphinxhyphen{}coordinates (shape: Nx x Ny if 2D; Nx if 1D).

\item {} 
\sphinxAtStartPar
\sphinxstylestrong{yy} (numpy.ndarray): y\sphinxhyphen{}coordinates (shape: Nx x Ny if 2D;
or topography for 1D).

\item {} 
\sphinxAtStartPar
\sphinxstylestrong{zz} (numpy.ndarray): Bathymetry/topography data in 2D case;
None or irrelevant in 1D case (depending on interpretation).

\end{itemize}


\sphinxlineitem{Return type}
\sphinxAtStartPar
tuple

\end{description}\end{quote}

\end{fulllineitems}

\index{maxdepth() (celeris.domain.Domain method)@\spxentry{maxdepth()}\spxextra{celeris.domain.Domain method}}

\begin{fulllineitems}
\phantomsection\label{\detokenize{celeris:celeris.domain.Domain.maxdepth}}
\pysigstartsignatures
\pysiglinewithargsret
{\sphinxbfcode{\sphinxupquote{maxdepth}}}
{}
{}
\pysigstopsignatures
\sphinxAtStartPar
Computes the maximum depth in the domain. If \sphinxtitleref{base\_depth\_} is specified,
returns that. Otherwise:
\begin{itemize}
\item {} 
\sphinxAtStartPar
For 1D (‘xz’): returns the maximum of the interpolated topofield array.

\item {} 
\sphinxAtStartPar
For 2D (‘xyz’, ‘celeris’): returns the maximum of topofield array values.

\end{itemize}
\begin{quote}\begin{description}
\sphinxlineitem{Returns}
\sphinxAtStartPar
Maximum depth (\sphinxtitleref{base\_depth\_} if set, else maximum from topofield).

\sphinxlineitem{Return type}
\sphinxAtStartPar
float

\end{description}\end{quote}

\end{fulllineitems}

\index{maxtopo() (celeris.domain.Domain method)@\spxentry{maxtopo()}\spxextra{celeris.domain.Domain method}}

\begin{fulllineitems}
\phantomsection\label{\detokenize{celeris:celeris.domain.Domain.maxtopo}}
\pysigstartsignatures
\pysiglinewithargsret
{\sphinxbfcode{\sphinxupquote{maxtopo}}}
{}
{}
\pysigstopsignatures
\sphinxAtStartPar
Computes the highest topographic elevation in the domain.
For 1D (‘xz’), returns the minimum of the array (assuming negative values
represent depth). For 2D, returns the minimum of the topofield array
for a similar reason.
\begin{quote}\begin{description}
\sphinxlineitem{Returns}
\sphinxAtStartPar
The highest elevation (or least negative) in the domain.

\sphinxlineitem{Return type}
\sphinxAtStartPar
float

\end{description}\end{quote}

\end{fulllineitems}

\index{reflect\_x() (celeris.domain.Domain method)@\spxentry{reflect\_x()}\spxextra{celeris.domain.Domain method}}

\begin{fulllineitems}
\phantomsection\label{\detokenize{celeris:celeris.domain.Domain.reflect_x}}
\pysigstartsignatures
\pysiglinewithargsret
{\sphinxbfcode{\sphinxupquote{reflect\_x}}}
{}
{}
\pysigstopsignatures
\sphinxAtStartPar
Computes an x\sphinxhyphen{}reflection index for enforcing solid boundary conditions.
\begin{quote}\begin{description}
\sphinxlineitem{Returns}
\sphinxAtStartPar
The x\sphinxhyphen{}reflection index (2*(Nx\sphinxhyphen{}3)).

\sphinxlineitem{Return type}
\sphinxAtStartPar
int

\end{description}\end{quote}

\end{fulllineitems}

\index{reflect\_y() (celeris.domain.Domain method)@\spxentry{reflect\_y()}\spxextra{celeris.domain.Domain method}}

\begin{fulllineitems}
\phantomsection\label{\detokenize{celeris:celeris.domain.Domain.reflect_y}}
\pysigstartsignatures
\pysiglinewithargsret
{\sphinxbfcode{\sphinxupquote{reflect\_y}}}
{}
{}
\pysigstopsignatures
\sphinxAtStartPar
Computes a y\sphinxhyphen{}reflection index for enforcing solid boundary conditions.
\begin{quote}\begin{description}
\sphinxlineitem{Returns}
\sphinxAtStartPar
The y\sphinxhyphen{}reflection index (2*(Ny\sphinxhyphen{}3)).

\sphinxlineitem{Return type}
\sphinxAtStartPar
int

\end{description}\end{quote}

\end{fulllineitems}

\index{states() (celeris.domain.Domain method)@\spxentry{states()}\spxextra{celeris.domain.Domain method}}

\begin{fulllineitems}
\phantomsection\label{\detokenize{celeris:celeris.domain.Domain.states}}
\pysigstartsignatures
\pysiglinewithargsret
{\sphinxbfcode{\sphinxupquote{states}}}
{}
{}
\pysigstopsignatures
\sphinxAtStartPar
Creates a Taichi Vector field of shape {[}Nx, Ny{]}, each containing 4 components
(e.g., water depth, momentum in x, momentum in y, and an scalaritional parameter).
\begin{quote}\begin{description}
\sphinxlineitem{Returns}
\sphinxAtStartPar
A 4\sphinxhyphen{}component vector field in Taichi.

\sphinxlineitem{Return type}
\sphinxAtStartPar
ti.types.vector.field

\end{description}\end{quote}

\end{fulllineitems}

\index{states\_one() (celeris.domain.Domain method)@\spxentry{states\_one()}\spxextra{celeris.domain.Domain method}}

\begin{fulllineitems}
\phantomsection\label{\detokenize{celeris:celeris.domain.Domain.states_one}}
\pysigstartsignatures
\pysiglinewithargsret
{\sphinxbfcode{\sphinxupquote{states\_one}}}
{}
{}
\pysigstopsignatures
\sphinxAtStartPar
Creates a Taichi Vector field of shape {[}Nx, Ny{]}, each containing 1 component.
\begin{quote}\begin{description}
\sphinxlineitem{Returns}
\sphinxAtStartPar
A 1\sphinxhyphen{}component vector field in Taichi.

\sphinxlineitem{Return type}
\sphinxAtStartPar
ti.types.vector.field

\end{description}\end{quote}

\end{fulllineitems}

\index{topofield() (celeris.domain.Domain method)@\spxentry{topofield()}\spxextra{celeris.domain.Domain method}}

\begin{fulllineitems}
\phantomsection\label{\detokenize{celeris:celeris.domain.Domain.topofield}}
\pysigstartsignatures
\pysiglinewithargsret
{\sphinxbfcode{\sphinxupquote{topofield}}}
{}
{}
\pysigstopsignatures
\sphinxAtStartPar
Loads and/or interpolates topographic/bathymetric data into a NumPy meshgrid
and returns it in a format suitable for use in the solver.
\begin{quote}\begin{description}
\sphinxlineitem{Returns}
\sphinxAtStartPar
\begin{itemize}
\item {} 
\sphinxAtStartPar
\sphinxstylestrong{x\_out} (numpy.ndarray): Mesh of x\sphinxhyphen{}coordinates for the domain.

\item {} 
\sphinxAtStartPar
\sphinxstylestrong{y\_out} (numpy.ndarray): Mesh of y\sphinxhyphen{}coordinates for the domain
(or 1D array if \sphinxtitleref{datatype=’xz’}).

\item {} 
\sphinxAtStartPar
\sphinxstylestrong{z\_out} (numpy.ndarray): Corresponding bathymetry/topography values.

\end{itemize}


\sphinxlineitem{Return type}
\sphinxAtStartPar
tuple

\sphinxlineitem{Raises}
\sphinxAtStartPar
\sphinxstyleliteralstrong{\sphinxupquote{ValueError}} \textendash{} If \sphinxtitleref{topodata.datatype} is not one of ‘celeris’, ‘xyz’, or ‘xz’.

\end{description}\end{quote}

\end{fulllineitems}


\end{fulllineitems}

\index{checjson() (in module celeris.domain)@\spxentry{checjson()}\spxextra{in module celeris.domain}}

\begin{fulllineitems}
\phantomsection\label{\detokenize{celeris:celeris.domain.checjson}}
\pysigstartsignatures
\pysiglinewithargsret
{\sphinxcode{\sphinxupquote{celeris.domain.}}\sphinxbfcode{\sphinxupquote{checjson}}}
{\sphinxparam{\DUrole{n}{variable}}\sphinxparamcomma \sphinxparam{\DUrole{n}{data}}}
{}
\pysigstopsignatures
\sphinxAtStartPar
Checks if a key exists in a JSON config file (i.e., CelerisWebGPU configuration file).
\begin{quote}\begin{description}
\sphinxlineitem{Parameters}\begin{itemize}
\item {} 
\sphinxAtStartPar
\sphinxstyleliteralstrong{\sphinxupquote{key}} (\sphinxstyleliteralemphasis{\sphinxupquote{str}}) \textendash{} The key to check in the JSON config file.

\item {} 
\sphinxAtStartPar
\sphinxstyleliteralstrong{\sphinxupquote{config}} (\sphinxstyleliteralemphasis{\sphinxupquote{dict}}) \textendash{} The JSON\sphinxhyphen{}loaded dictionary.

\end{itemize}

\sphinxlineitem{Returns}
\sphinxAtStartPar
1 if the key is found in the JSON dictionary, 0 otherwise.

\sphinxlineitem{Return type}
\sphinxAtStartPar
int

\end{description}\end{quote}

\end{fulllineitems}

\index{ti2np() (in module celeris.domain)@\spxentry{ti2np()}\spxextra{in module celeris.domain}}

\begin{fulllineitems}
\phantomsection\label{\detokenize{celeris:celeris.domain.ti2np}}
\pysigstartsignatures
\pysiglinewithargsret
{\sphinxcode{\sphinxupquote{celeris.domain.}}\sphinxbfcode{\sphinxupquote{ti2np}}}
{\sphinxparam{\DUrole{n}{ti\_type}}}
{}
\pysigstopsignatures
\sphinxAtStartPar
Converts a Taichi precision type to a NumPy dtype.
\begin{quote}\begin{description}
\sphinxlineitem{Parameters}
\sphinxAtStartPar
\sphinxstyleliteralstrong{\sphinxupquote{precision}} (\sphinxstyleliteralemphasis{\sphinxupquote{ti.types.primitive\_types}}) \textendash{} Taichi precision type (e.g., ti.f32, ti.f64).

\sphinxlineitem{Returns}
\sphinxAtStartPar
Corresponding NumPy dtype (e.g., np.float32, np.float64).

\sphinxlineitem{Return type}
\sphinxAtStartPar
numpy.dtype

\end{description}\end{quote}

\end{fulllineitems}



\subsubsection{Solver module}
\label{\detokenize{celeris:module-celeris.solver}}\label{\detokenize{celeris:solver-module}}\index{module@\spxentry{module}!celeris.solver@\spxentry{celeris.solver}}\index{celeris.solver@\spxentry{celeris.solver}!module@\spxentry{module}}\index{Solver (class in celeris.solver)@\spxentry{Solver}\spxextra{class in celeris.solver}}

\begin{fulllineitems}
\phantomsection\label{\detokenize{celeris:celeris.solver.Solver}}
\pysigstartsignatures
\pysiglinewithargsret
{\sphinxbfcode{\sphinxupquote{class\DUrole{w}{ }}}\sphinxcode{\sphinxupquote{celeris.solver.}}\sphinxbfcode{\sphinxupquote{Solver}}}
{\sphinxparam{\DUrole{n}{domain=None}}\sphinxparamcomma \sphinxparam{\DUrole{n}{boundary\_conditions=None}}\sphinxparamcomma \sphinxparam{\DUrole{n}{dissipation\_threshold=0.3}}\sphinxparamcomma \sphinxparam{\DUrole{n}{theta=2.0}}\sphinxparamcomma \sphinxparam{\DUrole{n}{timeScheme=2}}\sphinxparamcomma \sphinxparam{\DUrole{n}{pred\_or\_corrector=1}}\sphinxparamcomma \sphinxparam{\DUrole{n}{show\_window=True}}\sphinxparamcomma \sphinxparam{\DUrole{n}{maxsteps=1000}}\sphinxparamcomma \sphinxparam{\DUrole{n}{Bcoef=0.06666666666666667}}\sphinxparamcomma \sphinxparam{\DUrole{n}{outdir=None}}\sphinxparamcomma \sphinxparam{\DUrole{n}{model=\textquotesingle{}SWE\textquotesingle{}}}\sphinxparamcomma \sphinxparam{\DUrole{n}{useBreakingModel=False}}\sphinxparamcomma \sphinxparam{\DUrole{n}{whiteWaterDecayRate=0.01}}\sphinxparamcomma \sphinxparam{\DUrole{n}{whiteWaterDispersion=0.1}}\sphinxparamcomma \sphinxparam{\DUrole{n}{useSedTransModel=False}}\sphinxparamcomma \sphinxparam{\DUrole{n}{sediment=\textless{}celeris.solver.SedClass object\textgreater{}}}\sphinxparamcomma \sphinxparam{\DUrole{n}{infiltrationRate=0.001}}\sphinxparamcomma \sphinxparam{\DUrole{n}{clearCon=1}}\sphinxparamcomma \sphinxparam{\DUrole{n}{showBreaking=0}}\sphinxparamcomma \sphinxparam{\DUrole{n}{delta\_breaking=2.0}}\sphinxparamcomma \sphinxparam{\DUrole{n}{T\_star\_coef=5.0}}\sphinxparamcomma \sphinxparam{\DUrole{n}{dzdt\_I\_coef=0.5}}\sphinxparamcomma \sphinxparam{\DUrole{n}{dzdt\_F\_coef=0.15}}}
{}
\pysigstopsignatures
\sphinxAtStartPar
Main numerical solver class for the CelerisAi model.
\begin{description}
\sphinxlineitem{This class manages the entire simulation process, including:}\begin{itemize}
\item {} 
\sphinxAtStartPar
Initializing solution states and bottom fields from the \sphinxtitleref{Domain} class.

\item {} 
\sphinxAtStartPar
Controlling boundary conditions from the \sphinxtitleref{BoundaryConditions} class.

\item {} 
\sphinxAtStartPar
Executing the time\sphinxhyphen{}stepping scheme (Euler, Adams\sphinxhyphen{}Bashforth variants) and
1D/2D flow updates (SWE or Boussinesq).

\item {} 
\sphinxAtStartPar
Incorporating breaking models, sediment transport, and the
kernels for reconstruction (Pass1), flux computations (Pass2),
and final updates (Pass3).

\end{itemize}

\end{description}
\index{domain (celeris.solver.Solver attribute)@\spxentry{domain}\spxextra{celeris.solver.Solver attribute}}

\begin{fulllineitems}
\phantomsection\label{\detokenize{celeris:celeris.solver.Solver.domain}}
\pysigstartsignatures
\pysigline
{\sphinxbfcode{\sphinxupquote{domain}}}
\pysigstopsignatures
\sphinxAtStartPar
Instance of the Domain class containing grid info and topography.
\begin{quote}\begin{description}
\sphinxlineitem{Type}
\sphinxAtStartPar
{\hyperref[\detokenize{celeris:celeris.domain.Domain}]{\sphinxcrossref{Domain}}}

\end{description}\end{quote}

\end{fulllineitems}

\index{bc (celeris.solver.Solver attribute)@\spxentry{bc}\spxextra{celeris.solver.Solver attribute}}

\begin{fulllineitems}
\phantomsection\label{\detokenize{celeris:celeris.solver.Solver.bc}}
\pysigstartsignatures
\pysigline
{\sphinxbfcode{\sphinxupquote{bc}}}
\pysigstopsignatures
\sphinxAtStartPar
Instance managing boundary condition types and wave settings.
\begin{quote}\begin{description}
\sphinxlineitem{Type}
\sphinxAtStartPar
{\hyperref[\detokenize{celeris:celeris.domain.BoundaryConditions}]{\sphinxcrossref{BoundaryConditions}}}

\end{description}\end{quote}

\end{fulllineitems}

\index{dissipation\_threshold (celeris.solver.Solver attribute)@\spxentry{dissipation\_threshold}\spxextra{celeris.solver.Solver attribute}}

\begin{fulllineitems}
\phantomsection\label{\detokenize{celeris:celeris.solver.Solver.dissipation_threshold}}
\pysigstartsignatures
\pysigline
{\sphinxbfcode{\sphinxupquote{dissipation\_threshold}}}
\pysigstopsignatures
\sphinxAtStartPar
Used for visualization (mark cells above certain foam/dissipation).
\begin{quote}\begin{description}
\sphinxlineitem{Type}
\sphinxAtStartPar
float

\end{description}\end{quote}

\end{fulllineitems}

\index{theta (celeris.solver.Solver attribute)@\spxentry{theta}\spxextra{celeris.solver.Solver attribute}}

\begin{fulllineitems}
\phantomsection\label{\detokenize{celeris:celeris.solver.Solver.theta}}
\pysigstartsignatures
\pysigline
{\sphinxbfcode{\sphinxupquote{theta}}}
\pysigstopsignatures
\sphinxAtStartPar
Midmod limiter parameter (1.0 more dissipative, 2.0 less dissipative).
\begin{quote}\begin{description}
\sphinxlineitem{Type}
\sphinxAtStartPar
float

\end{description}\end{quote}

\end{fulllineitems}

\index{timeScheme (celeris.solver.Solver attribute)@\spxentry{timeScheme}\spxextra{celeris.solver.Solver attribute}}

\begin{fulllineitems}
\phantomsection\label{\detokenize{celeris:celeris.solver.Solver.timeScheme}}
\pysigstartsignatures
\pysigline
{\sphinxbfcode{\sphinxupquote{timeScheme}}}
\pysigstopsignatures
\sphinxAtStartPar
Time integration scheme:
\sphinxhyphen{} 0 =\textgreater{} Euler
\sphinxhyphen{} 1 =\textgreater{} 3rd\sphinxhyphen{}order  predictor
\sphinxhyphen{} 2 =\textgreater{} 4th\sphinxhyphen{}order  predictor/corrector
\begin{quote}\begin{description}
\sphinxlineitem{Type}
\sphinxAtStartPar
int

\end{description}\end{quote}

\end{fulllineitems}

\index{pred\_or\_corrector (celeris.solver.Solver attribute)@\spxentry{pred\_or\_corrector}\spxextra{celeris.solver.Solver attribute}}

\begin{fulllineitems}
\phantomsection\label{\detokenize{celeris:celeris.solver.Solver.pred_or_corrector}}
\pysigstartsignatures
\pysigline
{\sphinxbfcode{\sphinxupquote{pred\_or\_corrector}}}
\pysigstopsignatures
\sphinxAtStartPar
Indicates stage in solver loop (1 =\textgreater{} predictor, 2 =\textgreater{} corrector).
\begin{quote}\begin{description}
\sphinxlineitem{Type}
\sphinxAtStartPar
int

\end{description}\end{quote}

\end{fulllineitems}

\index{Bcoef (celeris.solver.Solver attribute)@\spxentry{Bcoef}\spxextra{celeris.solver.Solver attribute}}

\begin{fulllineitems}
\phantomsection\label{\detokenize{celeris:celeris.solver.Solver.Bcoef}}
\pysigstartsignatures
\pysigline
{\sphinxbfcode{\sphinxupquote{Bcoef}}}
\pysigstopsignatures
\sphinxAtStartPar
Dispersion parameter for Boussinesq model; default 1/15.
\begin{quote}\begin{description}
\sphinxlineitem{Type}
\sphinxAtStartPar
float

\end{description}\end{quote}

\end{fulllineitems}

\index{model (celeris.solver.Solver attribute)@\spxentry{model}\spxextra{celeris.solver.Solver attribute}}

\begin{fulllineitems}
\phantomsection\label{\detokenize{celeris:celeris.solver.Solver.model}}
\pysigstartsignatures
\pysigline
{\sphinxbfcode{\sphinxupquote{model}}}
\pysigstopsignatures
\sphinxAtStartPar
Type of model, ‘SWE’ or ‘Bouss’.
\begin{quote}\begin{description}
\sphinxlineitem{Type}
\sphinxAtStartPar
str

\end{description}\end{quote}

\end{fulllineitems}

\index{useBreakingModel (celeris.solver.Solver attribute)@\spxentry{useBreakingModel}\spxextra{celeris.solver.Solver attribute}}

\begin{fulllineitems}
\phantomsection\label{\detokenize{celeris:celeris.solver.Solver.useBreakingModel}}
\pysigstartsignatures
\pysigline
{\sphinxbfcode{\sphinxupquote{useBreakingModel}}}
\pysigstopsignatures
\sphinxAtStartPar
True if wave\sphinxhyphen{}breaking model is included.
\begin{quote}\begin{description}
\sphinxlineitem{Type}
\sphinxAtStartPar
bool

\end{description}\end{quote}

\end{fulllineitems}

\index{whiteWaterDecayRate (celeris.solver.Solver attribute)@\spxentry{whiteWaterDecayRate}\spxextra{celeris.solver.Solver attribute}}

\begin{fulllineitems}
\phantomsection\label{\detokenize{celeris:celeris.solver.Solver.whiteWaterDecayRate}}
\pysigstartsignatures
\pysigline
{\sphinxbfcode{\sphinxupquote{whiteWaterDecayRate}}}
\pysigstopsignatures
\sphinxAtStartPar
Turbulence decay rate for foam (visualization).
\begin{quote}\begin{description}
\sphinxlineitem{Type}
\sphinxAtStartPar
float

\end{description}\end{quote}

\end{fulllineitems}

\index{whiteWaterDispersion (celeris.solver.Solver attribute)@\spxentry{whiteWaterDispersion}\spxextra{celeris.solver.Solver attribute}}

\begin{fulllineitems}
\phantomsection\label{\detokenize{celeris:celeris.solver.Solver.whiteWaterDispersion}}
\pysigstartsignatures
\pysigline
{\sphinxbfcode{\sphinxupquote{whiteWaterDispersion}}}
\pysigstopsignatures
\sphinxAtStartPar
Turbulence dispersion factor.
\begin{quote}\begin{description}
\sphinxlineitem{Type}
\sphinxAtStartPar
float

\end{description}\end{quote}

\end{fulllineitems}

\index{useSedTransModel (celeris.solver.Solver attribute)@\spxentry{useSedTransModel}\spxextra{celeris.solver.Solver attribute}}

\begin{fulllineitems}
\phantomsection\label{\detokenize{celeris:celeris.solver.Solver.useSedTransModel}}
\pysigstartsignatures
\pysigline
{\sphinxbfcode{\sphinxupquote{useSedTransModel}}}
\pysigstopsignatures
\sphinxAtStartPar
True if sediment transport is included.
\begin{quote}\begin{description}
\sphinxlineitem{Type}
\sphinxAtStartPar
bool

\end{description}\end{quote}

\end{fulllineitems}

\index{sediment (celeris.solver.Solver attribute)@\spxentry{sediment}\spxextra{celeris.solver.Solver attribute}}

\begin{fulllineitems}
\phantomsection\label{\detokenize{celeris:celeris.solver.Solver.sediment}}
\pysigstartsignatures
\pysigline
{\sphinxbfcode{\sphinxupquote{sediment}}}
\pysigstopsignatures
\sphinxAtStartPar
Default or custom sediment parameters.
\begin{quote}\begin{description}
\sphinxlineitem{Type}
\sphinxAtStartPar
object

\end{description}\end{quote}

\end{fulllineitems}

\index{infiltrationRate (celeris.solver.Solver attribute)@\spxentry{infiltrationRate}\spxextra{celeris.solver.Solver attribute}}

\begin{fulllineitems}
\phantomsection\label{\detokenize{celeris:celeris.solver.Solver.infiltrationRate}}
\pysigstartsignatures
\pysigline
{\sphinxbfcode{\sphinxupquote{infiltrationRate}}}
\pysigstopsignatures
\sphinxAtStartPar
For modeling infiltration on dry beaches.
\begin{quote}\begin{description}
\sphinxlineitem{Type}
\sphinxAtStartPar
float

\end{description}\end{quote}

\end{fulllineitems}

\index{clearCon (celeris.solver.Solver attribute)@\spxentry{clearCon}\spxextra{celeris.solver.Solver attribute}}

\begin{fulllineitems}
\phantomsection\label{\detokenize{celeris:celeris.solver.Solver.clearCon}}
\pysigstartsignatures
\pysigline
{\sphinxbfcode{\sphinxupquote{clearCon}}}
\pysigstopsignatures
\sphinxAtStartPar
If 1, concentration channel is cleared for visualization.
\begin{quote}\begin{description}
\sphinxlineitem{Type}
\sphinxAtStartPar
int

\end{description}\end{quote}

\end{fulllineitems}

\index{showBreaking (celeris.solver.Solver attribute)@\spxentry{showBreaking}\spxextra{celeris.solver.Solver attribute}}

\begin{fulllineitems}
\phantomsection\label{\detokenize{celeris:celeris.solver.Solver.showBreaking}}
\pysigstartsignatures
\pysigline
{\sphinxbfcode{\sphinxupquote{showBreaking}}}
\pysigstopsignatures
\sphinxAtStartPar
If \textgreater{} 0, shows wave breaking foam areas.
\begin{quote}\begin{description}
\sphinxlineitem{Type}
\sphinxAtStartPar
int

\end{description}\end{quote}

\end{fulllineitems}

\index{delta\_breaking (celeris.solver.Solver attribute)@\spxentry{delta\_breaking}\spxextra{celeris.solver.Solver attribute}}

\begin{fulllineitems}
\phantomsection\label{\detokenize{celeris:celeris.solver.Solver.delta_breaking}}
\pysigstartsignatures
\pysigline
{\sphinxbfcode{\sphinxupquote{delta\_breaking}}}
\pysigstopsignatures
\sphinxAtStartPar
Eddy viscosity coefficient in breaking zones.
\begin{quote}\begin{description}
\sphinxlineitem{Type}
\sphinxAtStartPar
float

\end{description}\end{quote}

\end{fulllineitems}

\index{T\_star\_coef (celeris.solver.Solver attribute)@\spxentry{T\_star\_coef}\spxextra{celeris.solver.Solver attribute}}

\begin{fulllineitems}
\phantomsection\label{\detokenize{celeris:celeris.solver.Solver.T_star_coef}}
\pysigstartsignatures
\pysigline
{\sphinxbfcode{\sphinxupquote{T\_star\_coef}}}
\pysigstopsignatures
\sphinxAtStartPar
Timescale factor for fully developed breaking.
\begin{quote}\begin{description}
\sphinxlineitem{Type}
\sphinxAtStartPar
float

\end{description}\end{quote}

\end{fulllineitems}

\index{dzdt\_I\_coef (celeris.solver.Solver attribute)@\spxentry{dzdt\_I\_coef}\spxextra{celeris.solver.Solver attribute}}

\begin{fulllineitems}
\phantomsection\label{\detokenize{celeris:celeris.solver.Solver.dzdt_I_coef}}
\pysigstartsignatures
\pysigline
{\sphinxbfcode{\sphinxupquote{dzdt\_I\_coef}}}
\pysigstopsignatures
\sphinxAtStartPar
Start\sphinxhyphen{}breaking parameter threshold.
\begin{quote}\begin{description}
\sphinxlineitem{Type}
\sphinxAtStartPar
float

\end{description}\end{quote}

\end{fulllineitems}

\index{dzdt\_F\_coef (celeris.solver.Solver attribute)@\spxentry{dzdt\_F\_coef}\spxextra{celeris.solver.Solver attribute}}

\begin{fulllineitems}
\phantomsection\label{\detokenize{celeris:celeris.solver.Solver.dzdt_F_coef}}
\pysigstartsignatures
\pysigline
{\sphinxbfcode{\sphinxupquote{dzdt\_F\_coef}}}
\pysigstopsignatures
\sphinxAtStartPar
End\sphinxhyphen{}breaking parameter threshold.
\begin{quote}\begin{description}
\sphinxlineitem{Type}
\sphinxAtStartPar
float

\end{description}\end{quote}

\end{fulllineitems}


\sphinxAtStartPar
The class initializes a large set of Taichi vector fields (state vectors, flux arrays,
intermediate arrays for Boussinesq tridiagonal solves, sediment transport arrays, etc.)
to manage the numerical solution.
\subsubsection*{Example}

\begin{sphinxVerbatim}[commandchars=\\\{\}]
\PYG{g+gp}{\PYGZgt{}\PYGZgt{}\PYGZgt{} }\PYG{n}{solver} \PYG{o}{=} \PYG{n}{Solver}\PYG{p}{(}\PYG{n}{domain}\PYG{o}{=}\PYG{n}{dom}\PYG{p}{,} \PYG{n}{boundary\PYGZus{}conditions}\PYG{o}{=}\PYG{n}{bc}\PYG{p}{,} \PYG{n}{model}\PYG{o}{=}\PYG{l+s+s1}{\PYGZsq{}}\PYG{l+s+s1}{SWE}\PYG{l+s+s1}{\PYGZsq{}}\PYG{p}{)}
\end{sphinxVerbatim}
\index{BoundSineWaves() (celeris.solver.Solver method)@\spxentry{BoundSineWaves()}\spxextra{celeris.solver.Solver method}}

\begin{fulllineitems}
\phantomsection\label{\detokenize{celeris:celeris.solver.Solver.BoundSineWaves}}
\pysigstartsignatures
\pysiglinewithargsret
{\sphinxbfcode{\sphinxupquote{BoundSineWaves}}}
{\sphinxparam{\DUrole{n}{NumWaves}}\sphinxparamcomma \sphinxparam{\DUrole{n}{Waves}}\sphinxparamcomma \sphinxparam{\DUrole{n}{x}}\sphinxparamcomma \sphinxparam{\DUrole{n}{y}}\sphinxparamcomma \sphinxparam{\DUrole{n}{t}}\sphinxparamcomma \sphinxparam{\DUrole{n}{d\_here}}\sphinxparamcomma \sphinxparam{\DUrole{n}{grav}}}
{}
\pysigstopsignatures
\sphinxAtStartPar
Computes boundary conditions for incoming sine waves at a domain boundary.
\begin{quote}\begin{description}
\sphinxlineitem{Parameters}\begin{itemize}
\item {} 
\sphinxAtStartPar
\sphinxstyleliteralstrong{\sphinxupquote{NumWaves}} (\sphinxstyleliteralemphasis{\sphinxupquote{int}}) \textendash{} Number of wave components in \sphinxtitleref{Waves}.

\item {} 
\sphinxAtStartPar
\sphinxstyleliteralstrong{\sphinxupquote{Waves}} (\sphinxstyleliteralemphasis{\sphinxupquote{ti.field}}) \textendash{} Wave parameter array {[}numWaves, 4{]}.

\item {} 
\sphinxAtStartPar
\sphinxstyleliteralstrong{\sphinxupquote{x}} (\sphinxstyleliteralemphasis{\sphinxupquote{float}}) \textendash{} x\sphinxhyphen{}coordinate at boundary cell.

\item {} 
\sphinxAtStartPar
\sphinxstyleliteralstrong{\sphinxupquote{y}} (\sphinxstyleliteralemphasis{\sphinxupquote{float}}) \textendash{} y\sphinxhyphen{}coordinate at boundary cell.

\item {} 
\sphinxAtStartPar
\sphinxstyleliteralstrong{\sphinxupquote{t}} (\sphinxstyleliteralemphasis{\sphinxupquote{float}}) \textendash{} Current time.

\item {} 
\sphinxAtStartPar
\sphinxstyleliteralstrong{\sphinxupquote{d\_here}} (\sphinxstyleliteralemphasis{\sphinxupquote{float}}) \textendash{} Local water depth if positive; 0 if dry.

\item {} 
\sphinxAtStartPar
\sphinxstyleliteralstrong{\sphinxupquote{grav}} (\sphinxstyleliteralemphasis{\sphinxupquote{float}}) \textendash{} Gravity constant.

\end{itemize}

\sphinxlineitem{Returns}
\sphinxAtStartPar
{[}eta, hu, hv{]} aggregated from all wave components.

\sphinxlineitem{Return type}
\sphinxAtStartPar
ti.types.vector(3, float)

\end{description}\end{quote}

\end{fulllineitems}

\index{BoundaryPass() (celeris.solver.Solver method)@\spxentry{BoundaryPass()}\spxextra{celeris.solver.Solver method}}

\begin{fulllineitems}
\phantomsection\label{\detokenize{celeris:celeris.solver.Solver.BoundaryPass}}
\pysigstartsignatures
\pysiglinewithargsret
{\sphinxbfcode{\sphinxupquote{BoundaryPass}}}
{\sphinxparam{\DUrole{n}{time}\DUrole{p}{:}\DUrole{w}{ }\DUrole{n}{ti.f32}}\sphinxparamcomma \sphinxparam{\DUrole{n}{txState}\DUrole{p}{:}\DUrole{w}{ }\DUrole{n}{DummyTemplate}}}
{}
\pysigstopsignatures\begin{description}
\sphinxlineitem{Updates boundary cells with the appropriate boundary conditions:}\begin{itemize}
\item {} 
\sphinxAtStartPar
Sponge layers (type=1)

\item {} 
\sphinxAtStartPar
Solid walls (type=0)

\item {} 
\sphinxAtStartPar
Incoming waves (type=2) including sine wave or solitary wave

\end{itemize}

\end{description}

\sphinxAtStartPar
This kernel is also responsible for handling near\sphinxhyphen{}dry logic and simple
wetting/drying checks at the domain edges.
\begin{quote}\begin{description}
\sphinxlineitem{Parameters}\begin{itemize}
\item {} 
\sphinxAtStartPar
\sphinxstyleliteralstrong{\sphinxupquote{time}} (\sphinxstyleliteralemphasis{\sphinxupquote{float}}) \textendash{} Current time.

\item {} 
\sphinxAtStartPar
\sphinxstyleliteralstrong{\sphinxupquote{txState}} (\sphinxstyleliteralemphasis{\sphinxupquote{ti.field}}) \textendash{} A Taichi 2D vector field for the state to be updated.

\end{itemize}

\end{description}\end{quote}

\end{fulllineitems}

\index{InitStates() (celeris.solver.Solver method)@\spxentry{InitStates()}\spxextra{celeris.solver.Solver method}}

\begin{fulllineitems}
\phantomsection\label{\detokenize{celeris:celeris.solver.Solver.InitStates}}
\pysigstartsignatures
\pysiglinewithargsret
{\sphinxbfcode{\sphinxupquote{InitStates}}}
{}
{}
\pysigstopsignatures
\sphinxAtStartPar
Initializes the solver states (State, stateUVstar) to zeros at the start
of the simulation.

\end{fulllineitems}

\index{Pass1() (celeris.solver.Solver method)@\spxentry{Pass1()}\spxextra{celeris.solver.Solver method}}

\begin{fulllineitems}
\phantomsection\label{\detokenize{celeris:celeris.solver.Solver.Pass1}}
\pysigstartsignatures
\pysiglinewithargsret
{\sphinxbfcode{\sphinxupquote{Pass1}}}
{\sphinxparam{\DUrole{n}{step}\DUrole{p}{:}\DUrole{w}{ }\DUrole{n}{ti.f32}}}
{}
\pysigstopsignatures\begin{description}
\sphinxlineitem{Reconstruction step (Pass1):}\begin{itemize}
\item {} 
\sphinxAtStartPar
Builds left/right (or N/E/S/W) interface values of eta, momentum, and
scalar concentration using a generalized minmod limiter.

\item {} 
\sphinxAtStartPar
Applies near\sphinxhyphen{}dry checks to skip processing cells that are effectively dry.

\item {} 
\sphinxAtStartPar
Computes velocity components and partial Froude\sphinxhyphen{}limiter logic.

\end{itemize}

\end{description}

\sphinxAtStartPar
For 1D (ny=1), a simpler logic is used. For 2D, reconstruction is in both
x\sphinxhyphen{} and y\sphinxhyphen{}directions.
\begin{quote}\begin{description}
\sphinxlineitem{Parameters}
\sphinxAtStartPar
\sphinxstyleliteralstrong{\sphinxupquote{step}} (\sphinxstyleliteralemphasis{\sphinxupquote{int}}) \textendash{} Counter to compute statistics.

\end{description}\end{quote}

\end{fulllineitems}

\index{Pass1\_SedTrans() (celeris.solver.Solver method)@\spxentry{Pass1\_SedTrans()}\spxextra{celeris.solver.Solver method}}

\begin{fulllineitems}
\phantomsection\label{\detokenize{celeris:celeris.solver.Solver.Pass1_SedTrans}}
\pysigstartsignatures
\pysiglinewithargsret
{\sphinxbfcode{\sphinxupquote{Pass1\_SedTrans}}}
{}
{}
\pysigstopsignatures
\sphinxAtStartPar
Reconstruction step (Pass1) for sediment transport scalar. Uses the same
generalized minmod approach to reconstruct sediment concentration at edges.

\end{fulllineitems}

\index{Pass2() (celeris.solver.Solver method)@\spxentry{Pass2()}\spxextra{celeris.solver.Solver method}}

\begin{fulllineitems}
\phantomsection\label{\detokenize{celeris:celeris.solver.Solver.Pass2}}
\pysigstartsignatures
\pysiglinewithargsret
{\sphinxbfcode{\sphinxupquote{Pass2}}}
{}
{}
\pysigstopsignatures\begin{description}
\sphinxlineitem{Flux computation step (Pass2):}\begin{itemize}
\item {} 
\sphinxAtStartPar
Computes fluxes at each cell edge in x and y directions using the function
numerical flux.

\item {} 
\sphinxAtStartPar
If sediment transport is enabled, calculates sediment fluxes similarly.

\end{itemize}

\end{description}

\end{fulllineitems}

\index{Pass3() (celeris.solver.Solver method)@\spxentry{Pass3()}\spxextra{celeris.solver.Solver method}}

\begin{fulllineitems}
\phantomsection\label{\detokenize{celeris:celeris.solver.Solver.Pass3}}
\pysigstartsignatures
\pysiglinewithargsret
{\sphinxbfcode{\sphinxupquote{Pass3}}}
{\sphinxparam{\DUrole{n}{pred\_or\_corrector}\DUrole{p}{:}\DUrole{w}{ }\DUrole{n}{ti.f32}}}
{}
\pysigstopsignatures\begin{description}
\sphinxlineitem{Main time\sphinxhyphen{}update step (Pass3) for the SWE model:}\begin{itemize}
\item {} 
\sphinxAtStartPar
Updates eta, hu, hv, c for each cell based on fluxes and source terms
(bed slope, friction, infiltration, etc.).

\item {} 
\sphinxAtStartPar
Uses an explicit time scheme (Euler,  predictor,  predictor/corrector).

\item {} 
\sphinxAtStartPar
Optionally includes a foam/breaking parameter if showBreaking\textgreater{}0.

\end{itemize}

\end{description}
\begin{quote}\begin{description}
\sphinxlineitem{Parameters}
\sphinxAtStartPar
\sphinxstyleliteralstrong{\sphinxupquote{pred\_or\_corrector}} (\sphinxstyleliteralemphasis{\sphinxupquote{int}}) \textendash{} Stage in predictor\sphinxhyphen{}corrector scheme (1 =\textgreater{} predictor, 2 =\textgreater{} corrector).

\end{description}\end{quote}

\end{fulllineitems}

\index{Pass3Bous() (celeris.solver.Solver method)@\spxentry{Pass3Bous()}\spxextra{celeris.solver.Solver method}}

\begin{fulllineitems}
\phantomsection\label{\detokenize{celeris:celeris.solver.Solver.Pass3Bous}}
\pysigstartsignatures
\pysiglinewithargsret
{\sphinxbfcode{\sphinxupquote{Pass3Bous}}}
{\sphinxparam{\DUrole{n}{pred\_or\_corrector}\DUrole{p}{:}\DUrole{w}{ }\DUrole{n}{ti.f32}}}
{}
\pysigstopsignatures\begin{description}
\sphinxlineitem{Time\sphinxhyphen{}update step (Pass3) for the Boussinesq model:}\begin{itemize}
\item {} 
\sphinxAtStartPar
Updates wave height, hu, hv, concentration.

\item {} 
\sphinxAtStartPar
Incorporates dispersion terms, bed slope, friction, infiltration,
and wave breaking if enabled.

\item {} 
\sphinxAtStartPar
Uses a predictor\sphinxhyphen{}corrector approach for higher\sphinxhyphen{}order accuracy.

\end{itemize}

\end{description}
\begin{quote}\begin{description}
\sphinxlineitem{Parameters}
\sphinxAtStartPar
\sphinxstyleliteralstrong{\sphinxupquote{pred\_or\_corrector}} (\sphinxstyleliteralemphasis{\sphinxupquote{int}}) \textendash{} Stage in predictor\sphinxhyphen{}corrector scheme (1 =\textgreater{} predictor, 2 =\textgreater{} corrector).

\end{description}\end{quote}

\end{fulllineitems}

\index{Pass3\_SedTrans() (celeris.solver.Solver method)@\spxentry{Pass3\_SedTrans()}\spxextra{celeris.solver.Solver method}}

\begin{fulllineitems}
\phantomsection\label{\detokenize{celeris:celeris.solver.Solver.Pass3_SedTrans}}
\pysigstartsignatures
\pysiglinewithargsret
{\sphinxbfcode{\sphinxupquote{Pass3\_SedTrans}}}
{\sphinxparam{\DUrole{n}{pred\_or\_corrector}\DUrole{p}{:}\DUrole{w}{ }\DUrole{n}{ti.f32}}}
{}
\pysigstopsignatures\begin{description}
\sphinxlineitem{Time\sphinxhyphen{}update step (Pass3) for sediment transport scalar:}\begin{itemize}
\item {} 
\sphinxAtStartPar
Uses fluxes (XFlux\_Sed, YFlux\_Sed) plus simple diffusion.

\item {} 
\sphinxAtStartPar
Adds erosion/deposition terms based on local shear velocity or critical Shields.

\end{itemize}

\end{description}
\begin{quote}\begin{description}
\sphinxlineitem{Parameters}
\sphinxAtStartPar
\sphinxstyleliteralstrong{\sphinxupquote{pred\_or\_corrector}} (\sphinxstyleliteralemphasis{\sphinxupquote{int}}) \textendash{} Stage in predictor\sphinxhyphen{}corrector scheme (1 =\textgreater{} predictor, 2 =\textgreater{} corrector).

\end{description}\end{quote}

\end{fulllineitems}

\index{Pass\_Breaking() (celeris.solver.Solver method)@\spxentry{Pass\_Breaking()}\spxextra{celeris.solver.Solver method}}

\begin{fulllineitems}
\phantomsection\label{\detokenize{celeris:celeris.solver.Solver.Pass_Breaking}}
\pysigstartsignatures
\pysiglinewithargsret
{\sphinxbfcode{\sphinxupquote{Pass\_Breaking}}}
{\sphinxparam{\DUrole{n}{time}\DUrole{p}{:}\DUrole{w}{ }\DUrole{n}{ti.f32}}}
{}
\pysigstopsignatures\begin{description}
\sphinxlineitem{Wave\sphinxhyphen{}breaking model step (used if useBreakingModel == True):}\begin{itemize}
\item {} 
\sphinxAtStartPar
Applies Kennedy et al. or similar wave breaking logic to compute local
dissipation flux and update the Breaking field.

\item {} 
\sphinxAtStartPar
Incorporates eddy viscosity effects from breaking.

\end{itemize}

\end{description}
\begin{quote}\begin{description}
\sphinxlineitem{Parameters}
\sphinxAtStartPar
\sphinxstyleliteralstrong{\sphinxupquote{time}} (\sphinxstyleliteralemphasis{\sphinxupquote{float}}) \textendash{} Current simulation time.

\end{description}\end{quote}

\end{fulllineitems}

\index{Run\_Tridiag\_solver() (celeris.solver.Solver method)@\spxentry{Run\_Tridiag\_solver()}\spxextra{celeris.solver.Solver method}}

\begin{fulllineitems}
\phantomsection\label{\detokenize{celeris:celeris.solver.Solver.Run_Tridiag_solver}}
\pysigstartsignatures
\pysiglinewithargsret
{\sphinxbfcode{\sphinxupquote{Run\_Tridiag\_solver}}}
{}
{}
\pysigstopsignatures
\sphinxAtStartPar
Executes the parallel cyclic reduction (PCR) solver to handle the dispersion
terms in the Boussinesq model. If model is ‘SWE’, no action is taken.

\sphinxAtStartPar
For 1D (ny=1), only the x\sphinxhyphen{}direction solver is applied.
For 2D, runs PCR in x, then y directions.

\end{fulllineitems}

\index{SolitaryWave() (celeris.solver.Solver method)@\spxentry{SolitaryWave()}\spxextra{celeris.solver.Solver method}}

\begin{fulllineitems}
\phantomsection\label{\detokenize{celeris:celeris.solver.Solver.SolitaryWave}}
\pysigstartsignatures
\pysiglinewithargsret
{\sphinxbfcode{\sphinxupquote{SolitaryWave}}}
{\sphinxparam{\DUrole{n}{x0}}\sphinxparamcomma \sphinxparam{\DUrole{n}{y0}}\sphinxparamcomma \sphinxparam{\DUrole{n}{theta}}\sphinxparamcomma \sphinxparam{\DUrole{n}{x}}\sphinxparamcomma \sphinxparam{\DUrole{n}{y}}\sphinxparamcomma \sphinxparam{\DUrole{n}{t}}\sphinxparamcomma \sphinxparam{\DUrole{n}{d\_here}}}
{}
\pysigstopsignatures
\sphinxAtStartPar
Computes boundary conditions for a solitary wave type (WaveType=3).
\begin{quote}\begin{description}
\sphinxlineitem{Parameters}\begin{itemize}
\item {} 
\sphinxAtStartPar
\sphinxstyleliteralstrong{\sphinxupquote{x0}} (\sphinxstyleliteralemphasis{\sphinxupquote{float}}) \textendash{} Initial x\sphinxhyphen{}position of the wave crest.

\item {} 
\sphinxAtStartPar
\sphinxstyleliteralstrong{\sphinxupquote{y0}} (\sphinxstyleliteralemphasis{\sphinxupquote{float}}) \textendash{} Initial y\sphinxhyphen{}position of the wave crest.

\item {} 
\sphinxAtStartPar
\sphinxstyleliteralstrong{\sphinxupquote{theta}} (\sphinxstyleliteralemphasis{\sphinxupquote{float}}) \textendash{} Wave propagation angle in radians.

\item {} 
\sphinxAtStartPar
\sphinxstyleliteralstrong{\sphinxupquote{x}} (\sphinxstyleliteralemphasis{\sphinxupquote{float}}) \textendash{} x\sphinxhyphen{}coordinate at boundary cell.

\item {} 
\sphinxAtStartPar
\sphinxstyleliteralstrong{\sphinxupquote{y}} (\sphinxstyleliteralemphasis{\sphinxupquote{float}}) \textendash{} y\sphinxhyphen{}coordinate at boundary cell.

\item {} 
\sphinxAtStartPar
\sphinxstyleliteralstrong{\sphinxupquote{t}} (\sphinxstyleliteralemphasis{\sphinxupquote{float}}) \textendash{} Current simulation time.

\item {} 
\sphinxAtStartPar
\sphinxstyleliteralstrong{\sphinxupquote{d\_here}} (\sphinxstyleliteralemphasis{\sphinxupquote{float}}) \textendash{} Local water depth, if any.

\end{itemize}

\sphinxlineitem{Returns}
\sphinxAtStartPar
(eta, hu, hv) for a solitary wave boundary.

\sphinxlineitem{Return type}
\sphinxAtStartPar
tuple of (float, float, float)

\end{description}\end{quote}

\end{fulllineitems}

\index{TriDiag\_PCRx() (celeris.solver.Solver method)@\spxentry{TriDiag\_PCRx()}\spxextra{celeris.solver.Solver method}}

\begin{fulllineitems}
\phantomsection\label{\detokenize{celeris:celeris.solver.Solver.TriDiag_PCRx}}
\pysigstartsignatures
\pysiglinewithargsret
{\sphinxbfcode{\sphinxupquote{TriDiag\_PCRx}}}
{\sphinxparam{\DUrole{n}{p}\DUrole{p}{:}\DUrole{w}{ }\DUrole{n}{int}}\sphinxparamcomma \sphinxparam{\DUrole{n}{s}\DUrole{p}{:}\DUrole{w}{ }\DUrole{n}{int}}\sphinxparamcomma \sphinxparam{\DUrole{n}{current\_buffer}\DUrole{p}{:}\DUrole{w}{ }\DUrole{n}{DummyTemplate}}\sphinxparamcomma \sphinxparam{\DUrole{n}{next\_buffer}\DUrole{p}{:}\DUrole{w}{ }\DUrole{n}{DummyTemplate}}}
{}
\pysigstopsignatures
\sphinxAtStartPar
Parallel Cyclic Reduction step in x\sphinxhyphen{}direction for tridiagonal system
(Boussinesq dispersion). This kernel performs the p\sphinxhyphen{}th level of reduction.
\begin{quote}\begin{description}
\sphinxlineitem{Parameters}\begin{itemize}
\item {} 
\sphinxAtStartPar
\sphinxstyleliteralstrong{\sphinxupquote{p}} (\sphinxstyleliteralemphasis{\sphinxupquote{int}}) \textendash{} Current level in PCR (log2 scale).

\item {} 
\sphinxAtStartPar
\sphinxstyleliteralstrong{\sphinxupquote{s}} (\sphinxstyleliteralemphasis{\sphinxupquote{int}}) \textendash{} Step size (2\textasciicircum{}p).

\item {} 
\sphinxAtStartPar
\sphinxstyleliteralstrong{\sphinxupquote{current\_buffer}} (\sphinxstyleliteralemphasis{\sphinxupquote{ti.field}}) \textendash{} Current coefficients at p\sphinxhyphen{}1 level.

\item {} 
\sphinxAtStartPar
\sphinxstyleliteralstrong{\sphinxupquote{next\_buffer}} (\sphinxstyleliteralemphasis{\sphinxupquote{ti.field}}) \textendash{} Next coefficients to be filled for p\sphinxhyphen{}th level.

\end{itemize}

\end{description}\end{quote}

\end{fulllineitems}

\index{TriDiag\_PCRy() (celeris.solver.Solver method)@\spxentry{TriDiag\_PCRy()}\spxextra{celeris.solver.Solver method}}

\begin{fulllineitems}
\phantomsection\label{\detokenize{celeris:celeris.solver.Solver.TriDiag_PCRy}}
\pysigstartsignatures
\pysiglinewithargsret
{\sphinxbfcode{\sphinxupquote{TriDiag\_PCRy}}}
{\sphinxparam{\DUrole{n}{p}\DUrole{p}{:}\DUrole{w}{ }\DUrole{n}{int}}\sphinxparamcomma \sphinxparam{\DUrole{n}{s}\DUrole{p}{:}\DUrole{w}{ }\DUrole{n}{int}}\sphinxparamcomma \sphinxparam{\DUrole{n}{current\_buffer}\DUrole{p}{:}\DUrole{w}{ }\DUrole{n}{DummyTemplate}}\sphinxparamcomma \sphinxparam{\DUrole{n}{next\_buffer}\DUrole{p}{:}\DUrole{w}{ }\DUrole{n}{DummyTemplate}}}
{}
\pysigstopsignatures
\sphinxAtStartPar
Parallel Cyclic Reduction step in y\sphinxhyphen{}direction for tridiagonal system
(Boussinesq dispersion).
\begin{quote}\begin{description}
\sphinxlineitem{Parameters}\begin{itemize}
\item {} 
\sphinxAtStartPar
\sphinxstyleliteralstrong{\sphinxupquote{p}} (\sphinxstyleliteralemphasis{\sphinxupquote{int}}) \textendash{} Current level in PCR (log2 scale).

\item {} 
\sphinxAtStartPar
\sphinxstyleliteralstrong{\sphinxupquote{s}} (\sphinxstyleliteralemphasis{\sphinxupquote{int}}) \textendash{} Step size (2\textasciicircum{}p).

\item {} 
\sphinxAtStartPar
\sphinxstyleliteralstrong{\sphinxupquote{current\_buffer}} (\sphinxstyleliteralemphasis{\sphinxupquote{ti.field}}) \textendash{} Current coefficients at p\sphinxhyphen{}1 level.

\item {} 
\sphinxAtStartPar
\sphinxstyleliteralstrong{\sphinxupquote{next\_buffer}} (\sphinxstyleliteralemphasis{\sphinxupquote{ti.field}}) \textendash{} Next coefficients for p\sphinxhyphen{}th level.

\end{itemize}

\end{description}\end{quote}

\end{fulllineitems}

\index{copy\_states() (celeris.solver.Solver method)@\spxentry{copy\_states()}\spxextra{celeris.solver.Solver method}}

\begin{fulllineitems}
\phantomsection\label{\detokenize{celeris:celeris.solver.Solver.copy_states}}
\pysigstartsignatures
\pysiglinewithargsret
{\sphinxbfcode{\sphinxupquote{copy\_states}}}
{\sphinxparam{\DUrole{n}{src}\DUrole{p}{:}\DUrole{w}{ }\DUrole{n}{DummyTemplate}}\sphinxparamcomma \sphinxparam{\DUrole{n}{dst}\DUrole{p}{:}\DUrole{w}{ }\DUrole{n}{DummyTemplate}}}
{}
\pysigstopsignatures
\sphinxAtStartPar
Copies state data from one Taichi field to another, ensuring shapes match.
\begin{quote}\begin{description}
\sphinxlineitem{Parameters}\begin{itemize}
\item {} 
\sphinxAtStartPar
\sphinxstyleliteralstrong{\sphinxupquote{src}} (\sphinxstyleliteralemphasis{\sphinxupquote{ti.field}}) \textendash{} Source Taichi field.

\item {} 
\sphinxAtStartPar
\sphinxstyleliteralstrong{\sphinxupquote{dst}} (\sphinxstyleliteralemphasis{\sphinxupquote{ti.field}}) \textendash{} Destination Taichi field.

\end{itemize}

\sphinxlineitem{Raises}
\sphinxAtStartPar
\sphinxstyleliteralstrong{\sphinxupquote{AssertionError}} \textendash{} If shapes of src and dst do not match.

\end{description}\end{quote}

\end{fulllineitems}

\index{fill\_bottom\_field() (celeris.solver.Solver method)@\spxentry{fill\_bottom\_field()}\spxextra{celeris.solver.Solver method}}

\begin{fulllineitems}
\phantomsection\label{\detokenize{celeris:celeris.solver.Solver.fill_bottom_field}}
\pysigstartsignatures
\pysiglinewithargsret
{\sphinxbfcode{\sphinxupquote{fill\_bottom\_field}}}
{}
{}
\pysigstopsignatures
\sphinxAtStartPar
Fills the bottom vector field array with computed spatial derivatives (indices 0,1)
and near\sphinxhyphen{}dry/auxiliary flags at index 3.

\end{fulllineitems}

\index{tridiag\_coeffs\_X() (celeris.solver.Solver method)@\spxentry{tridiag\_coeffs\_X()}\spxextra{celeris.solver.Solver method}}

\begin{fulllineitems}
\phantomsection\label{\detokenize{celeris:celeris.solver.Solver.tridiag_coeffs_X}}
\pysigstartsignatures
\pysiglinewithargsret
{\sphinxbfcode{\sphinxupquote{tridiag\_coeffs\_X}}}
{}
{}
\pysigstopsignatures
\sphinxAtStartPar
Fills tridiagonal coefficient arrays in x\sphinxhyphen{}direction (coefMatx) for the
Boussinesq model. These coefficients are used later in the parallel cyclic
reduction solver (PCR) to handle dispersion terms.

\end{fulllineitems}

\index{tridiag\_coeffs\_Y() (celeris.solver.Solver method)@\spxentry{tridiag\_coeffs\_Y()}\spxextra{celeris.solver.Solver method}}

\begin{fulllineitems}
\phantomsection\label{\detokenize{celeris:celeris.solver.Solver.tridiag_coeffs_Y}}
\pysigstartsignatures
\pysiglinewithargsret
{\sphinxbfcode{\sphinxupquote{tridiag\_coeffs\_Y}}}
{}
{}
\pysigstopsignatures
\sphinxAtStartPar
Fills tridiagonal coefficient arrays in y\sphinxhyphen{}direction (coefMaty) for the
Boussinesq model. These coefficients are used later in the parallel cyclic
reduction solver (PCR) to handle dispersion terms.

\end{fulllineitems}


\end{fulllineitems}

\index{SedClass (class in celeris.solver)@\spxentry{SedClass}\spxextra{class in celeris.solver}}

\begin{fulllineitems}
\phantomsection\label{\detokenize{celeris:celeris.solver.SedClass}}
\pysigstartsignatures
\pysiglinewithargsret
{\sphinxbfcode{\sphinxupquote{class\DUrole{w}{ }}}\sphinxcode{\sphinxupquote{celeris.solver.}}\sphinxbfcode{\sphinxupquote{SedClass}}}
{\sphinxparam{\DUrole{n}{d50}\DUrole{o}{=}\DUrole{default_value}{0.004}}\sphinxparamcomma \sphinxparam{\DUrole{n}{p}\DUrole{o}{=}\DUrole{default_value}{0.4}}\sphinxparamcomma \sphinxparam{\DUrole{n}{psi}\DUrole{o}{=}\DUrole{default_value}{0.0005}}\sphinxparamcomma \sphinxparam{\DUrole{n}{CriticalShields}\DUrole{o}{=}\DUrole{default_value}{0.045}}\sphinxparamcomma \sphinxparam{\DUrole{n}{rhorat}\DUrole{o}{=}\DUrole{default_value}{2.65}}}
{}
\pysigstopsignatures
\end{fulllineitems}



\subsubsection{Runner module}
\label{\detokenize{celeris:module-celeris.runner}}\label{\detokenize{celeris:runner-module}}\index{module@\spxentry{module}!celeris.runner@\spxentry{celeris.runner}}\index{celeris.runner@\spxentry{celeris.runner}!module@\spxentry{module}}\index{Evolve (class in celeris.runner)@\spxentry{Evolve}\spxextra{class in celeris.runner}}

\begin{fulllineitems}
\phantomsection\label{\detokenize{celeris:celeris.runner.Evolve}}
\pysigstartsignatures
\pysiglinewithargsret
{\sphinxbfcode{\sphinxupquote{class\DUrole{w}{ }}}\sphinxcode{\sphinxupquote{celeris.runner.}}\sphinxbfcode{\sphinxupquote{Evolve}}}
{\sphinxparam{\DUrole{n}{domain}\DUrole{o}{=}\DUrole{default_value}{None}}\sphinxparamcomma \sphinxparam{\DUrole{n}{boundary\_conditions}\DUrole{o}{=}\DUrole{default_value}{None}}\sphinxparamcomma \sphinxparam{\DUrole{n}{solver}\DUrole{o}{=}\DUrole{default_value}{None}}\sphinxparamcomma \sphinxparam{\DUrole{n}{maxsteps}\DUrole{o}{=}\DUrole{default_value}{1000}}\sphinxparamcomma \sphinxparam{\DUrole{n}{outdir}\DUrole{o}{=}\DUrole{default_value}{None}}\sphinxparamcomma \sphinxparam{\DUrole{n}{saveimg}\DUrole{o}{=}\DUrole{default_value}{False}}\sphinxparamcomma \sphinxparam{\DUrole{n}{vmin}\DUrole{o}{=}\DUrole{default_value}{\sphinxhyphen{}1.5}}\sphinxparamcomma \sphinxparam{\DUrole{n}{vmax}\DUrole{o}{=}\DUrole{default_value}{1.5}}}
{}
\pysigstopsignatures
\sphinxAtStartPar
Controls and runs the main simulation loop of CelerisAi in various modes (headless,
1D, 2D with visualization, etc.).

\sphinxAtStartPar
This class ties together the \sphinxtitleref{Domain}, \sphinxtitleref{BoundaryConditions}, and \sphinxtitleref{Solver} classes, and
manages the time\sphinxhyphen{}stepping workflow. It includes methods for:
\begin{enumerate}
\sphinxsetlistlabels{\arabic}{enumi}{enumii}{}{.}%
\item {} 
\sphinxAtStartPar
\sphinxstylestrong{Initialization} (\sphinxtitleref{Evolve\_0}):
\begin{itemize}
\item {} 
\sphinxAtStartPar
Fills the bottom field with bathymetry/topography data.

\item {} 
\sphinxAtStartPar
Initializes solver states (water height, velocity, etc.).

\item {} 
\sphinxAtStartPar
Computes tridiagonal coefficients if using a Boussinesq model.

\end{itemize}

\item {} 
\sphinxAtStartPar
\sphinxstylestrong{Main Time\sphinxhyphen{}Stepping} (\sphinxtitleref{Evolve\_Steps}):
\begin{itemize}
\item {} 
\sphinxAtStartPar
Runs reconstruction (Pass1) and flux computations (Pass2).

\item {} 
\sphinxAtStartPar
Handles wave breaking if enabled.

\item {} 
\sphinxAtStartPar
Integrates the solution one or more steps forward in time (Pass3, Pass3Bous, Pass3\_SedTrans).

\item {} 
\sphinxAtStartPar
Updates boundary conditions.

\item {} 
\sphinxAtStartPar
Optionally solves tridiagonal systems for Boussinesq dispersion.

\item {} 
\sphinxAtStartPar
Copies or shifts data between fields for multi\sphinxhyphen{}stage time integrators.

\end{itemize}

\item {} 
\sphinxAtStartPar
\sphinxstylestrong{Headless Execution} (\sphinxtitleref{Evolve\_Headless}):
\begin{itemize}
\item {} 
\sphinxAtStartPar
Executes the simulation loop without rendering or displaying results,
minimizing overhead and focusing on performance.

\item {} 
\sphinxAtStartPar
Periodically logs timing information and can save the simulation states (e.g. \sphinxtitleref{State} arrays).

\end{itemize}

\item {} 
\sphinxAtStartPar
\sphinxstylestrong{1D Visualization} (\sphinxtitleref{Evolve\_1D\_Display}):
\begin{itemize}
\item {} 
\sphinxAtStartPar
Specialized loop for 1D simulations, displaying free surface (eta)
and bathymetry in a window using either taichi\sphinxhyphen{}gui or legacy GUI fallback.

\end{itemize}

\item {} 
\sphinxAtStartPar
\sphinxstylestrong{2D Visualization} (\sphinxtitleref{Evolve\_Display}):
\begin{itemize}
\item {} 
\sphinxAtStartPar
Interactive loop for 2D simulations.

\item {} 
\sphinxAtStartPar
Renders wave height (h), free surface elevation (eta), or vorticity (vor)
in real\sphinxhyphen{}time.

\item {} 
\sphinxAtStartPar
Allows saving images and assembling them into a GIF.

\end{itemize}

\item {} 
\sphinxAtStartPar
\sphinxstylestrong{Rendering and Color Mapping} (Kernels like \sphinxtitleref{paint}, \sphinxtitleref{paint\_new}, \sphinxtitleref{painting\_h}, \sphinxtitleref{painting\_eta}, \sphinxtitleref{painting\_vor}, etc.):
\begin{itemize}
\item {} 
\sphinxAtStartPar
Populates 2D Taichi fields (\sphinxtitleref{self.image}, \sphinxtitleref{self.solver.pixel}, etc.) based on
solver results, for real\sphinxhyphen{}time visualization.

\item {} 
\sphinxAtStartPar
Supports multiple coloring strategies (e.g. realistic wave colors, topography shading, sediment rendering).

\end{itemize}

\end{enumerate}
\begin{quote}\begin{description}
\sphinxlineitem{Parameters}\begin{itemize}
\item {} 
\sphinxAtStartPar
\sphinxstyleliteralstrong{\sphinxupquote{domain}} ({\hyperref[\detokenize{celeris:celeris.domain.Domain}]{\sphinxcrossref{\sphinxstyleliteralemphasis{\sphinxupquote{Domain}}}}}) \textendash{} The domain class containing spatial parameters.

\item {} 
\sphinxAtStartPar
\sphinxstyleliteralstrong{\sphinxupquote{boundary\_conditions}} ({\hyperref[\detokenize{celeris:celeris.domain.BoundaryConditions}]{\sphinxcrossref{\sphinxstyleliteralemphasis{\sphinxupquote{BoundaryConditions}}}}}) \textendash{} Class managing boundary setup (walls, waves, etc.).

\item {} 
\sphinxAtStartPar
\sphinxstyleliteralstrong{\sphinxupquote{solver}} ({\hyperref[\detokenize{celeris:celeris.solver.Solver}]{\sphinxcrossref{\sphinxstyleliteralemphasis{\sphinxupquote{Solver}}}}}) \textendash{} The main numerical solver class controlling the fluid model, time scheme, etc.

\item {} 
\sphinxAtStartPar
\sphinxstyleliteralstrong{\sphinxupquote{maxsteps}} (\sphinxstyleliteralemphasis{\sphinxupquote{int}}\sphinxstyleliteralemphasis{\sphinxupquote{, }}\sphinxstyleliteralemphasis{\sphinxupquote{optional}}) \textendash{} Maximum number of time steps to simulate. Defaults to 1000.

\item {} 
\sphinxAtStartPar
\sphinxstyleliteralstrong{\sphinxupquote{outdir}} (\sphinxstyleliteralemphasis{\sphinxupquote{str}}\sphinxstyleliteralemphasis{\sphinxupquote{, }}\sphinxstyleliteralemphasis{\sphinxupquote{optional}}) \textendash{} Output directory path for saving states, frames, etc. Defaults to None.

\item {} 
\sphinxAtStartPar
\sphinxstyleliteralstrong{\sphinxupquote{saveimg}} (\sphinxstyleliteralemphasis{\sphinxupquote{bool}}\sphinxstyleliteralemphasis{\sphinxupquote{, }}\sphinxstyleliteralemphasis{\sphinxupquote{optional}}) \textendash{} If True, saves image frames at intervals (for creating GIFs or offline processing).
Defaults to False.

\item {} 
\sphinxAtStartPar
\sphinxstyleliteralstrong{\sphinxupquote{vmin}} (\sphinxstyleliteralemphasis{\sphinxupquote{float}}\sphinxstyleliteralemphasis{\sphinxupquote{, }}\sphinxstyleliteralemphasis{\sphinxupquote{optional}}) \textendash{} Minimum value for visualization color scaling (e.g. wave elevation). Defaults to \sphinxhyphen{}1.5.

\item {} 
\sphinxAtStartPar
\sphinxstyleliteralstrong{\sphinxupquote{vmax}} (\sphinxstyleliteralemphasis{\sphinxupquote{float}}\sphinxstyleliteralemphasis{\sphinxupquote{, }}\sphinxstyleliteralemphasis{\sphinxupquote{optional}}) \textendash{} Maximum value for visualization color scaling. Defaults to 1.5.

\end{itemize}

\end{description}\end{quote}
\index{solver (celeris.runner.Evolve attribute)@\spxentry{solver}\spxextra{celeris.runner.Evolve attribute}}

\begin{fulllineitems}
\phantomsection\label{\detokenize{celeris:celeris.runner.Evolve.solver}}
\pysigstartsignatures
\pysigline
{\sphinxbfcode{\sphinxupquote{solver}}}
\pysigstopsignatures
\sphinxAtStartPar
The numerical solver controlling fluid/morphodynamics.
\begin{quote}\begin{description}
\sphinxlineitem{Type}
\sphinxAtStartPar
{\hyperref[\detokenize{celeris:celeris.solver.Solver}]{\sphinxcrossref{Solver}}}

\end{description}\end{quote}

\end{fulllineitems}

\index{maxsteps (celeris.runner.Evolve attribute)@\spxentry{maxsteps}\spxextra{celeris.runner.Evolve attribute}}

\begin{fulllineitems}
\phantomsection\label{\detokenize{celeris:celeris.runner.Evolve.maxsteps}}
\pysigstartsignatures
\pysigline
{\sphinxbfcode{\sphinxupquote{maxsteps}}}
\pysigstopsignatures
\sphinxAtStartPar
Number of time steps for the simulation run.
\begin{quote}\begin{description}
\sphinxlineitem{Type}
\sphinxAtStartPar
int

\end{description}\end{quote}

\end{fulllineitems}

\index{dt (celeris.runner.Evolve attribute)@\spxentry{dt}\spxextra{celeris.runner.Evolve attribute}}

\begin{fulllineitems}
\phantomsection\label{\detokenize{celeris:celeris.runner.Evolve.dt}}
\pysigstartsignatures
\pysigline
{\sphinxbfcode{\sphinxupquote{dt}}}
\pysigstopsignatures
\sphinxAtStartPar
Time step size imported from the solver.
\begin{quote}\begin{description}
\sphinxlineitem{Type}
\sphinxAtStartPar
float

\end{description}\end{quote}

\end{fulllineitems}

\index{timeScheme (celeris.runner.Evolve attribute)@\spxentry{timeScheme}\spxextra{celeris.runner.Evolve attribute}}

\begin{fulllineitems}
\phantomsection\label{\detokenize{celeris:celeris.runner.Evolve.timeScheme}}
\pysigstartsignatures
\pysigline
{\sphinxbfcode{\sphinxupquote{timeScheme}}}
\pysigstopsignatures
\sphinxAtStartPar
Time integration scheme (Euler, predictor, predictor\sphinxhyphen{}corrector).
\begin{quote}\begin{description}
\sphinxlineitem{Type}
\sphinxAtStartPar
int

\end{description}\end{quote}

\end{fulllineitems}

\index{saveimg (celeris.runner.Evolve attribute)@\spxentry{saveimg}\spxextra{celeris.runner.Evolve attribute}}

\begin{fulllineitems}
\phantomsection\label{\detokenize{celeris:celeris.runner.Evolve.saveimg}}
\pysigstartsignatures
\pysigline
{\sphinxbfcode{\sphinxupquote{saveimg}}}
\pysigstopsignatures
\sphinxAtStartPar
Flag indicating whether to save frames.
\begin{quote}\begin{description}
\sphinxlineitem{Type}
\sphinxAtStartPar
bool

\end{description}\end{quote}

\end{fulllineitems}

\index{vmin (celeris.runner.Evolve attribute)@\spxentry{vmin}\spxextra{celeris.runner.Evolve attribute}}

\begin{fulllineitems}
\phantomsection\label{\detokenize{celeris:celeris.runner.Evolve.vmin}}
\pysigstartsignatures
\pysigline
{\sphinxbfcode{\sphinxupquote{vmin}}}
\pysigstopsignatures
\sphinxAtStartPar
Minimum scale for color mapping wave or vorticity values.
\begin{quote}\begin{description}
\sphinxlineitem{Type}
\sphinxAtStartPar
float

\end{description}\end{quote}

\end{fulllineitems}

\index{vmax (celeris.runner.Evolve attribute)@\spxentry{vmax}\spxextra{celeris.runner.Evolve attribute}}

\begin{fulllineitems}
\phantomsection\label{\detokenize{celeris:celeris.runner.Evolve.vmax}}
\pysigstartsignatures
\pysigline
{\sphinxbfcode{\sphinxupquote{vmax}}}
\pysigstopsignatures
\sphinxAtStartPar
Maximum scale for color mapping wave or vorticity values.
\begin{quote}\begin{description}
\sphinxlineitem{Type}
\sphinxAtStartPar
float

\end{description}\end{quote}

\end{fulllineitems}

\index{outdir (celeris.runner.Evolve attribute)@\spxentry{outdir}\spxextra{celeris.runner.Evolve attribute}}

\begin{fulllineitems}
\phantomsection\label{\detokenize{celeris:celeris.runner.Evolve.outdir}}
\pysigstartsignatures
\pysigline
{\sphinxbfcode{\sphinxupquote{outdir}}}
\pysigstopsignatures
\sphinxAtStartPar
Directory for saving outputs.
\begin{quote}\begin{description}
\sphinxlineitem{Type}
\sphinxAtStartPar
str

\end{description}\end{quote}

\end{fulllineitems}

\index{image (celeris.runner.Evolve attribute)@\spxentry{image}\spxextra{celeris.runner.Evolve attribute}}

\begin{fulllineitems}
\phantomsection\label{\detokenize{celeris:celeris.runner.Evolve.image}}
\pysigstartsignatures
\pysigline
{\sphinxbfcode{\sphinxupquote{image}}}
\pysigstopsignatures
\sphinxAtStartPar
2D field to hold RGB color information for visualization.
\begin{quote}\begin{description}
\sphinxlineitem{Type}
\sphinxAtStartPar
ti.Vector.field

\end{description}\end{quote}

\end{fulllineitems}

\index{ocean (celeris.runner.Evolve attribute)@\spxentry{ocean}\spxextra{celeris.runner.Evolve attribute}}

\begin{fulllineitems}
\phantomsection\label{\detokenize{celeris:celeris.runner.Evolve.ocean}}
\pysigstartsignatures
\pysigline
{\sphinxbfcode{\sphinxupquote{ocean}}}
\pysigstopsignatures
\sphinxAtStartPar
1D array of color samples (RGB) for water visualization or general colormap usage.
\begin{quote}\begin{description}
\sphinxlineitem{Type}
\sphinxAtStartPar
ti.Vector.field

\end{description}\end{quote}

\end{fulllineitems}

\index{colormap\_ocean (celeris.runner.Evolve attribute)@\spxentry{colormap\_ocean}\spxextra{celeris.runner.Evolve attribute}}

\begin{fulllineitems}
\phantomsection\label{\detokenize{celeris:celeris.runner.Evolve.colormap_ocean}}
\pysigstartsignatures
\pysigline
{\sphinxbfcode{\sphinxupquote{colormap\_ocean}}}
\pysigstopsignatures
\sphinxAtStartPar
A string identifier for the colormap used for water.
\begin{quote}\begin{description}
\sphinxlineitem{Type}
\sphinxAtStartPar
str

\end{description}\end{quote}

\end{fulllineitems}



\begin{fulllineitems}

\pysigstartsignatures
\pysigline
{\sphinxbfcode{\sphinxupquote{bottom1D,~indexbottom1D,~eta1D}}}
\pysigstopsignatures
\sphinxAtStartPar
Fields used in 1D visualization of bottom topography and water surface.

\end{fulllineitems}



\begin{fulllineitems}

\pysigstartsignatures
\pysigline
{\sphinxbfcode{\sphinxupquote{x\_scale,~y\_scale}}}
\pysigstopsignatures
\sphinxAtStartPar
Scaling factors for 1D plots in the GUI.
\begin{quote}\begin{description}
\sphinxlineitem{Type}
\sphinxAtStartPar
float

\end{description}\end{quote}

\end{fulllineitems}


\sphinxAtStartPar
Typical Usage:

\begin{sphinxVerbatim}[commandchars=\\\{\}]
\PYG{g+gp}{\PYGZgt{}\PYGZgt{}\PYGZgt{} }\PYG{n}{evolve} \PYG{o}{=} \PYG{n}{Evolve}\PYG{p}{(}\PYG{n}{domain}\PYG{o}{=}\PYG{n}{dom}\PYG{p}{,} \PYG{n}{boundary\PYGZus{}conditions}\PYG{o}{=}\PYG{n}{bc}\PYG{p}{,} \PYG{n}{solver}\PYG{o}{=}\PYG{n}{sol}\PYG{p}{,} \PYG{n}{maxsteps}\PYG{o}{=}\PYG{l+m+mi}{2000}\PYG{p}{,} \PYG{n}{outdir}\PYG{o}{=}\PYG{l+s+s2}{\PYGZdq{}}\PYG{l+s+s2}{results}\PYG{l+s+s2}{\PYGZdq{}}\PYG{p}{)}
\PYG{g+gp}{\PYGZgt{}\PYGZgt{}\PYGZgt{} }\PYG{n}{evolve}\PYG{o}{.}\PYG{n}{Evolve\PYGZus{}Display}\PYG{p}{(}\PYG{n}{vmin}\PYG{o}{=}\PYG{o}{\PYGZhy{}}\PYG{l+m+mf}{1.0}\PYG{p}{,} \PYG{n}{vmax}\PYG{o}{=}\PYG{l+m+mf}{1.0}\PYG{p}{,} \PYG{n}{variable}\PYG{o}{=}\PYG{l+s+s1}{\PYGZsq{}}\PYG{l+s+s1}{eta}\PYG{l+s+s1}{\PYGZsq{}}\PYG{p}{,} \PYG{n}{cmapWater}\PYG{o}{=}\PYG{l+s+s1}{\PYGZsq{}}\PYG{l+s+s1}{Blues\PYGZus{}r}\PYG{l+s+s1}{\PYGZsq{}}\PYG{p}{,} \PYG{n}{showSediment}\PYG{o}{=}\PYG{k+kc}{True}\PYG{p}{)}
\end{sphinxVerbatim}

\begin{sphinxadmonition}{note}{Note:}\begin{itemize}
\item {} 
\sphinxAtStartPar
The class attempts to use Taichi’s GGUI if available for improved performance
and better UI control. If GGUI is not available, it falls back to legacy Taichi GUI.

\item {} 
\sphinxAtStartPar
Various coloring kernels (\sphinxtitleref{painting\_h}, \sphinxtitleref{painting\_eta}, etc.) can be customized
to match user\sphinxhyphen{}defined styles or to highlight specific flow features.

\end{itemize}
\end{sphinxadmonition}
\index{Evolve\_0() (celeris.runner.Evolve method)@\spxentry{Evolve\_0()}\spxextra{celeris.runner.Evolve method}}

\begin{fulllineitems}
\phantomsection\label{\detokenize{celeris:celeris.runner.Evolve.Evolve_0}}
\pysigstartsignatures
\pysiglinewithargsret
{\sphinxbfcode{\sphinxupquote{Evolve\_0}}}
{}
{}
\pysigstopsignatures\begin{description}
\sphinxlineitem{One\sphinxhyphen{}time initialization steps:}\begin{itemize}
\item {} 
\sphinxAtStartPar
Fills bottom field (bathymetry/topo).

\item {} 
\sphinxAtStartPar
Initializes solver states (fluid variables).

\item {} 
\sphinxAtStartPar
Computes tridiagonal coefficients if the model is Boussinesq.

\item {} 
\sphinxAtStartPar
Prints simulation parameters (model type, time step, etc.).

\end{itemize}

\end{description}

\end{fulllineitems}

\index{Evolve\_1D\_Display() (celeris.runner.Evolve method)@\spxentry{Evolve\_1D\_Display()}\spxextra{celeris.runner.Evolve method}}

\begin{fulllineitems}
\phantomsection\label{\detokenize{celeris:celeris.runner.Evolve.Evolve_1D_Display}}
\pysigstartsignatures
\pysiglinewithargsret
{\sphinxbfcode{\sphinxupquote{Evolve\_1D\_Display}}}
{}
{}
\pysigstopsignatures
\sphinxAtStartPar
Interactive loop for a 1D simulation of CelerisAi.
\begin{itemize}
\item {} 
\sphinxAtStartPar
Initializes bottom, runs the main solver steps, and displays the results
in a small taichi\sphinxhyphen{}gui or GGUI window.

\item {} 
\sphinxAtStartPar
Plots the free surface (eta) and bottom profile in each iteration.

\item {} 
\sphinxAtStartPar
Optionally saves frames and can compile them into a GIF if desired.

\end{itemize}

\end{fulllineitems}

\index{Evolve\_Display() (celeris.runner.Evolve method)@\spxentry{Evolve\_Display()}\spxextra{celeris.runner.Evolve method}}

\begin{fulllineitems}
\phantomsection\label{\detokenize{celeris:celeris.runner.Evolve.Evolve_Display}}
\pysigstartsignatures
\pysiglinewithargsret
{\sphinxbfcode{\sphinxupquote{Evolve\_Display}}}
{\sphinxparam{\DUrole{n}{vmin}\DUrole{o}{=}\DUrole{default_value}{None}}\sphinxparamcomma \sphinxparam{\DUrole{n}{vmax}\DUrole{o}{=}\DUrole{default_value}{None}}\sphinxparamcomma \sphinxparam{\DUrole{n}{variable}\DUrole{o}{=}\DUrole{default_value}{\textquotesingle{}h\textquotesingle{}}}\sphinxparamcomma \sphinxparam{\DUrole{n}{cmapWater}\DUrole{o}{=}\DUrole{default_value}{\textquotesingle{}Blues\_r\textquotesingle{}}}\sphinxparamcomma \sphinxparam{\DUrole{n}{showSediment}\DUrole{o}{=}\DUrole{default_value}{False}}}
{}
\pysigstopsignatures
\sphinxAtStartPar
Interactive loop for a 2D simulation of CelerisAi with real\sphinxhyphen{}time visualization.
\begin{itemize}
\item {} 
\sphinxAtStartPar
Calls \sphinxtitleref{Evolve\_0()} once to initialize solver fields.

\item {} 
\sphinxAtStartPar
Creates a window (either GGUI or legacy GUI).

\item {} 
\sphinxAtStartPar
Uses a custom colormap from \sphinxtitleref{celeris\_matplotlib()} if \sphinxtitleref{showSediment} is True
and the solver has sediment transport enabled.

\item {} \begin{description}
\sphinxlineitem{Allows switching between different visualization variables:}\begin{itemize}
\item {} 
\sphinxAtStartPar
\sphinxtitleref{h}: Water depth

\item {} 
\sphinxAtStartPar
\sphinxtitleref{eta}: Free surface elevation

\item {} 
\sphinxAtStartPar
\sphinxtitleref{vor}: Vorticity

\end{itemize}

\end{description}

\item {} 
\sphinxAtStartPar
Saves frames if \sphinxtitleref{saveimg} is True, can compile them into a GIF, and optionally
saves solver states to \sphinxtitleref{.npy}.

\end{itemize}
\begin{quote}\begin{description}
\sphinxlineitem{Parameters}\begin{itemize}
\item {} 
\sphinxAtStartPar
\sphinxstyleliteralstrong{\sphinxupquote{vmin}} (\sphinxstyleliteralemphasis{\sphinxupquote{float}}\sphinxstyleliteralemphasis{\sphinxupquote{, }}\sphinxstyleliteralemphasis{\sphinxupquote{optional}}) \textendash{} Minimum colormap value for rendering. Defaults to None (class\sphinxhyphen{}level vmin).

\item {} 
\sphinxAtStartPar
\sphinxstyleliteralstrong{\sphinxupquote{vmax}} (\sphinxstyleliteralemphasis{\sphinxupquote{float}}\sphinxstyleliteralemphasis{\sphinxupquote{, }}\sphinxstyleliteralemphasis{\sphinxupquote{optional}}) \textendash{} Maximum colormap value for rendering. Defaults to None (class\sphinxhyphen{}level vmax).

\item {} 
\sphinxAtStartPar
\sphinxstyleliteralstrong{\sphinxupquote{variable}} (\sphinxstyleliteralemphasis{\sphinxupquote{str}}\sphinxstyleliteralemphasis{\sphinxupquote{, }}\sphinxstyleliteralemphasis{\sphinxupquote{optional}}) \textendash{} Which variable to render (\sphinxtitleref{h}, \sphinxtitleref{eta}, or \sphinxtitleref{vor}). Defaults to ‘h’.

\item {} 
\sphinxAtStartPar
\sphinxstyleliteralstrong{\sphinxupquote{cmapWater}} (\sphinxstyleliteralemphasis{\sphinxupquote{str}}\sphinxstyleliteralemphasis{\sphinxupquote{, }}\sphinxstyleliteralemphasis{\sphinxupquote{optional}}) \textendash{} Matplotlib colormap name for water. Defaults to ‘Blues\_r’.

\item {} 
\sphinxAtStartPar
\sphinxstyleliteralstrong{\sphinxupquote{showSediment}} (\sphinxstyleliteralemphasis{\sphinxupquote{bool}}\sphinxstyleliteralemphasis{\sphinxupquote{, }}\sphinxstyleliteralemphasis{\sphinxupquote{optional}}) \textendash{} If True, merges in a sediment colormap when sediment transport is active.
Defaults to False.

\end{itemize}

\end{description}\end{quote}

\end{fulllineitems}

\index{Evolve\_Headless() (celeris.runner.Evolve method)@\spxentry{Evolve\_Headless()}\spxextra{celeris.runner.Evolve method}}

\begin{fulllineitems}
\phantomsection\label{\detokenize{celeris:celeris.runner.Evolve.Evolve_Headless}}
\pysigstartsignatures
\pysiglinewithargsret
{\sphinxbfcode{\sphinxupquote{Evolve\_Headless}}}
{}
{}
\pysigstopsignatures
\sphinxAtStartPar
Runs CelerisAi without any visualization, printing timing info periodically
and optionally saving states to disk.
\begin{description}
\sphinxlineitem{Steps:}\begin{enumerate}
\sphinxsetlistlabels{\arabic}{enumi}{enumii}{}{.}%
\item {} 
\sphinxAtStartPar
Calls Evolve\_0() to initialize fields and solver state.

\item {} 
\sphinxAtStartPar
Loops over \sphinxtitleref{maxsteps}, calling Evolve\_Steps() each iteration.

\item {} 
\sphinxAtStartPar
Logs simulation time and performance metrics every 100 steps.

\item {} 
\sphinxAtStartPar
If an output directory is specified, saves solver state arrays to .npy files.

\end{enumerate}

\end{description}

\end{fulllineitems}

\index{Evolve\_Steps() (celeris.runner.Evolve method)@\spxentry{Evolve\_Steps()}\spxextra{celeris.runner.Evolve method}}

\begin{fulllineitems}
\phantomsection\label{\detokenize{celeris:celeris.runner.Evolve.Evolve_Steps}}
\pysigstartsignatures
\pysiglinewithargsret
{\sphinxbfcode{\sphinxupquote{Evolve\_Steps}}}
{\sphinxparam{\DUrole{n}{step}\DUrole{o}{=}\DUrole{default_value}{0}}}
{}
\pysigstopsignatures
\sphinxAtStartPar
Advances the solution by one time step (or one sub\sphinxhyphen{}step) according to the selected time scheme.
\begin{description}
\sphinxlineitem{Internally calls:}\begin{itemize}
\item {} 
\sphinxAtStartPar
Pass1 (and Pass1\_SedTrans if sediment is enabled)

\item {} 
\sphinxAtStartPar
Pass2

\item {} 
\sphinxAtStartPar
Optional wave breaking model

\item {} 
\sphinxAtStartPar
Pass3 or Pass3Bous for predictor step

\item {} 
\sphinxAtStartPar
BoundaryPass to enforce boundary conditions

\item {} 
\sphinxAtStartPar
Run\_Tridiag\_solver for Boussinesq models

\item {} 
\sphinxAtStartPar
(If 4th\sphinxhyphen{}order  predictor\sphinxhyphen{}corrector) a second cycle of Pass1, Pass2, Pass3 or Pass3Bous

\item {} 
\sphinxAtStartPar
Copies or shifts old/predicted states for multi\sphinxhyphen{}stage time integrators

\end{itemize}

\end{description}

\end{fulllineitems}

\index{InitColors() (celeris.runner.Evolve method)@\spxentry{InitColors()}\spxextra{celeris.runner.Evolve method}}

\begin{fulllineitems}
\phantomsection\label{\detokenize{celeris:celeris.runner.Evolve.InitColors}}
\pysigstartsignatures
\pysiglinewithargsret
{\sphinxbfcode{\sphinxupquote{InitColors}}}
{\sphinxparam{\DUrole{n}{arr}\DUrole{p}{:}\DUrole{w}{ }\DUrole{n}{ndarray\DUrole{p}{(}\DUrole{n}{dtype}\DUrole{o}{=}ti.f32\DUrole{p}{,}\DUrole{w}{ }\DUrole{n}{ndim}\DUrole{o}{=}\DUrole{m}{2}\DUrole{p}{)}}}}
{}
\pysigstopsignatures
\sphinxAtStartPar
Copies an external NumPy array of shape (N, 3) (RGB colors) into the
internal taichi field \sphinxtitleref{self.ocean}.
\begin{quote}\begin{description}
\sphinxlineitem{Parameters}
\sphinxAtStartPar
\sphinxstyleliteralstrong{\sphinxupquote{arr}} (\sphinxstyleliteralemphasis{\sphinxupquote{np.ndarray}}) \textendash{} (N, 3) array of float16 color data (e.g., from a Matplotlib colormap).

\end{description}\end{quote}

\end{fulllineitems}

\index{bottom\_paint() (celeris.runner.Evolve method)@\spxentry{bottom\_paint()}\spxextra{celeris.runner.Evolve method}}

\begin{fulllineitems}
\phantomsection\label{\detokenize{celeris:celeris.runner.Evolve.bottom_paint}}
\pysigstartsignatures
\pysiglinewithargsret
{\sphinxbfcode{\sphinxupquote{bottom\_paint}}}
{}
{}
\pysigstopsignatures
\sphinxAtStartPar
Fills \sphinxtitleref{bottom1D} with scaled bottom data for 1D visualization.
Also populates \sphinxtitleref{indexbottom1D} so that line plotting can connect them in order.

\end{fulllineitems}

\index{brk\_color() (celeris.runner.Evolve method)@\spxentry{brk\_color()}\spxextra{celeris.runner.Evolve method}}

\begin{fulllineitems}
\phantomsection\label{\detokenize{celeris:celeris.runner.Evolve.brk_color}}
\pysigstartsignatures
\pysiglinewithargsret
{\sphinxbfcode{\sphinxupquote{brk\_color}}}
{\sphinxparam{\DUrole{n}{x}}\sphinxparamcomma \sphinxparam{\DUrole{n}{y0}}\sphinxparamcomma \sphinxparam{\DUrole{n}{y1}}\sphinxparamcomma \sphinxparam{\DUrole{n}{x0}}\sphinxparamcomma \sphinxparam{\DUrole{n}{x1}}}
{}
\pysigstopsignatures
\sphinxAtStartPar
Interpolates between two values (y0, y1) based on x in {[}x0, x1{]}.
\begin{description}
\sphinxlineitem{Used to smoothly vary color or other scalar values between two extremes:}
\sphinxAtStartPar
(x0 \sphinxhyphen{}\textgreater{} y0) to (x1 \sphinxhyphen{}\textgreater{} y1).

\end{description}

\end{fulllineitems}

\index{eta\_paint() (celeris.runner.Evolve method)@\spxentry{eta\_paint()}\spxextra{celeris.runner.Evolve method}}

\begin{fulllineitems}
\phantomsection\label{\detokenize{celeris:celeris.runner.Evolve.eta_paint}}
\pysigstartsignatures
\pysiglinewithargsret
{\sphinxbfcode{\sphinxupquote{eta\_paint}}}
{}
{}
\pysigstopsignatures
\sphinxAtStartPar
Fills \sphinxtitleref{eta1D} with scaled free\sphinxhyphen{}surface data for 1D visualization.

\end{fulllineitems}

\index{paint() (celeris.runner.Evolve method)@\spxentry{paint()}\spxextra{celeris.runner.Evolve method}}

\begin{fulllineitems}
\phantomsection\label{\detokenize{celeris:celeris.runner.Evolve.paint}}
\pysigstartsignatures
\pysiglinewithargsret
{\sphinxbfcode{\sphinxupquote{paint}}}
{}
{}
\pysigstopsignatures
\sphinxAtStartPar
General paint kernel that merges bottom topography and free\sphinxhyphen{}surface height.
\begin{itemize}
\item {} 
\sphinxAtStartPar
Assigns color based on bottom topography if there’s no water.

\item {} 
\sphinxAtStartPar
Interpolates water color if flow depth \textgreater{} 0.0001.

\item {} 
\sphinxAtStartPar
If sediment transport is enabled, merges sediment concentration into final color.

\end{itemize}

\end{fulllineitems}

\index{paint\_new() (celeris.runner.Evolve method)@\spxentry{paint\_new()}\spxextra{celeris.runner.Evolve method}}

\begin{fulllineitems}
\phantomsection\label{\detokenize{celeris:celeris.runner.Evolve.paint_new}}
\pysigstartsignatures
\pysiglinewithargsret
{\sphinxbfcode{\sphinxupquote{paint\_new}}}
{}
{}
\pysigstopsignatures
\sphinxAtStartPar
A simple rendering kernel mixing bottom topography and wave height.
Uses a linear interpolation (brk\_color) to assign colors to each cell
based on water depth or topography.

\end{fulllineitems}

\index{painting\_eta() (celeris.runner.Evolve method)@\spxentry{painting\_eta()}\spxextra{celeris.runner.Evolve method}}

\begin{fulllineitems}
\phantomsection\label{\detokenize{celeris:celeris.runner.Evolve.painting_eta}}
\pysigstartsignatures
\pysiglinewithargsret
{\sphinxbfcode{\sphinxupquote{painting\_eta}}}
{}
{}
\pysigstopsignatures
\sphinxAtStartPar
Kernel for visualizing free surface elevation (eta).
\begin{itemize}
\item {} 
\sphinxAtStartPar
Normalizes eta to {[}vmin, vmax{]} for color lookup.

\item {} 
\sphinxAtStartPar
Distinguishes water areas from wet sand and land similar to \sphinxtitleref{painting\_h}.

\end{itemize}

\end{fulllineitems}

\index{painting\_h() (celeris.runner.Evolve method)@\spxentry{painting\_h()}\spxextra{celeris.runner.Evolve method}}

\begin{fulllineitems}
\phantomsection\label{\detokenize{celeris:celeris.runner.Evolve.painting_h}}
\pysigstartsignatures
\pysiglinewithargsret
{\sphinxbfcode{\sphinxupquote{painting\_h}}}
{}
{}
\pysigstopsignatures
\sphinxAtStartPar
Kernel for visualizing water depth (h) in a “realistic wave” style.
\begin{itemize}
\item {} 
\sphinxAtStartPar
Normalizes water depth to {[}0, base\_depth{]}.

\item {} 
\sphinxAtStartPar
Chooses colors from the \sphinxtitleref{ocean} array based on the normalized depth (linear interpolation).

\item {} 
\sphinxAtStartPar
Make a difference between the water areas (flow \textgreater{} 0.25) from shallow/wet sand and land.

\end{itemize}

\end{fulllineitems}

\index{painting\_vor() (celeris.runner.Evolve method)@\spxentry{painting\_vor()}\spxextra{celeris.runner.Evolve method}}

\begin{fulllineitems}
\phantomsection\label{\detokenize{celeris:celeris.runner.Evolve.painting_vor}}
\pysigstartsignatures
\pysiglinewithargsret
{\sphinxbfcode{\sphinxupquote{painting\_vor}}}
{}
{}
\pysigstopsignatures
\sphinxAtStartPar
Kernel for visualizing vorticity (vor).
\begin{itemize}
\item {} 
\sphinxAtStartPar
Approximates vorticity by differences in velocity between adjacent cells.

\item {} 
\sphinxAtStartPar
Normalizes the result into {[}vmin, vmax{]} for color lookups in \sphinxtitleref{ocean}.

\item {} 
\sphinxAtStartPar
Distinguishes water vs. land similar to \sphinxtitleref{painting\_h}/\sphinxtitleref{painting\_eta}.

\end{itemize}

\end{fulllineitems}


\end{fulllineitems}



\subsubsection{Utils module}
\label{\detokenize{celeris:module-celeris.utils}}\label{\detokenize{celeris:utils-module}}\index{module@\spxentry{module}!celeris.utils@\spxentry{celeris.utils}}\index{celeris.utils@\spxentry{celeris.utils}!module@\spxentry{module}}\index{CalcUV() (in module celeris.utils)@\spxentry{CalcUV()}\spxextra{in module celeris.utils}}

\begin{fulllineitems}
\phantomsection\label{\detokenize{celeris:celeris.utils.CalcUV}}
\pysigstartsignatures
\pysiglinewithargsret
{\sphinxcode{\sphinxupquote{celeris.utils.}}\sphinxbfcode{\sphinxupquote{CalcUV}}}
{\sphinxparam{\DUrole{n}{h}}\sphinxparamcomma \sphinxparam{\DUrole{n}{hu}}\sphinxparamcomma \sphinxparam{\DUrole{n}{hv}}\sphinxparamcomma \sphinxparam{\DUrole{n}{hc}}\sphinxparamcomma \sphinxparam{\DUrole{n}{epsilon}}\sphinxparamcomma \sphinxparam{\DUrole{n}{dB\_max}}}
{}
\pysigstopsignatures
\sphinxAtStartPar
Computes velocity and scalar concentration at cell edges given water height and momentum.

\sphinxAtStartPar
This function takes the water depth (h), x\sphinxhyphen{}momentum (hu), y\sphinxhyphen{}momentum (hv), and
a scalar quantity (hc) at the cell edges (usually indexed {[}N, E, S, W{]}) and
returns the velocity components (u, v) and scalar concentration (c) at those edges.
It applies a limiting factor to avoid division by a near\sphinxhyphen{}zero depth.
\begin{description}
\sphinxlineitem{The key step involves computing:}
\sphinxAtStartPar
divide\_by\_h = 2.0 * h / (h * h + ti.max(h * h, epsilon\_c))

\end{description}

\sphinxAtStartPar
where:
\begin{itemize}
\item {} 
\sphinxAtStartPar
\sphinxtitleref{epsilon\_c = max(epsilon, dB\_max)}

\item {} 
\sphinxAtStartPar
\sphinxtitleref{epsilon} is a small threshold to prevent division by zero,

\item {} 
\sphinxAtStartPar
\sphinxtitleref{dB\_max} represents the maximum bed\sphinxhyphen{}elevation difference across edges,
ensuring the local depth used for velocity calculation is not less than
the difference in water depth across an edge.

\end{itemize}
\begin{quote}\begin{description}
\sphinxlineitem{Parameters}\begin{itemize}
\item {} 
\sphinxAtStartPar
\sphinxstyleliteralstrong{\sphinxupquote{h}} (\sphinxstyleliteralemphasis{\sphinxupquote{ti.types.vector}}) \textendash{} Water depth at edges, shaped {[}N, E, S, W{]}.

\item {} 
\sphinxAtStartPar
\sphinxstyleliteralstrong{\sphinxupquote{hu}} (\sphinxstyleliteralemphasis{\sphinxupquote{ti.types.vector}}) \textendash{} Momentum in the x\sphinxhyphen{}direction at edges.

\item {} 
\sphinxAtStartPar
\sphinxstyleliteralstrong{\sphinxupquote{hv}} (\sphinxstyleliteralemphasis{\sphinxupquote{ti.types.vector}}) \textendash{} Momentum in the y\sphinxhyphen{}direction at edges.

\item {} 
\sphinxAtStartPar
\sphinxstyleliteralstrong{\sphinxupquote{hc}} (\sphinxstyleliteralemphasis{\sphinxupquote{ti.types.vector}}) \textendash{} Scalar quantity (e.g. concentration) at edges.

\item {} 
\sphinxAtStartPar
\sphinxstyleliteralstrong{\sphinxupquote{epsilon}} (\sphinxstyleliteralemphasis{\sphinxupquote{float}}) \textendash{} Small threshold to avoid division by zero in near\sphinxhyphen{}dry cells.

\item {} 
\sphinxAtStartPar
\sphinxstyleliteralstrong{\sphinxupquote{dB\_max}} (\sphinxstyleliteralemphasis{\sphinxupquote{ti.types.vector}}\sphinxstyleliteralemphasis{\sphinxupquote{ or }}\sphinxstyleliteralemphasis{\sphinxupquote{float}}) \textendash{} Maximum bed\sphinxhyphen{}elevation difference used to limit depth.

\end{itemize}

\sphinxlineitem{Returns}
\sphinxAtStartPar
\begin{description}
\sphinxlineitem{(u, v, c) where each is a vector shaped {[}N, E, S, W{]}.}\begin{itemize}
\item {} 
\sphinxAtStartPar
\sphinxstylestrong{u}: x\sphinxhyphen{}velocity at edges.

\item {} 
\sphinxAtStartPar
\sphinxstylestrong{v}: y\sphinxhyphen{}velocity at edges.

\item {} 
\sphinxAtStartPar
\sphinxstylestrong{c}: Scalar concentration at edges.

\end{itemize}

\end{description}


\sphinxlineitem{Return type}
\sphinxAtStartPar
tuple of ti.types.vector

\end{description}\end{quote}

\end{fulllineitems}

\index{CalcUV\_Sed() (in module celeris.utils)@\spxentry{CalcUV\_Sed()}\spxextra{in module celeris.utils}}

\begin{fulllineitems}
\phantomsection\label{\detokenize{celeris:celeris.utils.CalcUV_Sed}}
\pysigstartsignatures
\pysiglinewithargsret
{\sphinxcode{\sphinxupquote{celeris.utils.}}\sphinxbfcode{\sphinxupquote{CalcUV\_Sed}}}
{\sphinxparam{\DUrole{n}{h}}\sphinxparamcomma \sphinxparam{\DUrole{n}{hc1}}\sphinxparamcomma \sphinxparam{\DUrole{n}{hc2}}\sphinxparamcomma \sphinxparam{\DUrole{n}{hc3}}\sphinxparamcomma \sphinxparam{\DUrole{n}{hc4}}\sphinxparamcomma \sphinxparam{\DUrole{n}{epsilon}}\sphinxparamcomma \sphinxparam{\DUrole{n}{dB\_max}}}
{}
\pysigstopsignatures
\sphinxAtStartPar
Computes sediment scalar concentrations at cell edges given the water height.

\sphinxAtStartPar
This function calculates four sediment\sphinxhyphen{}related quantities (e.g., scalar concentrations
or sediment fractions) at the edges of a cell. It applies a limiting factor based on
water depth (\sphinxtitleref{h}) to avoid division by a near\sphinxhyphen{}zero depth and to account for
significant bed\sphinxhyphen{}elevation differences.
\begin{quote}\begin{description}
\sphinxlineitem{Parameters}\begin{itemize}
\item {} 
\sphinxAtStartPar
\sphinxstyleliteralstrong{\sphinxupquote{h}} (\sphinxstyleliteralemphasis{\sphinxupquote{float}}) \textendash{} Water depth at the cell edge.

\item {} 
\sphinxAtStartPar
\sphinxstyleliteralstrong{\sphinxupquote{hc1}} (\sphinxstyleliteralemphasis{\sphinxupquote{float}}) \textendash{} Sediment/scalar quantity \#1 at the edge.

\item {} 
\sphinxAtStartPar
\sphinxstyleliteralstrong{\sphinxupquote{hc2}} (\sphinxstyleliteralemphasis{\sphinxupquote{float}}) \textendash{} Sediment/scalar quantity \#2 at the edge.

\item {} 
\sphinxAtStartPar
\sphinxstyleliteralstrong{\sphinxupquote{hc3}} (\sphinxstyleliteralemphasis{\sphinxupquote{float}}) \textendash{} Sediment/scalar quantity \#3 at the edge.

\item {} 
\sphinxAtStartPar
\sphinxstyleliteralstrong{\sphinxupquote{hc4}} (\sphinxstyleliteralemphasis{\sphinxupquote{float}}) \textendash{} Sediment/scalar quantity \#4 at the edge.

\item {} 
\sphinxAtStartPar
\sphinxstyleliteralstrong{\sphinxupquote{epsilon}} (\sphinxstyleliteralemphasis{\sphinxupquote{float}}) \textendash{} Small threshold to avoid division by zero.

\item {} 
\sphinxAtStartPar
\sphinxstyleliteralstrong{\sphinxupquote{dB\_max}} (\sphinxstyleliteralemphasis{\sphinxupquote{float}}) \textendash{} Maximum bed\sphinxhyphen{}elevation difference, used in limiting depth.

\end{itemize}

\sphinxlineitem{Returns}
\sphinxAtStartPar
A 4\sphinxhyphen{}component vector containing the scaled sediment/scalar values
\sphinxtitleref{{[}c1, c2, c3, c4{]}} after applying the depth\sphinxhyphen{}limiting factor.

\sphinxlineitem{Return type}
\sphinxAtStartPar
ti.Vector({[}float, float, float, float{]})

\end{description}\end{quote}

\end{fulllineitems}

\index{ColorsfromMPL() (in module celeris.utils)@\spxentry{ColorsfromMPL()}\spxextra{in module celeris.utils}}

\begin{fulllineitems}
\phantomsection\label{\detokenize{celeris:celeris.utils.ColorsfromMPL}}
\pysigstartsignatures
\pysiglinewithargsret
{\sphinxcode{\sphinxupquote{celeris.utils.}}\sphinxbfcode{\sphinxupquote{ColorsfromMPL}}}
{\sphinxparam{\DUrole{n}{cmap}\DUrole{o}{=}\DUrole{default_value}{\textquotesingle{}Blues\textquotesingle{}}}}
{}
\pysigstopsignatures
\sphinxAtStartPar
Extracts a small set of color values from a Matplotlib colormap.

\sphinxAtStartPar
This function accesses any Matplotlib colormap by name, samples 16 color
entries from it, and returns them as a NumPy array of type float16.
Users can choose any of the built\sphinxhyphen{}in Matplotlib colormaps (e.g.,
“viridis”, “jet”, “inferno”, “Blues”, etc.).
\begin{quote}\begin{description}
\sphinxlineitem{Parameters}
\sphinxAtStartPar
\sphinxstyleliteralstrong{\sphinxupquote{cmap}} (\sphinxstyleliteralemphasis{\sphinxupquote{str}}\sphinxstyleliteralemphasis{\sphinxupquote{, }}\sphinxstyleliteralemphasis{\sphinxupquote{optional}}) \textendash{} Name of the Matplotlib colormap to sample.
Defaults to “Blues”.

\sphinxlineitem{Returns}
\sphinxAtStartPar
\begin{description}
\sphinxlineitem{A (16, 4) array of RGBA color values (float16).}
\sphinxAtStartPar
The array index corresponds to a discrete point along the
specified colormap.

\end{description}


\sphinxlineitem{Return type}
\sphinxAtStartPar
numpy.ndarray

\end{description}\end{quote}

\end{fulllineitems}

\index{FrictionCalc() (in module celeris.utils)@\spxentry{FrictionCalc()}\spxextra{in module celeris.utils}}

\begin{fulllineitems}
\phantomsection\label{\detokenize{celeris:celeris.utils.FrictionCalc}}
\pysigstartsignatures
\pysiglinewithargsret
{\sphinxcode{\sphinxupquote{celeris.utils.}}\sphinxbfcode{\sphinxupquote{FrictionCalc}}}
{\sphinxparam{\DUrole{n}{hu}}\sphinxparamcomma \sphinxparam{\DUrole{n}{hv}}\sphinxparamcomma \sphinxparam{\DUrole{n}{h}}\sphinxparamcomma \sphinxparam{\DUrole{n}{base\_depth}}\sphinxparamcomma \sphinxparam{\DUrole{n}{delta}}\sphinxparamcomma \sphinxparam{\DUrole{n}{isManning}}\sphinxparamcomma \sphinxparam{\DUrole{n}{g}}\sphinxparamcomma \sphinxparam{\DUrole{n}{friction}}}
{}
\pysigstopsignatures
\sphinxAtStartPar
Computes a bottom friction term for shallow\sphinxhyphen{}water or Boussinesq\sphinxhyphen{}type flows.

\sphinxAtStartPar
This function calculates a friction coefficient based on either a constant friction
parameter (\sphinxtitleref{friction}) or Manning’s formula (if \sphinxtitleref{isManning == 1}), and then applies it
to the momentum components. The water depth \sphinxtitleref{h} is scaled by the \sphinxtitleref{base\_depth} for
numerical stability and to avoid singularities near dry cells.

\sphinxAtStartPar
The steps are:
\begin{enumerate}
\sphinxsetlistlabels{\arabic}{enumi}{enumii}{}{.}%
\item {} 
\sphinxAtStartPar
Scale the water depth (\sphinxtitleref{h\_scaled = h / base\_depth}) and compute powers
(\sphinxtitleref{h2 = h\_scaled\textasciicircum{}2}, \sphinxtitleref{h4 = h2\textasciicircum{}2}).

\item {} 
\sphinxAtStartPar
Compute a term \sphinxtitleref{divide\_by\_h2} which further scales friction based on squared depth.

\item {} 
\sphinxAtStartPar
Ensure the local depth \sphinxtitleref{h} is not below a small threshold \sphinxtitleref{delta} to prevent
division by zero.

\item {} 
\sphinxAtStartPar
If \sphinxtitleref{isManning == 1}, convert the \sphinxtitleref{friction} input into Mannings n and compute:
\begin{quote}

\sphinxAtStartPar
f = g * (friction\textasciicircum{}2) * (1 / h\textasciicircum{}(1/3))
\end{quote}

\sphinxAtStartPar
otherwise, keep a constant friction value.

\item {} 
\sphinxAtStartPar
Clamp the friction factor to a maximum of 0.5 as a safety measure.

\item {} 
\sphinxAtStartPar
Multiply by the flow speed (computed from \sphinxtitleref{hu}, \sphinxtitleref{hv}) and the scaling factor
\sphinxtitleref{divide\_by\_h2}.

\end{enumerate}
\begin{quote}\begin{description}
\sphinxlineitem{Parameters}\begin{itemize}
\item {} 
\sphinxAtStartPar
\sphinxstyleliteralstrong{\sphinxupquote{hu}} (\sphinxstyleliteralemphasis{\sphinxupquote{float}}) \textendash{} Momentum in the x\sphinxhyphen{}direction.

\item {} 
\sphinxAtStartPar
\sphinxstyleliteralstrong{\sphinxupquote{hv}} (\sphinxstyleliteralemphasis{\sphinxupquote{float}}) \textendash{} Momentum in the y\sphinxhyphen{}direction.

\item {} 
\sphinxAtStartPar
\sphinxstyleliteralstrong{\sphinxupquote{h}} (\sphinxstyleliteralemphasis{\sphinxupquote{float}}) \textendash{} Local water depth.

\item {} 
\sphinxAtStartPar
\sphinxstyleliteralstrong{\sphinxupquote{base\_depth}} (\sphinxstyleliteralemphasis{\sphinxupquote{float}}) \textendash{} Reference (base) depth for scaling.

\item {} 
\sphinxAtStartPar
\sphinxstyleliteralstrong{\sphinxupquote{delta}} (\sphinxstyleliteralemphasis{\sphinxupquote{float}}) \textendash{} Minimum threshold for water depth to avoid division by zero.

\item {} 
\sphinxAtStartPar
\sphinxstyleliteralstrong{\sphinxupquote{isManning}} (\sphinxstyleliteralemphasis{\sphinxupquote{int}}) \textendash{} If 1, uses Mannings n formulation; otherwise uses a constant friction.

\item {} 
\sphinxAtStartPar
\sphinxstyleliteralstrong{\sphinxupquote{g}} (\sphinxstyleliteralemphasis{\sphinxupquote{float}}) \textendash{} Gravitational acceleration.

\item {} 
\sphinxAtStartPar
\sphinxstyleliteralstrong{\sphinxupquote{friction}} (\sphinxstyleliteralemphasis{\sphinxupquote{float}}) \textendash{} Either a constant friction coefficient or Mannings n value
depending on \sphinxtitleref{isManning}.

\end{itemize}

\sphinxlineitem{Returns}
\sphinxAtStartPar
Computed friction term, capped at 0.5, that will be applied to momentum.

\sphinxlineitem{Return type}
\sphinxAtStartPar
float

\end{description}\end{quote}

\end{fulllineitems}

\index{MinMod() (in module celeris.utils)@\spxentry{MinMod()}\spxextra{in module celeris.utils}}

\begin{fulllineitems}
\phantomsection\label{\detokenize{celeris:celeris.utils.MinMod}}
\pysigstartsignatures
\pysiglinewithargsret
{\sphinxcode{\sphinxupquote{celeris.utils.}}\sphinxbfcode{\sphinxupquote{MinMod}}}
{\sphinxparam{\DUrole{n}{a}}\sphinxparamcomma \sphinxparam{\DUrole{n}{b}}\sphinxparamcomma \sphinxparam{\DUrole{n}{c}}}
{}
\pysigstopsignatures
\sphinxAtStartPar
Computes a simple minmod function of three values.

\sphinxAtStartPar
The minmod function returns:
\begin{itemize}
\item {} 
\sphinxAtStartPar
The minimum among (a, b, c) if all three are positive.

\item {} 
\sphinxAtStartPar
The maximum among (a, b, c) if all three are negative.

\item {} 
\sphinxAtStartPar
Zero otherwise.

\end{itemize}
\begin{quote}\begin{description}
\sphinxlineitem{Parameters}\begin{itemize}
\item {} 
\sphinxAtStartPar
\sphinxstyleliteralstrong{\sphinxupquote{a}} (\sphinxstyleliteralemphasis{\sphinxupquote{float}}) \textendash{} First value.

\item {} 
\sphinxAtStartPar
\sphinxstyleliteralstrong{\sphinxupquote{b}} (\sphinxstyleliteralemphasis{\sphinxupquote{float}}) \textendash{} Second value.

\item {} 
\sphinxAtStartPar
\sphinxstyleliteralstrong{\sphinxupquote{c}} (\sphinxstyleliteralemphasis{\sphinxupquote{float}}) \textendash{} Third value.

\end{itemize}

\sphinxlineitem{Returns}
\sphinxAtStartPar
The minmod result based on the sign of the inputs.

\sphinxlineitem{Return type}
\sphinxAtStartPar
float

\end{description}\end{quote}

\end{fulllineitems}

\index{NumericalFlux() (in module celeris.utils)@\spxentry{NumericalFlux()}\spxextra{in module celeris.utils}}

\begin{fulllineitems}
\phantomsection\label{\detokenize{celeris:celeris.utils.NumericalFlux}}
\pysigstartsignatures
\pysiglinewithargsret
{\sphinxcode{\sphinxupquote{celeris.utils.}}\sphinxbfcode{\sphinxupquote{NumericalFlux}}}
{\sphinxparam{\DUrole{n}{aplus}}\sphinxparamcomma \sphinxparam{\DUrole{n}{aminus}}\sphinxparamcomma \sphinxparam{\DUrole{n}{Fplus}}\sphinxparamcomma \sphinxparam{\DUrole{n}{Fminus}}\sphinxparamcomma \sphinxparam{\DUrole{n}{Udifference}}}
{}
\pysigstopsignatures
\sphinxAtStartPar
Computes a wave\sphinxhyphen{}speed\sphinxhyphen{}based numerical flux between two adjacent cells.

\sphinxAtStartPar
This function calculates the flux across a cell interface using the wave speeds
\sphinxtitleref{aplus} (maximum positive speed) and \sphinxtitleref{aminus} (maximum negative speed) along with
flux values from the “plus” and “minus” sides (\sphinxtitleref{Fplus}, \sphinxtitleref{Fminus}) and the state
difference (\sphinxtitleref{Udifference}). If the wave speeds cancel each other out
(\sphinxtitleref{aplus \sphinxhyphen{} aminus == 0.0}), the flux is set to zero.

\sphinxAtStartPar
The formula implemented is:
\begin{quote}

\sphinxAtStartPar
flux = (aplus * Fminus \sphinxhyphen{} aminus * Fplus + (aplus * aminus) * Udifference) / (aplus \sphinxhyphen{} aminus)
\end{quote}
\begin{quote}\begin{description}
\sphinxlineitem{Parameters}\begin{itemize}
\item {} 
\sphinxAtStartPar
\sphinxstyleliteralstrong{\sphinxupquote{aplus}} (\sphinxstyleliteralemphasis{\sphinxupquote{float}}) \textendash{} Maximum positive wave speed at the cell interface.

\item {} 
\sphinxAtStartPar
\sphinxstyleliteralstrong{\sphinxupquote{aminus}} (\sphinxstyleliteralemphasis{\sphinxupquote{float}}) \textendash{} Maximum negative wave speed at the cell interface.

\item {} 
\sphinxAtStartPar
\sphinxstyleliteralstrong{\sphinxupquote{Fplus}} (\sphinxstyleliteralemphasis{\sphinxupquote{float}}) \textendash{} Flux contribution from the “plus” (right) side.

\item {} 
\sphinxAtStartPar
\sphinxstyleliteralstrong{\sphinxupquote{Fminus}} (\sphinxstyleliteralemphasis{\sphinxupquote{float}}) \textendash{} Flux contribution from the “minus” (left) side.

\item {} 
\sphinxAtStartPar
\sphinxstyleliteralstrong{\sphinxupquote{Udifference}} (\sphinxstyleliteralemphasis{\sphinxupquote{float}}) \textendash{} Difference in the conserved variable across the interface
(e.g., U\_right \sphinxhyphen{} U\_left).

\end{itemize}

\sphinxlineitem{Returns}
\sphinxAtStartPar
The computed numerical flux. Returns 0.0 if \sphinxtitleref{(aplus \sphinxhyphen{} aminus) == 0.0}.

\sphinxlineitem{Return type}
\sphinxAtStartPar
float

\end{description}\end{quote}

\end{fulllineitems}

\index{Reconstruct() (in module celeris.utils)@\spxentry{Reconstruct()}\spxextra{in module celeris.utils}}

\begin{fulllineitems}
\phantomsection\label{\detokenize{celeris:celeris.utils.Reconstruct}}
\pysigstartsignatures
\pysiglinewithargsret
{\sphinxcode{\sphinxupquote{celeris.utils.}}\sphinxbfcode{\sphinxupquote{Reconstruct}}}
{\sphinxparam{\DUrole{n}{west}}\sphinxparamcomma \sphinxparam{\DUrole{n}{here}}\sphinxparamcomma \sphinxparam{\DUrole{n}{east}}\sphinxparamcomma \sphinxparam{\DUrole{n}{TWO\_THETAc}}}
{}
\pysigstopsignatures
\sphinxAtStartPar
Performs a piecewise linear reconstruction of a variable using a generalized minmod limiter.

\sphinxAtStartPar
This function takes three consecutive cell\sphinxhyphen{}centered values (\sphinxtitleref{west}, \sphinxtitleref{here}, and \sphinxtitleref{east})
along with a limiter parameter (\sphinxtitleref{TWO\_THETAc}) and returns two reconstructed interface
values at the current cell interfaces (left/right or west/east edges).

\sphinxAtStartPar
The reconstruction logic:
\begin{itemize}
\item {} 
\sphinxAtStartPar
Computes slopes (z1, z2, z3) that scale differences between neighboring cells.

\item {} 
\sphinxAtStartPar
Finds the minimum among those slopes (when all have the same sign) or zero otherwise.

\item {} 
\sphinxAtStartPar
Applies a factor of 0.25 to that minimum slope to limit the reconstruction (i.e.,
controlling oscillations).

\item {} 
\sphinxAtStartPar
Returns the reconstructed values at the left (west) and right (east) edges of
the current cell.

\end{itemize}
\begin{quote}\begin{description}
\sphinxlineitem{Parameters}\begin{itemize}
\item {} 
\sphinxAtStartPar
\sphinxstyleliteralstrong{\sphinxupquote{west}} (\sphinxstyleliteralemphasis{\sphinxupquote{float}}) \textendash{} Value of the variable at the cell immediately to the left (j\sphinxhyphen{}1).

\item {} 
\sphinxAtStartPar
\sphinxstyleliteralstrong{\sphinxupquote{here}} (\sphinxstyleliteralemphasis{\sphinxupquote{float}}) \textendash{} Value of the variable at the current cell (j).

\item {} 
\sphinxAtStartPar
\sphinxstyleliteralstrong{\sphinxupquote{east}} (\sphinxstyleliteralemphasis{\sphinxupquote{float}}) \textendash{} Value of the variable at the cell immediately to the right (j+1).

\item {} 
\sphinxAtStartPar
\sphinxstyleliteralstrong{\sphinxupquote{TWO\_THETAc}} (\sphinxstyleliteralemphasis{\sphinxupquote{float}}) \textendash{} Limiter parameter, typically 2 * theta, where theta is in range {[}1, 2{]}
for generalized minmod\sphinxhyphen{}type limiters.

\end{itemize}

\sphinxlineitem{Returns}
\sphinxAtStartPar
A 2\sphinxhyphen{}component vector representing the reconstructed value at:
\sphinxhyphen{} {[}0{]}: The left (west) interface of the current cell.
\sphinxhyphen{} {[}1{]}: The right (east) interface of the current cell.

\sphinxlineitem{Return type}
\sphinxAtStartPar
ti.types.vector(2, float)

\end{description}\end{quote}

\end{fulllineitems}

\index{ScalarAntiDissipation() (in module celeris.utils)@\spxentry{ScalarAntiDissipation()}\spxextra{in module celeris.utils}}

\begin{fulllineitems}
\phantomsection\label{\detokenize{celeris:celeris.utils.ScalarAntiDissipation}}
\pysigstartsignatures
\pysiglinewithargsret
{\sphinxcode{\sphinxupquote{celeris.utils.}}\sphinxbfcode{\sphinxupquote{ScalarAntiDissipation}}}
{\sphinxparam{\DUrole{n}{uplus}}\sphinxparamcomma \sphinxparam{\DUrole{n}{uminus}}\sphinxparamcomma \sphinxparam{\DUrole{n}{aplus}}\sphinxparamcomma \sphinxparam{\DUrole{n}{aminus}}\sphinxparamcomma \sphinxparam{\DUrole{n}{epsilon}}}
{}
\pysigstopsignatures
\sphinxAtStartPar
Computes an anti\sphinxhyphen{}dissipation factor based on local wave speeds and state magnitudes.

\sphinxAtStartPar
This function calculates a dimensionless ratio \sphinxtitleref{R} that adjusts numerical dissipation
in a flux\sphinxhyphen{}based scheme. The ratio depends on the maximum wave speed (\sphinxtitleref{aplus} or \sphinxtitleref{aminus})
and the magnitudes of the state variables \sphinxtitleref{uplus} and \sphinxtitleref{uminus}. If both wave speeds are
non\sphinxhyphen{}zero, a local “Froude\sphinxhyphen{}like” number is formed by dividing the larger magnitude of
\sphinxtitleref{uplus} or \sphinxtitleref{uminus} by the respective wave speed. This number is then augmented by a
small threshold \sphinxtitleref{epsilon} to yield a final ratio between 0 and 1. If either wave speed
is zero, the ratio is set to \sphinxtitleref{epsilon}.
\begin{quote}\begin{description}
\sphinxlineitem{Parameters}\begin{itemize}
\item {} 
\sphinxAtStartPar
\sphinxstyleliteralstrong{\sphinxupquote{uplus}} (\sphinxstyleliteralemphasis{\sphinxupquote{float}}) \textendash{} The “plus” state or velocity component.

\item {} 
\sphinxAtStartPar
\sphinxstyleliteralstrong{\sphinxupquote{uminus}} (\sphinxstyleliteralemphasis{\sphinxupquote{float}}) \textendash{} The “minus” state or velocity component.

\item {} 
\sphinxAtStartPar
\sphinxstyleliteralstrong{\sphinxupquote{aplus}} (\sphinxstyleliteralemphasis{\sphinxupquote{float}}) \textendash{} Positive wave speed at the cell interface.

\item {} 
\sphinxAtStartPar
\sphinxstyleliteralstrong{\sphinxupquote{aminus}} (\sphinxstyleliteralemphasis{\sphinxupquote{float}}) \textendash{} Negative wave speed at the cell interface.

\item {} 
\sphinxAtStartPar
\sphinxstyleliteralstrong{\sphinxupquote{epsilon}} (\sphinxstyleliteralemphasis{\sphinxupquote{float}}) \textendash{} Small threshold to prevent division by zero or extreme values.

\end{itemize}

\sphinxlineitem{Returns}
\sphinxAtStartPar
Anti\sphinxhyphen{}dissipation ratio \sphinxtitleref{R}. A value near 1 indicates lower numerical dissipation,
while a value near 0 increases dissipation. Defaults to 0 if neither condition applies.

\sphinxlineitem{Return type}
\sphinxAtStartPar
float

\end{description}\end{quote}

\end{fulllineitems}

\index{celeris\_matplotlib() (in module celeris.utils)@\spxentry{celeris\_matplotlib()}\spxextra{in module celeris.utils}}

\begin{fulllineitems}
\phantomsection\label{\detokenize{celeris:celeris.utils.celeris_matplotlib}}
\pysigstartsignatures
\pysiglinewithargsret
{\sphinxcode{\sphinxupquote{celeris.utils.}}\sphinxbfcode{\sphinxupquote{celeris\_matplotlib}}}
{\sphinxparam{\DUrole{n}{water}\DUrole{o}{=}\DUrole{default_value}{\textquotesingle{}seismic\textquotesingle{}}}\sphinxparamcomma \sphinxparam{\DUrole{n}{land}\DUrole{o}{=}\DUrole{default_value}{\textquotesingle{}terrain\textquotesingle{}}}\sphinxparamcomma \sphinxparam{\DUrole{n}{sediment}\DUrole{o}{=}\DUrole{default_value}{\textquotesingle{}default\textquotesingle{}}}\sphinxparamcomma \sphinxparam{\DUrole{n}{SedTrans}\DUrole{o}{=}\DUrole{default_value}{False}}}
{}
\pysigstopsignatures
\sphinxAtStartPar
Creates a customized matplotlib color map for visualizing water, land/topography,
and optionally sediment transport.

\sphinxAtStartPar
This function merges three main color segments:
\begin{enumerate}
\sphinxsetlistlabels{\arabic}{enumi}{enumii}{}{.}%
\item {} 
\sphinxAtStartPar
\sphinxstylestrong{Water}: Ranges from 0 to 0.75 or 0 to 0.5 (depending on sediment usage).

\item {} 
\sphinxAtStartPar
\sphinxstylestrong{Sediment} (optional): Placed between the water and land segments if \sphinxtitleref{SedTrans} is True.

\item {} \begin{description}
\sphinxlineitem{\sphinxstylestrong{Land/Topography}: Assigned to the higher range of the color bar (e.g., 0.75 \sphinxhyphen{} 1 or 0.75 \sphinxhyphen{} 1}
\sphinxAtStartPar
when \sphinxtitleref{SedTrans} is False, and 0.75 \sphinxhyphen{} 1 when \sphinxtitleref{SedTrans} is True).

\end{description}

\end{enumerate}
\begin{quote}\begin{description}
\sphinxlineitem{Parameters}\begin{itemize}
\item {} 
\sphinxAtStartPar
\sphinxstyleliteralstrong{\sphinxupquote{water}} (\sphinxstyleliteralemphasis{\sphinxupquote{str}}\sphinxstyleliteralemphasis{\sphinxupquote{, }}\sphinxstyleliteralemphasis{\sphinxupquote{optional}}) \textendash{} Name of the colormap to use for water (default “seismic”).

\item {} 
\sphinxAtStartPar
\sphinxstyleliteralstrong{\sphinxupquote{land}} (\sphinxstyleliteralemphasis{\sphinxupquote{str}}\sphinxstyleliteralemphasis{\sphinxupquote{, }}\sphinxstyleliteralemphasis{\sphinxupquote{optional}}) \textendash{} Name of the colormap to use for land/topo (default “terrain”).

\item {} 
\sphinxAtStartPar
\sphinxstyleliteralstrong{\sphinxupquote{sediment}} (\sphinxstyleliteralemphasis{\sphinxupquote{str}}\sphinxstyleliteralemphasis{\sphinxupquote{, }}\sphinxstyleliteralemphasis{\sphinxupquote{optional}}) \textendash{} Name of the colormap to use for sediment. If “default”,
a hard\sphinxhyphen{}coded set of color stops (skyblue, tan, peru, saddlebrown) is used.
Otherwise, a user\sphinxhyphen{}specified colormap is merged (default “default”).

\item {} 
\sphinxAtStartPar
\sphinxstyleliteralstrong{\sphinxupquote{SedTrans}} (\sphinxstyleliteralemphasis{\sphinxupquote{bool}}\sphinxstyleliteralemphasis{\sphinxupquote{, }}\sphinxstyleliteralemphasis{\sphinxupquote{optional}}) \textendash{} Indicates whether sediment transport is active.
If True, the colormap includes an additional segment for sediment;
if False, only water and land segments are used (default False).

\end{itemize}

\sphinxlineitem{Returns}
\sphinxAtStartPar
A single merged colormap with the
specified segments for water, (optionally) sediment, and land.

\sphinxlineitem{Return type}
\sphinxAtStartPar
matplotlib.colors.LinearSegmentedColormap

\end{description}\end{quote}
\subsubsection*{Example}

\begin{sphinxVerbatim}[commandchars=\\\{\}]
\PYG{g+gp}{\PYGZgt{}\PYGZgt{}\PYGZgt{} }\PYG{n}{cmap\PYGZus{}no\PYGZus{}sed} \PYG{o}{=} \PYG{n}{celeris\PYGZus{}matplotlib}\PYG{p}{(}\PYG{n}{SedTrans}\PYG{o}{=}\PYG{k+kc}{False}\PYG{p}{)}
\PYG{g+gp}{\PYGZgt{}\PYGZgt{}\PYGZgt{} }\PYG{n}{cmap\PYGZus{}sed} \PYG{o}{=} \PYG{n}{celeris\PYGZus{}matplotlib}\PYG{p}{(}\PYG{n}{water}\PYG{o}{=}\PYG{l+s+s2}{\PYGZdq{}}\PYG{l+s+s2}{Blues}\PYG{l+s+s2}{\PYGZdq{}}\PYG{p}{,} \PYG{n}{sediment}\PYG{o}{=}\PYG{l+s+s2}{\PYGZdq{}}\PYG{l+s+s2}{Reds}\PYG{l+s+s2}{\PYGZdq{}}\PYG{p}{,} \PYG{n}{SedTrans}\PYG{o}{=}\PYG{k+kc}{True}\PYG{p}{)}
\end{sphinxVerbatim}

\end{fulllineitems}

\index{celeris\_waves() (in module celeris.utils)@\spxentry{celeris\_waves()}\spxextra{in module celeris.utils}}

\begin{fulllineitems}
\phantomsection\label{\detokenize{celeris:celeris.utils.celeris_waves}}
\pysigstartsignatures
\pysiglinewithargsret
{\sphinxcode{\sphinxupquote{celeris.utils.}}\sphinxbfcode{\sphinxupquote{celeris\_waves}}}
{}
{}
\pysigstopsignatures
\sphinxAtStartPar
Creates a custom color map (colormap) designed to represent realistic sea water gradients.

\sphinxAtStartPar
This function defines a series of color stops spanning blues, greens, and yellows,
which can be used to visualize water\sphinxhyphen{}related data (e.g., wave heights or velocities).
The color map transitions from light blues (representing shallower or clearer water)
through darker blues/greens, and finally into yellowish tones that can highlight
areas of foam or breaking waves.
\begin{quote}\begin{description}
\sphinxlineitem{Returns}
\sphinxAtStartPar
A color map object suitable for
use with matplotlib plotting functions or other visualization frameworks.

\sphinxlineitem{Return type}
\sphinxAtStartPar
matplotlib.colors.LinearSegmentedColormap

\end{description}\end{quote}

\end{fulllineitems}

\index{cosh() (in module celeris.utils)@\spxentry{cosh()}\spxextra{in module celeris.utils}}

\begin{fulllineitems}
\phantomsection\label{\detokenize{celeris:celeris.utils.cosh}}
\pysigstartsignatures
\pysiglinewithargsret
{\sphinxcode{\sphinxupquote{celeris.utils.}}\sphinxbfcode{\sphinxupquote{cosh}}}
{\sphinxparam{\DUrole{n}{x}}}
{}
\pysigstopsignatures
\sphinxAtStartPar
Returns the hyperbolic cosine of \sphinxtitleref{x}.
\begin{description}
\sphinxlineitem{The hyperbolic cosine is defined as:}
\sphinxAtStartPar
cosh(x) = (e\textasciicircum{}x + e\textasciicircum{}(\sphinxhyphen{}x)) / 2

\end{description}
\begin{quote}\begin{description}
\sphinxlineitem{Parameters}
\sphinxAtStartPar
\sphinxstyleliteralstrong{\sphinxupquote{x}} (\sphinxstyleliteralemphasis{\sphinxupquote{float}}) \textendash{} The input value.

\sphinxlineitem{Returns}
\sphinxAtStartPar
The value of the hyperbolic cosine of \sphinxtitleref{x}.

\sphinxlineitem{Return type}
\sphinxAtStartPar
float

\end{description}\end{quote}

\end{fulllineitems}

\index{sineWave() (in module celeris.utils)@\spxentry{sineWave()}\spxextra{in module celeris.utils}}

\begin{fulllineitems}
\phantomsection\label{\detokenize{celeris:celeris.utils.sineWave}}
\pysigstartsignatures
\pysiglinewithargsret
{\sphinxcode{\sphinxupquote{celeris.utils.}}\sphinxbfcode{\sphinxupquote{sineWave}}}
{\sphinxparam{\DUrole{n}{x}}\sphinxparamcomma \sphinxparam{\DUrole{n}{y}}\sphinxparamcomma \sphinxparam{\DUrole{n}{t}}\sphinxparamcomma \sphinxparam{\DUrole{n}{d}}\sphinxparamcomma \sphinxparam{\DUrole{n}{amplitude}}\sphinxparamcomma \sphinxparam{\DUrole{n}{period}}\sphinxparamcomma \sphinxparam{\DUrole{n}{theta}}\sphinxparamcomma \sphinxparam{\DUrole{n}{phase}}\sphinxparamcomma \sphinxparam{\DUrole{n}{g}}\sphinxparamcomma \sphinxparam{\DUrole{n}{wave\_type}}}
{}
\pysigstopsignatures
\sphinxAtStartPar
Computes a sine wave (and related momentum terms) at a given point (x, y) and time t.

\sphinxAtStartPar
This function uses a dispersion relation and hyperbolic tangent of wave depth
to approximate wave number (k) and phase speed (c). It then calculates free\sphinxhyphen{}surface
elevation (eta) and horizontal momentum components(hu, hv) based on wave parameters. An optional
decay term is applied for certain wave types.
\begin{quote}\begin{description}
\sphinxlineitem{Parameters}\begin{itemize}
\item {} 
\sphinxAtStartPar
\sphinxstyleliteralstrong{\sphinxupquote{x}} (\sphinxstyleliteralemphasis{\sphinxupquote{float}}) \textendash{} x\sphinxhyphen{}coordinate where the wave is evaluated.

\item {} 
\sphinxAtStartPar
\sphinxstyleliteralstrong{\sphinxupquote{y}} (\sphinxstyleliteralemphasis{\sphinxupquote{float}}) \textendash{} y\sphinxhyphen{}coordinate where the wave is evaluated.

\item {} 
\sphinxAtStartPar
\sphinxstyleliteralstrong{\sphinxupquote{t}} (\sphinxstyleliteralemphasis{\sphinxupquote{float}}) \textendash{} Current time in the simulation.

\item {} 
\sphinxAtStartPar
\sphinxstyleliteralstrong{\sphinxupquote{d}} (\sphinxstyleliteralemphasis{\sphinxupquote{float}}) \textendash{} Local water depth.

\item {} 
\sphinxAtStartPar
\sphinxstyleliteralstrong{\sphinxupquote{amplitude}} (\sphinxstyleliteralemphasis{\sphinxupquote{float}}) \textendash{} Wave amplitude.

\item {} 
\sphinxAtStartPar
\sphinxstyleliteralstrong{\sphinxupquote{period}} (\sphinxstyleliteralemphasis{\sphinxupquote{float}}) \textendash{} Wave period.

\item {} 
\sphinxAtStartPar
\sphinxstyleliteralstrong{\sphinxupquote{theta}} (\sphinxstyleliteralemphasis{\sphinxupquote{float}}) \textendash{} Wave propagation angle in radians.

\item {} 
\sphinxAtStartPar
\sphinxstyleliteralstrong{\sphinxupquote{phase}} (\sphinxstyleliteralemphasis{\sphinxupquote{float}}) \textendash{} Additional phase offset.

\item {} 
\sphinxAtStartPar
\sphinxstyleliteralstrong{\sphinxupquote{g}} (\sphinxstyleliteralemphasis{\sphinxupquote{float}}) \textendash{} Gravitational acceleration.

\item {} 
\sphinxAtStartPar
\sphinxstyleliteralstrong{\sphinxupquote{wave\_type}} (\sphinxstyleliteralemphasis{\sphinxupquote{int}}) \textendash{} Wave type indicator. If \sphinxtitleref{wave\_type == 2}, a decay multiplier
is applied to the wave for demonstration/limiting purposes.

\end{itemize}

\sphinxlineitem{Returns}
\sphinxAtStartPar
\begin{description}
\sphinxlineitem{A 3\sphinxhyphen{}component vector:}\begin{itemize}
\item {} 
\sphinxAtStartPar
\sphinxstylestrong{eta}: Free\sphinxhyphen{}surface elevation (wave height) above still water level.

\item {} 
\sphinxAtStartPar
\sphinxstylestrong{hu}: Momentum in the x\sphinxhyphen{}direction (wave speed * wave height).

\item {} 
\sphinxAtStartPar
\sphinxstylestrong{hv}: Momentum in the y\sphinxhyphen{}direction (wave speed * wave height).

\end{itemize}

\end{description}


\sphinxlineitem{Return type}
\sphinxAtStartPar
ti.Vector({[}float, float, float{]})

\end{description}\end{quote}

\begin{sphinxadmonition}{note}{Note:}\begin{itemize}
\item {} 
\sphinxAtStartPar
The calculation for wave number \sphinxtitleref{k} uses a simplified relationship assuming
linear wave theory with a hyperbolic tangent term for finite depth.

\item {} 
\sphinxAtStartPar
The term \sphinxtitleref{ti.min(1.0, t / period)} is used to gradually ramp up the wave
from zero at t=0 (avoid sudden wave onset).

\item {} 
\sphinxAtStartPar
If \sphinxtitleref{wave\_type == 2}, an additional decay factor is applied as \sphinxtitleref{t} approaches
\sphinxtitleref{num\_waves * period} (here \sphinxtitleref{num\_waves} is hard\sphinxhyphen{}coded to 4 in the example). For a transient pulse

\item {} 
\sphinxAtStartPar
The returned \sphinxtitleref{hu} and \sphinxtitleref{hv} are computed as a fraction of \sphinxtitleref{g * eta / (c * k) * tanh(k * d)},
scaled by the direction cosines \sphinxtitleref{(cos(theta), sin(theta))}.

\end{itemize}
\end{sphinxadmonition}

\end{fulllineitems}



\subsubsection{Current Version}
\label{\detokenize{celeris:module-celeris}}\label{\detokenize{celeris:current-version}}\index{module@\spxentry{module}!celeris@\spxentry{celeris}}\index{celeris@\spxentry{celeris}!module@\spxentry{module}}
\sphinxAtStartPar
CelerisAi is a Python\sphinxhyphen{}Taichi\sphinxhyphen{}based software designed for nearshore wave modeling. This solver offers high\sphinxhyphen{}performance simulations on various hardware platforms and seamlessly integrates with machine learning and artificial intelligence environments. The solver leverages the flexibility of Python for customization and interoperability, while Taichi’s high\sphinxhyphen{}performance parallel programming capabilities ensure efficient computations.

\sphinxAtStartPar
This package contains modules for running domain\sphinxhyphen{}specific simulations, solving
problems, and providing general utilities:
\begin{itemize}
\item {} 
\sphinxAtStartPar
\sphinxstylestrong{domain.py}: Contains classes and functions defining the problem domain.

\item {} 
\sphinxAtStartPar
\sphinxstylestrong{runner.py}: Manages the execution flow, including initialization and orchestration.

\item {} 
\sphinxAtStartPar
\sphinxstylestrong{solver.py}: Implements the core solver logic for the defined domain and mathematical models.

\item {} 
\sphinxAtStartPar
\sphinxstylestrong{utils.py}: A collection of helper utilities used across the package.

\end{itemize}
\subsubsection*{Example}

\begin{sphinxVerbatim}[commandchars=\\\{\}]
\PYG{g+gp}{\PYGZgt{}\PYGZgt{}\PYGZgt{} }\PYG{k+kn}{from}\PYG{+w}{ }\PYG{n+nn}{celeris}\PYG{+w}{ }\PYG{k+kn}{import} \PYG{n}{domain}\PYG{p}{,} \PYG{n}{runner}\PYG{p}{,} \PYG{n}{solver}\PYG{p}{,} \PYG{n}{utils}
\PYG{g+gp}{\PYGZgt{}\PYGZgt{}\PYGZgt{} }\PYG{n}{domain\PYGZus{}obj} \PYG{o}{=} \PYG{n}{domain}\PYG{o}{.}\PYG{n}{Domain}\PYG{p}{(}\PYG{n}{params}\PYG{o}{=}\PYG{p}{\PYGZob{}}\PYG{o}{.}\PYG{o}{.}\PYG{o}{.}\PYG{p}{\PYGZcb{}}\PYG{p}{)}
\PYG{g+gp}{\PYGZgt{}\PYGZgt{}\PYGZgt{} }\PYG{n}{solver\PYGZus{}obj} \PYG{o}{=} \PYG{n}{solver}\PYG{o}{.}\PYG{n}{Solver}\PYG{p}{(}\PYG{n}{domain}\PYG{o}{=}\PYG{n}{domain\PYGZus{}obj}\PYG{p}{)}
\PYG{g+gp}{\PYGZgt{}\PYGZgt{}\PYGZgt{} }\PYG{n}{runner\PYGZus{}obj} \PYG{o}{=} \PYG{n}{runner}\PYG{o}{.}\PYG{n}{Evolve}\PYG{p}{(}\PYG{n}{solver} \PYG{o}{=} \PYG{n}{solver\PYGZus{}obj}\PYG{p}{)}
\PYG{g+gp}{\PYGZgt{}\PYGZgt{}\PYGZgt{} }\PYG{n}{runner\PYGZus{}obj}\PYG{o}{.}\PYG{n}{Evolve\PYGZus{}Headless}\PYG{p}{(}\PYG{p}{)}
\end{sphinxVerbatim}
\index{\_\_version\_\_ (in module celeris)@\spxentry{\_\_version\_\_}\spxextra{in module celeris}}

\begin{fulllineitems}
\phantomsection\label{\detokenize{celeris:celeris.__version__}}
\pysigstartsignatures
\pysigline
{\sphinxcode{\sphinxupquote{celeris.}}\sphinxbfcode{\sphinxupquote{\_\_version\_\_}}}
\pysigstopsignatures
\sphinxAtStartPar
The current version of the CelerisAi.
\begin{quote}\begin{description}
\sphinxlineitem{Type}
\sphinxAtStartPar
str

\end{description}\end{quote}

\end{fulllineitems}


\begin{sphinxthebibliography}{Tavakkol}
\bibitem[TavakkolLynett2020]{introduction:tavakkollynett2020}
\sphinxAtStartPar
Tavakkol, S., \& Lynett, P. (2020). Celeris Base: An interactive and immersive Boussinesq\sphinxhyphen{}type nearshore wave simulation software. Computer Physics Communications, 248, 106966. \sphinxurl{https://doi.org/10.1016/j.cpc.2019.106966}
\bibitem[MadseSorensen1992]{introduction:madsesorensen1992}
\sphinxAtStartPar
Madsen, P. A., \& Sørensen, O. R. (1992). A new form of the Boussinesq equations with improved linear dispersion characteristics. Part 2. A slowly\sphinxhyphen{}varying bathymetry. Coastal Engineering, 18(3\textendash{}4), 183\textendash{}204. \sphinxurl{https://doi.org/10.1016/0378-3839(92)90019-Q}
\end{sphinxthebibliography}


\renewcommand{\indexname}{Python Module Index}
\begin{sphinxtheindex}
\let\bigletter\sphinxstyleindexlettergroup
\bigletter{c}
\item\relax\sphinxstyleindexentry{celeris}\sphinxstyleindexpageref{celeris:\detokenize{module-celeris}}
\item\relax\sphinxstyleindexentry{celeris.domain}\sphinxstyleindexpageref{celeris:\detokenize{module-celeris.domain}}
\item\relax\sphinxstyleindexentry{celeris.runner}\sphinxstyleindexpageref{celeris:\detokenize{module-celeris.runner}}
\item\relax\sphinxstyleindexentry{celeris.solver}\sphinxstyleindexpageref{celeris:\detokenize{module-celeris.solver}}
\item\relax\sphinxstyleindexentry{celeris.utils}\sphinxstyleindexpageref{celeris:\detokenize{module-celeris.utils}}
\end{sphinxtheindex}

\renewcommand{\indexname}{Index}
\printindex
\end{document}